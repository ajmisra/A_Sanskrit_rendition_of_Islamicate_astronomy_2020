\setcounter{footnote}{0}
\renewcommand{\thefootnote}{[\roman{footnote}]} % for distinct numbering of Sanskrit edition (chapter vi) footnotes (starting at [i])
\setlength{\footnotesep}{\baselineskip} % for space between footnotes

{\hindifont\selectfont
\OnehalfSpacing
\large
\te{\normalmarginpar\marginnote{$\lceil$~{\footnotesize f.\thinspace 20r:\thinspace16~Kh}}}\raisebox{.5ex}{$\lceil$}{\bfseries || atha .sa.s.thaadhyaayaspa.s.takraanti{\dev{j~naa}}nam\te{\footnote{~Looking at the orthography of the other chapter-titles (in Part~II of Kh, \vid\ \S~\ref{chapter_title_comparision_persian_sanskrit}), 
\tsans{.sa.s.thaadhye spa.s.takraanti{\tsnb{ज्ञा}}nam} %
would be more consistent than \tsans{.sa.s.thaadhyaayaspa.s.takraanti{\tsnb{ज्ञा}}nam}.
Nevertheless, I maintain the reading attested in Kh in the absence of a second manuscript witness. Besides, the locative sense (\textit{adhi\-karaṇa}) of the modifier \tsans{.sa.s.tha-[.a]dhyaaya} in the \textit{tatpuruṣa} compound \tsans{.sa.s.tha-[.a]dhyaaya-spa.s.ta-kraanti-{\tsnb{ज्ञा}}nam}\,$_\text{\acrshort{nominative}-\acrshort{singular}}$ is identical to the use of the
prepositional phrase \tsans{.sa.s.tha-[.a]dhyaaye}\,$_\text{\acrshort{locative}-\acrshort{singular}}$ in 
the sentence \tsans{.sa.s.tha-[.a]dhye spa.s.ta-kraanti-{\tsnb{ज्ञा}}nam}\,$_\text{\acrshort{nominative}-\acrshort{singular}}$.}}||}
\\[.325\baselineskip]

\te{\noindent\reversemarginpar\marginnote{\hypertarget{Spass1}{[1]}}}%
khagasya baa.no.anyataraapama.h\te{\footnote{~\tsans{baa.no.anyatarapama.h}~{$\Big ]$}~\tsans{baa.nenyatarapama.h}\enskip Kh. The conjoined word \tsans{baa.nenyatarapama.h} in Kh can be meaninfully segemnted as \tsans{baa.ne}\,$_\text{\acrshort{locative}-\acrshort{singular}}$  + \tsans{anyatara-apama.h}\,$_\text{\acrshort{nominative}-\acrshort{singular}}$. However, this reading (the second noun \textit{in} the first) is contextually and semantically incoherent. The syntactic structure of the Sanskrit text mimics the syntax of the Persian text in passage~[\hyperlink{Ppass1}{1}], \vid\ \S~\ref{language_content_zij_sindhu_chapter_six}, remark~\ref{subject_fronting_passage_1} on page~\pageref{subject_fronting_passage_1}. This suggests that the emendation \tsans{baa.no}\,$_\text{\acrshort{nominative}-\acrshort{singular}}$ is better suited than \tsans{baa.ne}\,$_\text{\acrshort{locative}-\acrshort{singular}}$ as it agrees with \tsans{apama.h}\,$_\text{\acrshort{nominative}-\acrshort{singular}}$ in the subject-fronted noun phrase \tsans{khagasya-baana.h-.anyatara-apama.h puna.h}.}} puna-\\ ryadaa dvaya.m vaikadi"si sthita.m bhavet || \\
tadaa tayo.h sa.myutiranyathaantara.m\\ sphu.taapamaa.m"saakhya\te{\footnote{~\tsans{\selip maa.m"saakhya}~{$\Big ]$}~\tsans{\selip maa.m"saakhyakhya}\enskip Kh, dittography of the second \tsans{khya}.}}
ihocyate svadik || 1 || \te{\normalmarginpar\marginnote{\footnotesize \textit{vaṃśasthavila}}}
\medskip

\te{\noindent\reversemarginpar\marginnote{\hypertarget{Spass2}{[2]}}}%
sphu.taapamaa"nkasi{\dev{~nji}}nii sabhatrayadyujiivayaa || \\
nihanyate .adhariik.rtaa sphu.taapamajyakaa bhavet || 2 || \te{\normalmarginpar\marginnote{\footnotesize \textit{pramāṇikā}}}
\label{sanskrit_meter_typography_example}
\\[.65\baselineskip]

\te{\noindent\reversemarginpar\marginnote{\hypertarget{Spass3}{[3]}}}%
paramakraantiko.tijyaa sphu.takraantya"nkajiivayaa || \\
hataanya\-kraantiko.tijyaaptaa syaatspa.s.taapamajyakaa || 3 ||
\te{\normalmarginpar\marginnote{\footnotesize \textit{anuṣṭubh}}}
\medskip

\te{\noindent\reversemarginpar\marginnote{\hypertarget{Spass4}{[4]}}}%
ki.mvaa paramakraantijyaako.s.thikebhya.h\te{\footnote{~An apposite reading of the compound \tsans{parama-kraanti-jyaa-ko.s.thikebhya.h} should be \tsans{parama-kraanti-}{\normalsize <}\tsans{ko.ti}{\normalsize >}\tsans{jyaa-ko.s.thakebhya.h}. This would agree with an identical construction in passage~[\hyperlink{Spass6}{6}] construed in the same mathematical context. There is, however, no visible evidence (\eg interlinear lacunae or scribal corrections) on f.\thinspace 20r:\thinspace 20~Kh to suggest an omission. Therefore, I leave the attested reading unaltered in Sanskrit but include an emendation in my English translation.}} sphu.takraantya"nkajyayaa gu.nitaphalamutthaaya \\dvitiiyakraantiko.tijyayaa bhajellabdha.m spa.s.takraantijyaa syaat ||
\medskip

\te{\noindent\reversemarginpar\marginnote{\hypertarget{Spass5}{[5]}}}%
atha ca yadi khagasya  ba.no na syaattadaa 
tasya kraantireva spa.s.takraantirbhavet || 
}

%%%%%%%%%%%%%%%%%%%%%%%%%%%%%%%%%
\clearpage\phantomsection
%%%%%%%%%%%%%%%%%%%%%%%%%%%%%%%%%

\textbf{Now, the \gls{knowledge} (\textit{jñāna}) of the \gls{true_declination} (\textit{spaṣṭa-krānti}) in the sixth chapter.}
\\[.325\baselineskip]

\noindent\reversemarginpar\marginnote{\hypertarget{SEpass1}{[1]}}%
{[Given]} the \gls{latitude_celestial_object} (\textit{khagasya bāṇa}), [and] again, the \gls{other_declination} (\textit{anyatara-apama}) [\ie the second declination]: if indeed both should be situated in \gls{one_direction} (\textit{eka-diś}), then [we take] the \gls{sum} (\textit{saṃyuti}) of both of them; otherwise, [we take their] \gls{difference} (\textit{antara}). [The result] is known as the \gls{share_true_declination} (\textit{sphuṭa-apama-aṃśa}). Here, [it is] said to be [in] its \gls{own_direction} (\textit{sva-diś}). 1 
\medskip

\noindent\reversemarginpar\marginnote{\hypertarget{SEpass2}{[2]}}%
The \gls{Sine_curve_true_declination} (\textit{sphuṭa-apama-aṅka-siñjinī}), \gls{having_been_lowered} (\textit{adharī-kṛtā}), \glslink{multiplication}{is multiplied} (\textit{ni-hanyate}) by the \gls{day_Sine_increased_by_three_signs} (\textit{sa-bha-traya-dyujīvā}) [\ie Cosine of the first declination of the longitude increased by 90\degree]. [The result] will be the \gls{Sine_true_declination} (\textit{sphuṭa-apama-jyakā}). 2
\\[.65\baselineskip]

\noindent\reversemarginpar\marginnote{\hypertarget{SEpass3}{[3]}}%
The \gls{Cosine_maximum_declination} (\textit{parama-krānti-koṭijyā}) [\ie Cosine of the ecliptic obliquity], \glslink{multiplication}{having been multiplied} (\textit{hatā}) by the \gls{Sine_curve_true_declination} (\textit{sphuṭa-krānti-aṅka-jīvā}) [and] \glslink{division}{having been divided} (\textit{āptā}) by the \gls{Cosine_other_declination} (\textit{anya-krānti-koṭijyā}) [\ie Cosine of the second declination], should be the \gls{Sine_true_declination} (\textit{spaṣṭa-apama-jyakā}). 3 
\medskip

\noindent\reversemarginpar\marginnote{\hypertarget{SEpass4}{[4]}}%
Or, \glslink{extract}{having extracted} (\textit{utthāya}) the \glslink{product_multiplication}{product of the multiplication} (\textit{guṇita-phala}) with the \gls{Sine_curve_true_declination} (\textit{sphuṭa-krānti-aṅka-jyā}) from the \glslink{table_Cosine_maximum_declination}{tables of the <Co>sine of the maximum declination} (\textit{parama-krānti-<koṭi>jyā-koṣṭhika}s) [\ie from the tables of the Cosine of the ecliptic obliquity], [one] \glslink{division}{should divide} (\textit{bhajet}) [that product] by the \gls{Cosine_second_declination} (\textit{dvitīya-krānti-koṭijyā}). The [result] \gls{obtained} (\textit{labdha}) [\ie the quotient of the division] should be the \gls{Sine_true_declination} (\textit{spaṣṭa-krānti-jyā}).  
\medskip

\noindent\reversemarginpar\marginnote{\hypertarget{SEpass5}{[5]}}%
And now, if the \glslink{latitude_celestial_object}{latitude of a celestial object} (\textit{khagasya bāṇa}) should not exist, then its [first] \gls{declination} (\textit{krānti}) alone should be the \gls{true_declination} (\textit{spaṣṭa-krānti}). 


%%%%%%%%%%%%%%%%%%%%%%%%%%%%%%%%%
\clearpage\phantomsection
%%%%%%%%%%%%%%%%%%%%%%%%%%%%%%%%%

{\hindifont\selectfont
\OnehalfSpacing
\large
\te{\noindent\reversemarginpar\marginnote{\hypertarget{Spass6}{[6]}}}%
atha kraantiryadi na syaatpuna.h "saro bhavettadaa baa.najyaa  paramakraantiko.tijyayaa sa.mgu.nyaadha.h kuryaat || ki.mvaa paramakraantiko.tijyaako.s.thakebhyo baa.najyayaa gu.nita\-phalamutthaapayetspa.s.takraantijyaa baa.nadigbhavet || 
\medskip

\te{\noindent\reversemarginpar\marginnote{\hypertarget{Spass7}{[7]}}}%
yadi khagasya kraanti.h paramakraantitulyaa syaattadaa spa.s.takraantya"nka eva sphu.takraanti\-rbhavati || 
\bigskip

{\bfseries || atha prakaaraantare.na ||}
\bigskip

\te{\noindent\reversemarginpar\marginnote{\hypertarget{SpassA}{[α]}}}%
kadambavi.savadhruvadvayamupaiti\te{\footnote{~The word \tsans{vi.sava} (as a part of a compound) appears several times in Kh. I suspect this is an irregular (vernacular?) variant of the word \tsans{vi.suva}/\tsans{vi.suvat} that denotes the `equinox/equinoctial point' in Sanskrit astronomical literature. \Vid\ \textcite[p.\thinspace 4934a]{VacaspatyamVI} for the etymology of the word \tsans{vi.suva} (\textit{upapada tatpuṛusa})  or \tsans{vi.suvat} (\textit{matvarthīya taddhitavṛtti} or secondary nominal derivative from \tsans{vi.su}\,$_\text{\acrshort{indeclinable}}$ `in both directions'). In a lager \textit{tatpuruṣa} compound, \tsans{vi.suva}/\tsans{vi.suvat} refers to the equatorial reference frame, \eg in the genitive-compounds \tsans{vi.suva-v.rtta} `circle of the equinox' (\ie the celestial equator) and \tsans{vi.suvat-dhruva} `pole of the equinox' (\ie the celestial pole). The word \tsans{vi.sava} is not an attested form in any Sanskrit lexicon; however, it is consistently and frequently used throughout Kh. Therefore, I maintain \tsans{vi.sava}\,$_\text{\acrshort{irregular}}$ (as attested in Kh) in the \Nagari\ text but transliterate it using \tsans{vi.suva}\,$_\text{\acrshort{regular}}$ in my English translations. Both \tsans{vi.suva} and \tsans{vi.sava} have the same metrical signature ({\msf{\char"23D1}} {\msf{\char"23D1}} {\msf{\char"23D1}}).\label{vishava_deviant}}%
%
\footnote{~\tsans{\selip dvayamupaiti}~{$\Big ]$}~\tsans{\selip dvayamupeti}\enskip Kh. In the (emended) conjoined word \tsans{\selip dvayam-upaiti}, the terminal verb  
\tsans{upaiti}\,$_\text{\acrshort{present}-\acrshort{indicative}-\acrshort{singular}·\acrshort{third}}$  is derived from \tsans{upe}\,$_\text{\acrshort{compound}-\acrshort{verb}}$. A regular sandhi of the words \tsans{upa}\,$_\text{\acrshort{preverb}}$ (indeclinable \textit{upasarga}) + \tsans{eti}\,$_\text{\acrshort{present}-\acrshort{indicative}-\acrshort{singular}·\acrshort{third}}$ produces \tsans{u\underline{pai}ti} and not \tsans{upeti} (\Panini's \Astadhyayi: 6.1.89). The conjugated form \tsans{eti} derives from either √\tsans{i}\,$_{\text{\acrshort{class}\textsubscript{2}}}$ or its intensified version √\tsans{e}\,$_{\text{\acrshort{class}\textsubscript{2}}}$ (\tsans{aa}\,$_\text{\acrshort{preverb}}$ + √\tsans{i}\,$_{\text{\acrshort{class}\textsubscript{2}}}$). Any interpretation of \tsans{upeti} (in Kh), \eg  \tsans{upeti}\,$_\text{\acrshort{noun}}$, is morphosyntactically inadmissible here. The choice of using the \textit{guṇa} vowel (\textit{e}-diacritic in \tsans{pe}) instead of the \textit{vṛddhi} vowel (\textit{ai}-diacritic in \tsans{pai}) is either a grammatical error or a scribal mistake.}} %
%
v.rtta.m ca yat \\ tadaayanamudiirita.m\te{\footnote{~\tsans{\selip yanamudiirita.m}~{$\Big ]$}~\tsans{\selip yanamudiiriita.m}\enskip Kh. In the (emended) conjoined word \tsans{\selip [.a.a]yanam-udiiritam}, the terminal compound verb \tsans{ud}-√\tsans{iir}\,$_\text{\acrshort{class}\textsubscript{2}}$ takes the affix -\tsans{ita} to form \tsans{\selip udii\underline{ri}ta.m}\,$_\text{\acrshort{causative}-\acrshort{past}-\acrshort{passive}-\acrshort{participle}}$ (used as an adjective). The word \tsans{udiiriitam} (in Kh) is grammatically ill-formed; I suspect the \textit{ī}-diacritic in \tsans{rii} is a scribal hypercorrection.}} %
%
dhruvacatu.skayaata.m tathaa || \\ nabhogavi.savadvayopari\te{\footnote{~\tsans{\selip vi.sava\selip}\,$_\text{\acrshort{irregular}}$  is identical to \tsans{\selip  vi.suva\selip}\,$_\text{\acrshort{regular}}$ or \tsans{\selip vi.suvat\selip}\,$_\text{\acrshort{regular}}$, \vid\ footnote~\ref{vishava_deviant}.\label{vishava_deviant_2}}} %
%
patatsuv.rtta.m\te{\footnote{~\tsans{patatsuv.rtta.m}~{$\Big ]$}~\tsans{tatsuv.rtta.m}\enskip Kh. The third quarter \tsans{nabhoga\selip ...\selip ca yad} of the verse in passage~[α] is metrically short by one syllable (hypometric): the verse otherwise follows a regular \textit{atyaṣṭi samavṛtta} metre called \textit{pṛthvī} with seventeen-syllables per quarter. The context of the verse, and its repetition in \Nityananda's \Sarvasiddhantaraja\ (\spastakrantyadhikara:\,verse~4, \textcite{MisraTD}), suggest \tsans{tatsuv.rtta.m} should be \tsans{patatsuv.rtta.m}.}} %
%
ca yad \\
bhacakrasad.r"saahvaya.m taditi kalpayedgolavit || 1 || \te{\normalmarginpar\marginnote{\footnotesize \textit{pṛthvī}}} \label{verse_1_label_sans_example}
}

%%%%%%%%%%%%%%%%%%%%%%%%%%%%%%%%%
\clearpage\phantomsection
%%%%%%%%%%%%%%%%%%%%%%%%%%%%%%%%%

\noindent\reversemarginpar\marginnote{\hypertarget{SEpass6}{[6]}}%
Now, if the [first] \gls{declination} (\textit{krānti}) should not exist but the \gls{latitude} (\textit{śara}) should, then the \gls{Sine_the_latitude} (\textit{bāṇa-jyā}) that
\glslink{multiplication}{must be multiplied} (\textit{saṃ-guṇyā}) by the \gls{Cosine_maximum_declination} (\textit{parama-krānti-koṭijyā}) [\ie by the Cosine of the ecliptic obliquity], \gls{should_be_lowered} (\textit{adhaḥ kuryāt}). Or, [one] \glslink{extract}{may [again] extract} (\textit{utthāpayet}) the \glslink{product_multiplication}{product of the multiplication} (\textit{guṇita-phala}) with the \gls{Sine_the_latitude} (\textit{bāṇa-jyā}) from the \glslink{table_Cosine_maximum_declination}{tables of the Cosine of the maximum declination} (\textit{parama-krānti-koṭijyā-koṣṭhaka}s) [\ie from the tables of Cosine of the ecliptic obliquity]. [The result] should be the \gls{Sine_true_declination} (\textit{spaṣṭa-krānti-jyā}) in the \gls{direction_latitude} (\textit{bāṇa-diś}). \label{passge_6_sara_glossary_format_example}
\medskip

\noindent\reversemarginpar\marginnote{\hypertarget{SEpass7}{[7]}}%
If the \gls{declination_celestial_object} (\textit{khagasya krānti}) should be equal to the \gls{greatest_declination} (\textit{parama-krānti}) [\ie the obliquity of the ecliptic], then the very \gls{curve_true_declination} (\textit{spa\-ṣṭa-krānti-aṅka}) becomes the \gls{true_declination} (\textit{sphuṭa-krānti}).
\bigskip

\textbf{Now, in another way.}
\bigskip

\noindent\reversemarginpar\marginnote{\hypertarget{SEpassA}{[α]}}%
And what \gls{circle} (\textit{vṛtta}) reaches both the \gls{ecliptic_pole} (\textit{kadamba}) and the \gls{celestial_pole} (\textit{viṣuvat}\,$_\text{\acrshort{regular}}$-\textit{dhruva}), that has been stated to be the \glslink{circle_solstice}{solstitial [colure]} (\textit{āyana}[-\textit{vṛtta}]), and also, the \glslink{circle_four_poles}{[circle] passing through the four poles} (\textit{dhruva-catuṣka-yāta}[-\textit{vṛtta}]). And passing over a \gls{celestial_object} (\textit{nabhoga}) and the \gls{pair_equinoctial_points} (\textit{visuvat}\,$_\text{\acrshort{regular}}$-\textit{dvaya}), what [circle] is \gls{well_rounded} (\textit{su-vṛtta}), that the \gls{knower_spheres} (\textit{gola-vid}) should consider as
the \glslink{circle_congruent_ecliptic}{[circle] congruent to the ecliptic} (\textit{bhacakra-sadṛśa}[-\textit{vṛtta}]) by name. 1 
\label{verse_1_label_eng_example}


%%%%%%%%%%%%%%%%%%%%%%%%%%%%%%%%%
\clearpage\phantomsection
%%%%%%%%%%%%%%%%%%%%%%%%%%%%%%%%%

{\hindifont\selectfont
\OnehalfSpacing
\large
\te{\noindent\reversemarginpar\marginnote{\hypertarget{SpassB}{[β]}}}%
vi.savav.rttabhav.rttasad.r"sayorvivara\raisebox{.5ex}{$\lceil$}ga.m\te{\footnote{~\tsans{vi.savav.rtta\selip}\,$_\text{\acrshort{irregular}}$ is identical to \tsans{vi.suvav.rtta\selip}\,$_\text{\acrshort{regular}}$, \vid\ footnote~\ref{vishava_deviant}.\label{vishava_deviant_3}}} %
%
dhanuraayanav.rttajam || \te{\normalmarginpar\marginnote{$\lceil$~\footnotesize f.\thinspace 20v:\thinspace1~Kh}}\\
bhavati yatkathita.h sa parasphu.taapama iti dyucarasya ca samprati || 2 ||\label{folio_break_sanskrit_example}
\te{\normalmarginpar\marginnote{\footnotesize \textit{drutavilambita}}}
\medskip

\te{\noindent\reversemarginpar\marginnote{\hypertarget{SpassC}{[γ]}}}%
bhavanacakrabhacakrasad.rk.sayorvivaraga.m dhanuraayanav.rttajam ||\\
bhavati yatsa pare.surihodito vi.savapaatayuge\te{\footnote{~\tsans{vi.sava\selip}\,$_\text{\acrshort{irregular}}$ is identical to \tsans{vi.suva\selip}\,$_\text{\acrshort{regular}}$, \vid\ footnote~\ref{vishava_deviant}\label{vishava_deviant_4}.}} %
%
sati kalpite || 3 || \te{\normalmarginpar\marginnote{\footnotesize \textit{drutavilambita}}}
\medskip

\te{\noindent\reversemarginpar\marginnote{\hypertarget{SpassD}{[δ]}}}%
vi.savannabhogamadhye\te{\footnote{~The word \tsans{vi.savat} is an attested secondary nominal derivative (from \tsans{vi.sa}\,$_\text{\acrshort{noun}}$ `poison') meaning `poisonous'. However, in the \textit{tatpuruṣa} compound \tsans{vi.savat-nabhoga-madhye}, I believe \tsans{vi.savat\selip}\,$_\text{\acrshort{irregular}}$, like \tsans{vi.sava}\,$_\text{\acrshort{irregular}}$, is identical to \tsans{\selip  vi.suvat\selip}\,$_\text{\acrshort{regular}}$, \vid\ footnote~\ref{vishava_deviant}.\label{vishava_deviant_5}}} %
%
yatkoda.n.da.m bhav.rttasad.r"sasya ||\\
{\dev{j~ne}}ya.h sad.rgbhujo .asau bhaayanavivare sad.rkko.ti.h || 4 || 
\te{\normalmarginpar\marginnote{\footnotesize \textit{āryā}}}
\\[.65\baselineskip]

\te{\noindent\reversemarginpar\marginnote{\hypertarget{Spass8}{[8]}}}%
khagasya ko.tisi{\dev{~nji}}nii svabaa.nako.tijiivayaa ||\\
hataa .adhariik.rtaa\te{\footnote{~\tsans{hataa .adhariik.rtaa}~{$\Big ]$}~\tsans{hataadhariinvataa}\enskip Kh. The conjoined words \tsans{hataadhariinvataa} in Kh can be segmented as 
\tsans{hataa}\,$_\text{\acrshort{past}-\acrshort{passive}-\acrshort{participle}}$ (from √\thinspace\tsans{han}\,$_{\text{\acrshort{class}\textsubscript{2}}}$) +
\tsans{.adhariinvataa}; however, the compound \tsans{.adhariinvataa} is etymologically defective by the rules of Pāṇinian grammar. The adverbial \textit{CvI}-suffixation to \tsans{adhara}\,$_\text{\acrshort{noun}-\acrshort{stem}}$ (making it \tsans{adharii}\,$_\text{\acrshort{preverb}}$) can only occur with terminal verbs √\thinspace\tsans{k.r}\,$_{\text{\acrshort{class}}_8}$, √\thinspace\tsans{bhuu}\,$_{\text{\acrshort{class}}_1}$, and √\thinspace\tsans{as}\,$_{\text{\acrshort{class}\textsubscript{2}}}$ when forming factitive compound verbs like √\thinspace\tsans{adharii-k.r} (\Panini's \Astadhyayi: 5.4.50), \vid\ \textcite[\textbf{1094}, p.\thinspace 357]{Whitney}. \protect\\[0.5\baselineskip]
A regular sandhi of the words \tsans{hataa}\,$_\text{\acrshort{past}-\acrshort{passive}-\acrshort{participle}}$ 
+ \tsans{.adhariik.rtaa}\,$_\text{\acrshort{past}-\acrshort{passive}-\acrshort{participle}}$ (from √\thinspace\tsans{adharii-k.r}\,$_{\text{\acrshort{class}}_8}$) generates \tsans{hataadharii\underline{k.r}taa} that is morphologically correct and contextually apposite. Also, \tsans{.adhariik.r}\,$_\text{\acrshort{class}\textsubscript{8}}$ is variously attested in this chapter, as well as in \Nityananda's  \Sarvasiddhantaraja\ (\textcite{MisraTD}), in the same mathematical context. \Vid\ glossary entry: \glslink{lowering}{lowering}.}} %
%
bhavetsad.rk.sako.tisi{\dev{~nji}}nii || 5 ||
\te{\normalmarginpar\marginnote{\footnotesize \textit{pramāṇikā}}}
\medskip

\te{\noindent\reversemarginpar\marginnote{\hypertarget{Spass9}{[9]}}}%
taddhanurnavatita"scyuta.m yadaa jaayate sad.r"sabaahusa.m{\dev{~nja}}kam ||\\
yaa nabhogavi"sikhasya si{\dev{~nji}}nii bhaajitaa .adharasad.rk.sadorjyayaa || 6 ||
\te{\normalmarginpar\marginnote{\footnotesize \textit{rathoddhatā}}}\\
taddhanu.h para"saraahvayo bhavedvaa---
}


%%%%%%%%%%%%%%%%%%%%%%%%%%%%%%%%%
\clearpage\phantomsection
%%%%%%%%%%%%%%%%%%%%%%%%%%%%%%%%%

\noindent\reversemarginpar\marginnote{\hypertarget{SEpassB}{[β]}}%
What \gls{arc} (\textit{dhanus}) produced on the \gls{circle_solstice} (\textit{āyana-vṛtta}) becomes situated in the  \gls{difference} (\textit{vivara}) between the \gls{celestial_equator} (\textit{viṣuva}\,$_\text{\acrshort{regular}}$-\textit{vṛtta}) and the \glslink{circle_congruent_ecliptic}{[circle] congruent to the ecliptic} (\textit{bhavṛtta-sadṛśa}[-\textit{vṛtta}]), that is the stated [arc of] \gls{maximum_true_declination} (\textit{para-sphuṭa-apama}) of the \gls{celestial_object} (\textit{dyucara}) just at that present moment. 2 
\medskip

\noindent\reversemarginpar\marginnote{\hypertarget{SEpassC}{[γ]}}%
What \gls{arc} (\textit{dhanus}) produced on the \gls{circle_solstice} (\textit{āyana-vṛtta}) becomes situated in the \gls{difference} (\textit{vivara}) between the \gls{ecliptic} (\textit{bhavana-cakra}) and the \glslink{circle_congruent_ecliptic}{[circle] congruent to the ecliptic} (\textit{bhacakra-sadṛkṣa}[-\textit{vṛtta}]), in this case, that is the declared [arc of] \gls{maximum_latitude} (\textit{para-iṣu}) when the \glslink{conjunction_equinox_node}{conjunction of the equinoctial point and the node of the orbit [of the celestial object]} (\textit{viṣuva}\,$_\text{\acrshort{regular}}$-\textit{pāta-yuga}) has been supposed. 3 
\medskip

\noindent\reversemarginpar\marginnote{\hypertarget{SEpassD}{[δ]}}%
What \gls{arc} (\textit{kodaṇḍa}) of the  \glslink{circle_congruent_ecliptic}{[circle] congruent to the ecliptic} (\textit{bhavṛtta-sadṛśa}[-\textit{vṛtta}]) is between the \gls{equinoctial_point} (\textit{viṣuvat}\,$_\text{\acrshort{regular}}$) and the \gls{celestial_object} (\textit{nabhoga}), that [arc] should be known as the \gls{congruent_bhuja} (\textit{sadṛś-bhujā}); [and what is] between the \gls{celestial_object} (\textit{bha}) and the \gls{circle_solstice} (\textit{āyana}[-\textit{vṛtta}]), [that should be known as] the \gls{congruent_koti} (\textit{sadṛś-koṭi}) [\ie complement of \textit{sadṛś-bhujā}].~4
\\[.65\baselineskip]

\noindent\reversemarginpar\marginnote{\hypertarget{SEpass8}{[8]}}%
The \gls{Sine_koti_celestial_object} (\textit{khagasya koṭi-siñjinī}), \glslink{multiplication}{having been multiplied} (\textit{hatā}) by the \gls{Cosine_latitude} (\textit{sva-bāṇa-koṭijīvā}) [and] \gls{having_been_lowered} (\textit{adharī-kṛtā}), should be the \gls{Sine_congruent_koti} (\textit{sadrkṣa-koṭi-siñjinī}) [\ie Sine of the complement of the \textit{sadṛś-bhujā}]. 5 
\medskip

\noindent\reversemarginpar\marginnote{\hypertarget{SEpass9}{[9]}}%
When [the measure of] its \gls{arc} (\textit{dhanus}), \gls{reducing_from_ninety} [degrees] (\textit{navatitaś-cyuta}), is determined, [it] has the name \glslink{congruent_bhuja}{congruent arc} (\textit{sadṛśa-bāhu}). Or, what is the \gls{Sine_latitude_celestial_object} (\textit{nabhoga-viśikhasya siñjinī}), \glslink{division}{having been divided} (\textit{bhājitā}) by the \gls{lowered_Sine_congruent_base} (\textit{adhara-sadrkṣa-dorjyā}), 6\dots\\ \dots its \gls{arc} (\textit{dhanus}) should be [called] the \gls{maximum_latitude} (\textit{para-śara}) by name.---

%%%%%%%%%%%%%%%%%%%%%%%%%%%%%%%%%
\clearpage\phantomsection
%%%%%%%%%%%%%%%%%%%%%%%%%%%%%%%%%

{\hindifont\selectfont
\OnehalfSpacing
\large
\te{\noindent\reversemarginpar\marginnote{\hypertarget{Spass10}{[10]}}}%
---pare.suparamaapamaakhyayo.h ||\\
sa.myutirviyutirasti ca kramaadgolabaa.nasamabhinnadiktayaa || 7 ||
\te{\normalmarginpar\marginnote{\footnotesize \textit{rathoddhatā}}}\\
sa grahasya paramasphu.taapamo jaayate yutiviyogadiksthita.h ||\\
evamabhranavato.adhiko yadaa khaa.s.tabhuu 180 parimitervi"sodhita.h || 8 || 
\te{\normalmarginpar\marginnote{\footnotesize \textit{rathoddhatā}}}
\medskip

\te{\noindent\reversemarginpar\marginnote{\hypertarget{Spass11}{[11]}}}%
parasphu.takraantibhavajyakaa hataa  sad.rk.sabaahujyakayaa .adhariik.rtaa ||  \\
tadiiya caapa.m bhavati sphu.taapamo digasya sa.myogaviyogadiksamaa || 9 ||
\te{\normalmarginpar\marginnote{\footnotesize \textit{vaṃśasthavila}}}% 
}

%%%%%%%%%%%%%%%%%%%%%%%%%%%%%%%%%
\clearpage\phantomsection
%%%%%%%%%%%%%%%%%%%%%%%%%%%%%%%%%

\noindent\reversemarginpar\marginnote{\hypertarget{SEpass10}{[10]}}%
---There is the \gls{sum} (\textit{saṃyuti}) or the \gls{difference} (\textit{viyuti}) of the two [quantities] known as the \gls{maximum_latitude} (\textit{para-iṣu}) and the \gls{greatest_declination} (\textit{parama-apama}) [\ie the obliquity of the ecliptic] with the  \gls{latitude} (\textit{bāṇa}) and the \gls{celestial_hemisphere} (\textit{gola}) [\ie the declination of the celestial object] in the \gls{same_different_directions} (\textit{sama-bhinna-diś}) respectively. 7\\ That [result], being situated in the \gls{conjunction_disjunction_direction} (\textit{yuti-viyoga-diś}), becomes the \gls{maximum_true_declination_celestial_object} (\textit{grahasya parama-sphuṭa-apama}). Thus, when [its measure is] \gls{greater} (\textit{adhika}) than \gls{ninety} [degrees] (\textit{abhra-nava}), [it is] \glslink{subtraction}{made to be subtracted} (\textit{viśodhita}) from a measure of \gls{one_hundred_eighty} [degrees] (\textit{kha-aṣṭa-bhū}). 8 \medskip

\noindent\reversemarginpar\marginnote{\hypertarget{SEpass11}{[11]}}%
The \gls{Sine_maximum_true_declination} (\textit{para-sphuṭa-krānti-bhava-jyakā}), \glslink{multiplication}{having been multiplied} (\textit{hatā}) by the \gls{Sine_congruent_base} (\textit{sadṛkṣa-bāhu-jyakā}) [and] \gls{having_been_lowered} (\textit{adharī-kṛtā}), its \gls{arc} (\textit{cāpa}) becomes the \gls{true_declination} (\textit{sphuṭa-apama}). Its \gls{direction} (\textit{diś}) is the \gls{same} (\textit{sama}) as the \gls{conjunction_disjunction_direction} (\textit{saṃyoga-viyoga-diś}). 9

%%%%%%%%%%%%%%%%%%%%%%%%%%%%%%%%%%%%%%%%%%%%%%%%%%%%%%%%%%%%%%%%%
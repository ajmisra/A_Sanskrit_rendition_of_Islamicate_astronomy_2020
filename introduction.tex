\renewcommand{\thefootnote}{\arabic{footnote}} % Hindu-Arabic footnote numbering

Over the course of the history of Sanskrit mathematical astronomy (\ganita\ \jyotisa), foreign ideas have evoked a full range of emotions that extend from affinity to apathy, going all the way to antipathy. These reactions are a reflection of the intellectual diversity of Indian astral sciences (\jyotihsastra). Historical actors may have chosen to accept, reject, or ignore foreign ideas based on their scientific convictions; however, those choices could only be expressed under the aegis of the political and sociocultural institutions of the times. 

With the turn of the seventeenth century \ce, Sanskrit astronomers/astrologers (\jyotisa s) and their Persianate counterparts (\munajjim s) were working under the common patronage of the imperial court of Mughal India.\footnote{\,The \GurkaniAlam\ or the Mughal Empire was an early-modern Muslim empire in South Asia led by monarchs of the Timurid dynasty. From 1526 to 1857 \ce, the successors of \Baburfull, the first Mughal Emperor, extended their dominion over large swathes of the Indian subcontinent, and in doing so, helped create a highly complex cosmopolitan society extending beyond its imperial borders. I refer to this cultural sphere of influence of the Mughal rule as Mughal India.} At the court of Emperor \Shahjahan\ (r.\thinspace 1628--58), we find the Gauḍa Brahmin \Pandita\ \Nityanandafull\ (\fl 1630/50) working alongside \MullaFaridfull\ (d.\thinspace \circa 1629/32; henceforth identified as \MullaFarid) to translate into Sanskrit the latter's Persian \zij\ (a handbook of astronomical tables) the  \ZijiShahJahani\ (\circa 1629/30). \Nityananda's Sanskrit translation, the \Siddhantasindhu\ (\circa early 1630), was his first attempt at explaining Islamicate\footnote{\,I use the word Islamicate (instead of Islamic) to indicate the cultural outputs (\eg artistic, literary, or scientific works) of Muslim societies educated in the Arabic and Persian language traditions but not directly connected to the Islamic faith or any particular geographic region \parencite[\vid\ recent discussions on Islamicate Secularities in][]{Dressleretal}.\label{islamicate_defintion}} computations and astronomical tables to his fellow Sanskrit \jyotisa s. 

By the end of the decade, he included several of these Islamicate ideas in his canonical treatise the \Sarvasiddhantaraja\ `The King of all \Siddhanta s' (1639). The \Sarvasiddhantaraja\ is composed in the style of a traditional Sanskrit \textit{siddhānta} (a canonical treatise in astronomy) and has a tripartite structure: the \ganitadhyaya\ `chapter on computations', the \goladhyaya\ `chapter on spheres', and the \yantradhyaya\ `chapter on instruments'.\footnote{\,\textcite[Sections~1.1 and 1.2 on pp.\thinspace 1--20]{Misrathesis} offers a fuller discussion on Sanskrit astronomy in early-modern India, in particular, the contribution of \Nityananda\ and his \Sarvasiddhantaraja. Also, contemporary studies like \textcite{PingreeSSR}, \textcite{MonetelleSine1}, and \textcite{MonetelleSine2} discuss Islamicate influences in the mathematical computations described in the \Sarvasiddhantaraja.} In contrast, the \Siddhantasindhu\ mimics the structure and content of the Persian \ZijiShahJahani\ quite intimately. 

In this study, I compare the general structure of the \ZijiShahJahani\ and the \Siddhantasindhu\ in parallel, and subsequently focus on a chapter from each of these works that discusses the same topic, \viz the declination of a celestial object. My aim is to highlight the semantic and communicative aspects in \Nityananda's Sanskrit translation of \MullaFarid's Persian text. I defer all remarks on the mathematics in \Nityananda's text to \textcite[forthcoming]{MisraTD}. Instead, I first begin by discussing the practice of translating Sanskrit, Arabic, or Persian astronomical texts during the late-medieval and early-modern periods of Indian history. This overview, built from separate studies on the history, philosophy, and language of astral sciences in India,\footnote{\,For example, \vid\ \textcites{PingreeIslamAstSkt, PingreeIndianReception, Ansaritransmission, Ansarihindu, Ohasi} for historical accounts of Sanskrit and Persian astronomy in India; \textcites{Choudhuri, Minkowskilearnedbrahmin} for surveys of Sanskritic (Hindu and Jain) scholars under Muslim patronage; and \textcite{Truschke, Minkowski, MinkowskiSuryadasa, Nair} for linguistic and philosophical reforms affecting Sanskrit \jyotihsastra\ in early-modern India.} gives us the context to situate \Nityananda's works in the world of seventeenth-century Mughal India. His writings can then be seen as an ongoing dialogue between different scientific traditions in a changing society, instead of simply being judged as a `failure' and an `elaborate apology for using Muslim astronomy' \parencite[270]{PingreeSSR}. 


%%%%%%%%%%%%%%%%%%%%%%%%%%%%%%%%%%%%%%%%%%%%%%%%%%%%%%%%%%%%%%%%%%%%%%%%%%%%%%%%%%%%%%%%%
\subsection[Translating the astral sciences in Persianate India]{Translating the astral sciences in Persianate India\footnote{\,The word Persianate refers to a sociocultural association with the Persian language (Fārsī) extending beyond the ethnic identity and geographical boundaries of Persia (much like the word Islamicate distinguishes itself from Islamic, \vid\ footnote~\ref{islamicate_defintion}). I use the expression `Persianate India' to refer the geographical regions of late-medieval and early-modern India where Persian culture (expressed in its art, language, literature, and science) directly influenced society \parencite[\vid][]{Eaton2019}.}}\label{translating_the_astral_sciences_in_persianate_india}
%%%%%%%%%%%%%%%%%%%%%%%%%%%%%%%%%%%%%%%%%%%%%%%%%%%%%%%%%%%%%%%%%%%%%%%%%%%%%%%%%%%%%%%%%

\subsubsection{Before the Mughal court} \label{pre_Mughal_court}
Sanskrit texts on astronomical instruments (\yantra) written in the late fourteenth century offer some of the earliest extant evidence of a relationship between Islamicate and Sanskrit mathematical astronomy.\footnote{\,Sanskrit \tajika\ texts are Indian adaptations of Islamicate astrology that were composed from the thirteenth century \ce. \textcite{PingreeAstralOmens} provides a historical summary of the \tajika\ literature in Sanskrit, while \textcite{Gansten} studies the transmission of Perso-Arabic \tazig-astrology in the \Karmaprakasa\ (\circa 1274) of \Samarasimha, the earliest preserved Sanskrit \tajika\ work.\label{tajika_footnote}} \textcite{SarmaAstrolabeinSanskrit} provides an excellent overview of Sanskrit texts on astrolabes, many of which include lengthy discussion on Islamicate mathematical astronomy. \MahendraSuri's \Yantraraja\ (1370), along with his student \MalayenduSuri's commentary on it (in 1382), is the earliest and most recognised of such works \parencite{PlofkerAstrolabe}. 

\MahendraSuri\ was a Jain monk-astronomer at the court of \FiruzShah\ (r.\thinspace 1351--88), a Turko-Indian ruler of the pre-Mughal Sultanate of Delhi, and is thought to have worked in close association with `unnamed Muslim astronomers at Fīrūz's court' \parencite[148]{SarmaAstrolabeinSanskrit}. \citeauthor{SarmaAstrolabeinSanskrit} qualifies him as a `mediator between the Islamic and Sanskritic tradition of learning' (p.\thinspace 149). This is perhaps justly so, as three hundred years later,  the language and structure of \MahendraSuri's \Yantraraja\ continues to echo in the works of several seventeenth-century authors. The \Yantrasiromani\ (\circa 1612/15) of \Visrama\ of \Jambusara, the \Vasanavarttika\ (1621) of \NrsimhaDaivajna\ of \Kashi, and the \Sarvasiddhantaraja\ (1639) of \Nityananda\ are three such examples.\footnote{\,\MahendraSuri's \Yantraraja, along with \Visrama's \Yantrasiromani\ is edited by \textcite{Raikva}. \NrsimhaDaivajna's \Vasanavarttika, a commentary on \BhaskaraII's \Siddhantasiromani\ (1150), is edited by \textcite{Chaturvedi}. There are no known editions or translations of \Nityananda's \Sarvasiddhantaraja\ in its entirety. \textcite[149]{SarmaAstrolabeinSanskrit} describes how the structure of the \yantradhyaya\ `chapter on instruments' from the \Sarvasiddhantaraja\ mimics that of \MahendraSuri's \Yantraraja\ on the basis of MS~264 from the Asiatic Society of Bombay.} 

%--------------------%

\subsubsection{At the Mughal court} \label{Mughal_court}
During the Mughal rule of India, the practice of translation became an administrative activity under the patronage of the Mughal emperors. Various kinds of literary, historical, religious and scientific texts in Sanskrit were chosen to be translated into Persian. As \textcites{AlamSubramanyam, Truschke} have astutely observed, these translations served, more than anything else, to help the Mughal crown conceive and consolidate its self-identity as a ruling establishment harmonious with locally existing notions of kingship.\footnote{\,\textcite{Haider} offers an excellent study on the role of language and translations in the context of intercultural communication and Mughal state-building. \textcite{Israel} builds on this to examine the complex ways in which translation processes and political discourses are mobilised to shape cultural and national identities.} %
% \footnote{\,In talking about the scientific activities at the Mughal court of Emperor \Humayun\ (r.\thinspace 1530--56)  \textcite{Anooshahr} notes that `the court’s network of patronage reflected the cosmopolitan (and cosmocratic) ambitions of the emperor, extending to intellectuals from Shiraz, Herat, Istanbul, Gwalior and Samarqand. The changes continued and intensified as the remaining decades of the century unfolded' (p.\thinspace 315).} %
For a small group of professionals, however, these translation projects offered more immediate opportunities for employment at the Mughal court and with it, a chance for social recognition.

We learn from the sixteenth-century Mughal historian \Badaunifull\ that Emperor \Akbar\ (r.\thinspace 1556--1605) established a scriptorium  (\maktabkhana) where secretaries, scholars, and scribes worked collaboratively to produce Persian editions of Sanskrit texts.\footnote{\,\Vid\ \Badauni's \MuntakhabalTawarikh, \textcite[Vol.\thinspace II, p.\thinspace 344]{Lees} for the Persian text; its English translation can be found in \textcite[Vol.\thinspace II, p.\thinspace 356]{Lowe}. Also \vid\ \textcite[Chapter~6, pp.\thinspace 203--222]{Rizvi} for a descriptive list of the Sanskrit works translated at \Akbar's \textit{maktabkhāna}, including \AbualFaydi's Persian translation \TarjumayiLilawati\ of \BhaskaraII's \Lilavati\ (\circa mid-twelfth century \ce) from 1587.} According to \Badauni, Sanskrit interpreters (\muabbiran) and Persian translators (\mutarjiman) worked separately at different stages of the translation process.\footnote{\,\MuntakhabalTawarikh: Persian text in \textcite[Vol~II, pp.\thinspace 320--321]{Lees} and its English translation in \textcite[Vol.\thinspace II, pp.\thinspace 329--330]{Lowe}.} Starting with a vernacular paraphrasing of the Sanskrit text by Hindu/Jain scholars (\pandita s or \sastrin s), perhaps in a colloquial dialect of Hindavī, Khaṛībolī, or Braja\-bhāṣā,\footnote{\,As \textcite{AlamPersian1998} observes, `[h]indavī was recognized as a semi-official language by the Sūr Sultāns (1540--55) and their chancellery rescripts bore transcriptions in the Devanāgarī script of the Persian contents. The practice is said to have been introduced by the Lodīs (1451--1526)' (p.\thinspace 319). \Vid\ \textcite{Behl2012} for a study of the Hindavī literary traditions in pre-Mughal India; \textcite{Imre} for the emergence of Khaṛībolī literature in Northern India; and \textcite{Busch} for poetry in Braja\-bhāṣā at the Mughal courts.} a preliminary Persian translation was prepared by Muslim clerks/secretaries (\muharrir s). This was then refined by more accomplished Persian scholars (\ustadh s or \mutamarris es) into its final form over several revisions \parencite[564--566]{Hodivala}. 

It is reasonable to think that this process also occurred in reverse as Persian texts were translated into Sanskrit. In his \AiniAkbari, \Akbar's chronicler \AbulFadlAllami\ mentions at least one instance where a Persian astronomical text was translated into Sanskrit: the \ZijMirzai\ (alias \ZijUlughBeg) was translated into Sanskrit (\JicaUlugbegi) under the superintendence of \AmirFathullahofShiraz\ with the assistance of Kishan Joshī, Gaṅgādhar, and Mahesh Mahānand %
% \footnote{\,\textcite[\P~2.4 on p.\thinspace 367]{Sarmajyotisaraja} identifies Kishan Joshī as \Krsnadaivajna\ (\fl \circa 1600/25 \ce), a favourite of Emperor \Jahangir\ (r.\@ 1605-- 1627 \ce). \Krsnadaivajna\ belonged to a prominent family of Devarāta Brahmins who had emigrated to \Kashi\ \parencite[pp.\thinspace 53a--55b in Volume A2]{PingreeCESS}. In the seventeenth century, their academic rivalry with a family of Bhāradvāja Brahmins, also immigrants in \Kashi, ensured many Sanskrit texts debating Islamicate astronomy were authored by both sides \parencite[\vid][122-127]{Minkowskilearnedbrahmin}. \Munisvarafull\ (b.\thinspace 1603 \ce, Devarātagotra, the nephew of \Krsnadaivajna) and \Kamalakara\ (b.\thinspace 1610 \ce, Bhāradvājagotra, nephew of \NrsimhaDaivajna\ of \Kashi), two of \Nityananda's contemporaries, were noted scholars from these rival families. Along with \Nityananda's \Sarvasiddhantaraja\ (1639 \ce), \Munisvara's \Siddhantasarvabhauma\ (1646 \ce) and \Kamalakara's \Siddhantatattvaviveka\ (1658 \ce) form the corpus of Sanskrit \siddhanta s composed in seventeenth-century Mughal India that discuss Islamicate mathematical astronomy.}
\parencite[110]{BlochmannAiniAkbari}.\footnote{\,H.\thinspace Blochmann translated the first two books of the \AiniAkbari\ into English in 1873 (published by the Asiatic Society of Bengal). \citeauthor{BlochmannAiniAkbari} edited and revised the second edition in 1927, which was then reprinted in 1977. As \textcite[footnote~20 on p.\thinspace 367]{Sarmajyotisaraja} points out, Blochmann's statement on translating the \ZijUlughBeg\ by a consortium of Sanskrit scholars is indeed `hopelessly garbled' in his English translation in all three editions.} % 

In \Shahjahan's reign, beginning in 1628, we find a Persian translation \TarjumayiBijganit\ (\circa 1634/35) of \BhaskaraII's  \Bijaganita, a celebrated twelfth-century Sanskrit treatise on Algebra, written by the Mughal architect \AtaAllahRushdi\ and dedicated to Emperor \Shahjahan\ \parencite[384--386]{Ansaribhaskara}. This is also around the same time when \Nityananda\ translated \MullaFarid's Persian \ZijiShahJahani\ into his Sanskrit \Siddhantasindhu. Although there are no intermediaries (interpreters/translators) that are explicitly named in either of these works, there are historical precedents from the literary traditions, particularly those patronised by the Muslim nobility of early-modern India, to suppose the presence of bilingual interlocutors.\footnote{\,For instance, based on her study of the vernacular literary culture of early-modern North India, \textcite{Orsini} remarks that `it is better to understand the literary culture in fifteenth-century north-India as a multilingual and multilocation literary culture---with a trend towards Persian-Hindavi bilinguality in the domains of politics and literature of the various regional Sultans and in the Sufi religious and literary practices' (pp.\thinspace 238--239).} 

Beyond these literary traditions, we learn from \textcite[327--328]{AlamPersian1998} that
by the middle of the seventeenth century, most administrative positions in the Mughal chancellery were occupied by Persian-speaking Hindu \munshi s, many of whom made significant contributions to Persian literature.\footnote{\,Many Hindu \munshi s at the Mughal courts wrote epistolary prose (\textit{inshā}) and composed poetry in Persian. The story of Chandar Bhān Brahman (d.\thinspace \circa 1666--70), a Brahmin \munshi\ who lived through the reign of four Mughal emperors, is a fascinating tale of how a Hindu secretary came to be regarded as one of the great Persian prose stylists and poets of his era. \textcite{Kinra} offers a particularly compelling account of the literary, social, and political worlds of \Shahjahan's Mughal India through the life and works of Chandar Bhān Brahman.} %
The power and prestige associated with being literate in Persian extended beyond the circles of Hindu imperial administrators---including, of course, Hindu nobility like the Rajput kings---and even reached low-ranking officials in smaller towns and villages. By the time of \Shahjahan's reign, Persian classics like \AkhlaqiNasiri\ of \alTusifull\ or \MasnaviyiManavi\ of \Rumipoet\ became regular reading material even among the less-prominent Hindus associated with the Mughal state \parencite[328]{AlamPersian1998}. 

\citeauthor{AlamPersian1998}'s observations allow us to see how Persian became a tool of socioeconomic mobility for the professional classes in seventeenth-century Mughal India. The Sanskrit \jyotisa s (and perhaps, even the Muslim \munajjim s) served as astrologers for various high-ranking Hindus in the Mughal realm.\footnote{\,For example, \VedangarayaMalajit\ (\fl 1643, also known as \Srimalaji) was a \jyotisa\ at \Shahjahan's court. His admittance to the imperial court was presumably mediated by his immediate patron Rāja Giridhara Dāsa, the Rajput King of Ajmer, to whom, \Srimalaji\ dedicated his \Giridharananda\ `The joy of Giridhara' \parencite[121--122]{Minkowskilearnedbrahmin}. As \citeauthor{Minkowskilearnedbrahmin} remarks, `the presence of a \jyotisa\ at a particular court appears in some cases to have been rather notional. The Banarsī paṇḍits, in particular, received gifts, honors, or patronage simultaneously from several courts, large and small' (p.\thinspace 116).\label{srimalaji_footnote}} These Hindus, as \textcite{Truschke} describes them, `joined the Mughal administration and became absorbed into Persian-speaking communities' (p.\thinspace 8). Essentially, they were now a part of the Mughal Persianate elite. As their consultant astrologer, the ability to be reasonably bilingual (fluent in vernacular Hindi and conversant in Persian) would have been a competitive advantage and social distinction for any Sanskrit \jyotisa.

From the seventeenth century, the linguistic hegemony of Persian that served the political ambitions of the Mughal crown was met with the rising popularity (and patronage) of vernacular literature among the Persianate elite, \parencite[\eg \vid][chapters~3--4]{AllisonHindiPoetry}. The prominence of Hindavī/Brajabhāṣā literature, coupled with a politico-cultural shift towards the vernaculars (in other words, treating vernacular texts as sources of cultural history instead of those written in Sanskrit) led to renewed ways in which Persian writers engaged with the vernacular cultures.\footnote{\,For instance, \MirzaKhanbFakhralDinMuhammad\ wrote his encyclopedic Persian digest \TuhfatalHind\ (\circa 1674/75) `Gift from India' on the `current Indian sciences' (\Ayn \textit{ulūm\=/i mutadāwila\-/yi hindiya}) during the reign of Mughal Emperor \Awrangzeb\ (r.\thinspace 1658--1707). His book includes discussions on various topics of ordinary and academic interests peculiar to the people of who spoke \textit{Braj Bhākhā} (Brajabhāṣā). \Vid\ \textcite{Ziauddin} for an English translation of Mirzā Khān's elaborate exposition of the grammar of \textit{Braj Bhākhā}; and more generally, \vid\ \textcite[342--348]{AlamPersian1998} for a historical summary of the relationship between Persian and Hindavī at the Mughal courts.} 

% Sanskrit and vernacular accounts also reveal how select Hindu astrologers were conferred the title  (Skt.\@ \Jyotisaraja) (Skt.\@ \Vedangaraja), both implying `Royal Astrologer', by successive Mughal emperors beginning with Akbar \parencite[\eg \vid][58--61]{Truschke}.\footnote{\,The Sanskrit compositions of these title-holders, often including encomiums to their generous patrons, also reflect the Islamicate heritage of the Muslim  benefactors: for instance, the \TajikaNilakanthi\ (1587 \ce) of \Akbar's Jotik Rāi \Nilakantha\ (\fl 1569/87 \ce) was based on Islamic \tajika\ astrology \parencite[pp.\thinspace 177b--189a in Volume A3]{PingreeCESS}; or the \SamskrtaParasikaPadaPrakasa\ (1643 \ce) of \Shahjahan's \VedangarayaMalajit\ (\fl 1643\ce) was a Sanskrit text on Persian grammar \parencite[pp.\thinspace 421a--422a in Volume A4]{PingreeCESS}.} %

Sanskrit poets also learned to adapt to this shift towards the vernaculars. Many scholars maintain that the literary eminence of Sanskrit at the Mughal court began to wane in the reign of \Shahjahan\
\parencites[\eg][]{PollockDeathofSanskrit, Truschke}.%
% (\eg \cite[404--412]{PollockDeathofSanskrit}; \cite[pp.\thinspace 13--16]{Truschke}).
\footnote{\,In contrast, Sanskrit poetry (\textit{kāvya}) composed outside the central Mughal court played a critical role in elaborating the vernacular cultures and identities. As \textcite{BronnerShulman} elaborate in their study, Sanskrit was employed to articulate regional distinctiveness instead of occluding it.} The accounts of two Hindi-speaking Brahmin poets at \Shahjahan's court, \KavindracaryaSaravati\ (\fl \circa 1600/75) and \JagannathaPanditaraja\ (\fl \circa\ 1620/60), describe how two eminent Sanskrit scholars ingratiated themselves with the emperor and his retinue by composing panegyrics in Braja\-bhāṣā and singing Hindustānī \textit{dhrupad} songs at the Mughal court (\cite[50--53]{Truschke}).

For a lesser-known Sanskrit \jyotisa\ like \Nityananda, however, one can imagine that the changing tides of patronage and the competition to find patrons, would have presented very different challenges to those faced by courtly bards singing encomiums. \Nityananda's name appears in the annals of Sanskrit \Jyotihsastra\ as the author of \Siddhantasindhu---a Sanskrit translation of a Persian original sponsored by \AsafKhanshort, the prime minister (\vazir\idafaconsonant\ \azam) of \Shahjahan\ and a highly influential Mughal elite. It is, therefore, not inconceivable that \Nityananda\ might have had some basic level of Persian literacy to begin with, or at the very least, developed it through his interactions with \MullaFarid\ (in vernacular Hindi) over the course of his commission. The grammatical affinity between \MullaFarid's Persian passages and their Sanskrit translation in \Nityananda's \Siddhantasindhu\ supports this belief to a certain extent (more on this in \S~\ref{language_content_zij_sindhu_chapter_six}). 

\paragraph{Sanskrit manuals on learning Persian} 
Between the fourteenth and eighteenth centuries, several Sanskrit compendiums were authored to teach Persian to Sanskrit-speaking audiences \parencites[\eg \vid][]{SarmaSanskritmanuals, Truschkepersiansanskritlexica}. Typically, these compendiums comprised of two sections composed in metrical Sanskrit verses: namely, the \textit{kośa prakaraṇa}, a bilingual Persian-Sanskrit lexicon, and the \textit{vyākaraṇa prakaraṇa}, a section on the rules of Persian grammar. The \Parasiprakasa\ (\circa 1575) of \Krsnadasa\ dedicated to \Akbar\ and the \SamskrtaParasikaPadaPrakasa\ (1643) of \VedangarayaMalajit\ sponsored by \Shahjahan\ are two prominent exemplars \parencite[\vid][]{SarmaSanskritPersianLexica}. The former contains a general collection of Persian words, whereas, the latter includes a specialised lexicon on technical terms in Islamicate astrology/astronomy. It is doubtful if either of these manuals were ever sufficient to learn Persian. However, their value in promoting Persian as a language of sociopolitical influence in Mughal India is certainly conceivable.\footnote{A statement in support of this idea is found in the words of the Sanskrit scholar \Pandita\ \Suryadasa\ Daivajña (b.\thinspace 1508). \Suryadasa\ wrote a versified glossary of Perso-Arabic astrological terms as a section of the fifth chapter in his \Siddhantasamhitasarasamuccaya\ (1583). He begins the section by claiming (in v.\thinspace 56) that the knowledge of the `technical terms stated in the science of the foreigners'  (\textit{yavana-śāstra-uktā saṃjñā}) will be `useful in the royal court' (\textit{narapati-sabhā-upayogya}) and will also be `beneficial to astrologers' (\textit{upakāra-artha daivavidām}); \vid\ \textcite[p.\thinspace 329--330]{MinkowskiSuryadas} for the Sanskrit text of this verse, and also an overview of \Suryadasa's contributions in promoting Islamicate astrology in Sanskrit. More generally, \vid\ \textcite{AlamPersian2003} for an excellent study on the cultural and political role of Persian in the polity of Mughal India.}  

% More recently, \textcite{Nair} has studied how Sanskrit philosophical and religious texts were translated into Persian by collectives of Sanskrit and Persian scholars at the Mughal courts. \citeauthor{Nair} interprets philosophical translations as cross-cultural intellectual `dialogues' expressed through the creation of novel vocabulary. This interpretation can also be extended to technical translations, particularly those that help bridge language barriers in communicating foreign science. A hundred years after \Nityananda, his words continued to echo in the conversations between Sanskrit and Islamicate astronomy beyond the Mughal court.
%--------------------%

\subsubsection{Away from the Mughal court} \label{away_Mughal_court}
By the turn of the eighteenth century, the locus of Sanskrit patronage shifted from the Mughal court to the courts of the vassal states under Mughal suzerainty. Among these subimperial sponsors, the royal patronage of Mahārāja Savāī Jayasiṃha of Jayapura (Jaipur) is particularly pertinent to the history of Sanskrit astronomy. Savāī Jayasiṃha II (r.\thinspace 1699--1743) was the Kachvāha Raj\-put King of Āmera (and later Jayapura) who invested in Sanskrit astronomy both academically and economically. He not only paid for the construction of five astronomical observatories in India but also instituted an ambitious project to translate Islamicate scientific works into Sanskrit; in particular, Arabic and Persian version of Greco-Islamicate mathematics and astronomy---and to a lesser extent, even the European astronomical tables brought to him by the Jesuits \parencites{SarmaJaiSinghSanskrit, PingreeAstronomersprogress}.\footnote{\,In her doctoral dissertation, \textcite{Johnson_Roehr} describes the sociopolitical impact of Savāī Jayasiṃha's urban observatories, in particular, the emplacement of ancillary knowledge-systems (like accounting, masonry, \etcp) within the local landscape of his newly built city of Jayapura (Jaipur). Her observations locate these subsidiary activities within Jayasiṃha's programme of assimilating Islamicate and European astronomy, and in that capacity, offer an interesting parallel to the patronage of professional interpreters, scribes, accountants, and clerks in early-modern society of Mughal India \parencite[\eg  \vid][Chapter~7 `The Making of a Munshī']{AlamSubramanyam}.} %

\textcite{SharmaSawaiJaiSingh} provides a descriptive account of the Hindu astronomers, astrologers, observers, and scribes recruited under Savāī Jayasiṃha's programme. Among these names, \Jagannathafull\ (\fl \circa 1720/40), \Nayanasukhopadhyaya\ (\fl 1729), and \Kevlaramafull\ (\fl \circa 1730/80) are three notable Hindu \jyotisa s who translated the science of the \yavana s (foreigners) into Sanskrit.\footnote{\,\Kevalarama\ authored several works at Jayasiṃha's court; one of them is believed to be the \Drkpaksasarani\ (\circa 1733), a Sanskrit adaptation of Philippe de La Hire's \textit{Tabulae Astronomicae} (1702) based on its 1727 Latin edition 
\parencite{PingreeLaHireSanskrit}. There is at least one other Sanskrit text, the \Phirangicandracchedyopayogika\ (\circa 1732/24), also inspired by La Hire's work but collectively authored by the \jyotisa s at Jayasiṃha's court (\cite[248--249]{SAT_Montelle_Plofker}; \cite{PingreeLaHireJaysimha}).\label{kevalarama_works_foreign}} %
Table~\ref{sanskrit_translations_of_islamic_texts} lists some of the more prominent Sanskrit translations of  Arabic and Persian works, particularly, those that were commissioned by Savāī Jayasiṃha in the early eighteenth century \parencite[131--151]{PingreeJaipur}.\footnote{\,These works are translations in an \textit{explicit} sense; there are other works, mostly \siddhanta s composed in the late seventeenth- and early eighteenth-century Mughal India, that implicitly engage and discuss Islamicate astronomy. \Nityananda's \Sarvasiddhantaraja\ (1639), \Munisvara's \Siddhantasarvabhauma\ (1646), and \Kamalakara's \Siddhantatattvaviveka\ (1658) are three canonical examples of Sanskrit texts that discuss Islamicate mathematical astronomy. More on this in \S~\ref{sanskrit_translation_of_persian_zij} \label{explicit_implicit_translation_sanskrit}.}

%%%%%%%%%%%%%%%
\begin{table}[!hbp]
    \centering
    \renewcommand{\arraystretch}{1.25}
    \renewcommand{\baselinestretch}{1.25}\selectfont
    \begin{tabularx}{\textwidth}{cX}
    \hline
    Date of composition & Sanskrit text\\
    \hline
    \textit{ante} 1694 &
    the \Hayatagrantha, an anonymously authored Sanskrit translation of \aliQushji's Persian text \Risaladarilmalhaya\ (1458) `Treatise on Astronomy (\ilmalhaya)', edited by  \textcite{BhattacaryaHayata}\\
    1726 & the \Rekhaganita\ of \Jagannathafull\ (1652--1744), a Sanskrit translation of \alTusifull's Arabic text
    \KitabTahirUsulliUqlidus\ (\circa 1248) `The recension of Euclid's \Elements', edited by \textcite{Trivedi}\\
    \circa 1726--1732 & the \Samratsiddhanta\ of \Jagannathafull, a Sanskrit translation of \alTusifull's  Arabic recension \TahriralMijisti\ (1247) `Commentary on [Ptolemy's] \Almagest'; three versions of this text are attested: the earliest, called \Samratsiddhantakaustubha, is from 1726, while the two later expanded versions are from 1730 and 1732 respectively; the text dated 1732 is edited by \textcite{SharmaSamratSiddhanta}\\
    1729 & the \Ukara\ of \Nayanasukhopadhyaya\ (with the assistance of \MuhammadAbidda), a Sanskrit translation of \alTusifull's Arabic recension \Tahrialukarr\ (1253) `Commentary on [Theodosius'] \Sphaerica', edited by \textcite{BhattacaryaUkara}\\
    1729 & the \Sarahatajakiravirajandi\ of \Nayanasukhopadhyaya\ (with the assistance of \MuhammadAbidda), a Sanskrit translation of Chapter~11 from Book~II of \alTusifull's \alTadhkirafiilmalhaya\ `Memoirs on Astronomy' (1261--1274) with \alBirjandi's \SharhalTadhkirah\ (1507) `Commentary on the \textit{Tadhkira}', edited by \textcite{KusubaPingree}\\
    \circa 1730 & the \Yantrarajasyarasala\ aliases \Visavava, \Yantrarajavicaravimsadhyayi\ of \Nayanasukhopadhyaya\ (suspected), a Sanskrit translation of \alTusifull's Persian text \RisaladarBistBabMarifatUsturlab\ `Treatise in Twenty Chapters on the Knowledge of the Astrolabe' (\circa 1240), edited by \textcite{BhattacaryaYantrarajavicara}\\
    \hline
    \end{tabularx}
    \captionof{table}{Major Sanskrit translations of Arabic and Persian astronomical texts}
    \label{sanskrit_translations_of_islamic_texts}
\end{table}
%%%%%%%%%%%%%%%
% \pagebreak %%%%%%%%%%%%%%%%%%%%% PAGEBREAK FOR FORMATTING ON PDF, CAN BE REMOVED LATER 

It is worth noting that Savāī Jayasiṃha possessed a copy of \Nityananda's \Siddhantasindhu\ (\circa early 1630s).\footnote{\,MS~Museum~23 (444 folia) of the \Siddhantasindhu\ held at the Maharaja Sawai Man Singh II Museum at the City Palace of Jaipur attests that it was copied by Gaṅgārāma of Kaśmīra for Mahārāja Jayasiṃha on Thursday 6 April 1727 \ce\ (stated on f.\thinspace 443). A second note (in Hindi) appears on f.\thinspace 444v indicating it copied (from an earlier copy?) by Gaṇgāyaratna on \circa 24 May 1726 \ce\ \parencite[142--143]{PingreeJaipur}.\label{ss_museum_23_description}} It is very likely he also possessed a copy of \Nityananda's \Sarvasiddhantaraja\ (1639). As \textcite[79]{PingreeAstronomersprogress} notes, the earliest version of \Jagannathafull's \Samratsiddhanta\ (\ie the \Samratsiddhantakaustubha\ from 1726; the third entry in Table~\ref{sanskrit_translations_of_islamic_texts}) includes the astronomical parameters of \UlughBeg\ derived from \Nityananda's \Sarvasiddhantaraja. \citeauthor{PingreeAstronomersprogress} continues on to say: `From \Jagannatha's use of [the astronomical parameters] we come to realize what has long been suspected, that \Nityananda's arguments, originally advanced in 1639, finally found a receptive audience, nearly a century later, at Jayasiṃha's court' (p.\thinspace 79). In fact, \Nityananda's technical vocabulary also provides some of the terminology with which \Kevalarama\ (\vid\ footnote~\ref{kevalarama_works_foreign}) translates European astronomy into Sanskrit around the mid eighteenth-century \ce\ \parencite[283]{PingreeSSR}.

The method of translation at Savāī Jayasiṃha's court becomes evident with one of his astronomers' own statement on the process. \Nayanasukhopadhyaya, in his \Ukara\ and his \Sarahatajakiravirajandi\ (the fourth and fifth entry in Table~\ref{sanskrit_translations_of_islamic_texts}), expressly mention \MuhammadAbidda\ dictating the Arabic passages while he composes them into Sanskrit \parencite[73-74]{SarmaJaiSinghSanskrit}. As \textcite{KusubaPingree} remark, \Nayanasukhopadhyaya\ `did not simply render the Arabic commentary together with the original into Sanskrit literally, but expanded those passages that he found particularly difficult' (p.\thinspace 7). This suggests that between \MuhammadAbidda's  dictation of the Arabic passages and \Nayanasukhopadhyaya's translation of those passages into Sanskrit, they communicated directly or through an intermediary in a common link  language (perhaps, a colloquial dialect of Hindavī, Brajabhāṣā, or Rājasthānī). 

Bidirectional translations of texts between Sanskrit and other languages (\eg vernaculars like Hindustānī or Bāṅglā, or even European languages like English or German) continued beyond the eighteenth-century, with different methods and motivations \parencites[\eg \vid][]{Dodsontranslation, Raina, Gallien}. Texts in the exact sciences were included in many translation projects, and undoubtedly, they were repurposed to serve the ambitions of the benefactor and the beneficiary alike.  

\subsubsection{Why \Nityananda?} \label{nityananda_bio}

It is uncertain why \Nityananda\ was chosen to translate \MullaFarid's Persian \ZijiShahJahani\ into Sanskrit. Based on what we know, \Nityananda\ was not a decorated astronomer: he did not hold any titles like \textit{Jotik Rāi} or \textit{Vedāṅgarāya}, even though  \Shahjahan\ conferred such a title on \VedangarayaMalajit\ (\fl 1643) (\vid\ footnote~\ref{srimalaji_footnote}). \Nityananda\ identifies himself in the colophon of his \Sarvasiddhantaraja\ (1639) as a Gauḍa Brahmin of Mudgala \gotra\ (patronymic) from Indrapurī (Old Delhi), and provides a genealogy of his Brahmin ancestors beginning with his father: \Nityananda, son of \Devadatta, son of \Narayana, son of \Laksmana\ son of \IcchaDulinahatta\ \parencite[\eg \vid][228]{PetersonCatalogue}. Beyond this register of names, we have no reliable information on who these other Brahmins were, where they came from, or what works they wrote (if any).

Until any new evidence suggests otherwise, we believe \Nityananda's association with the Mughal court begins with \AsafKhanshort\ employing him in \circa \post\ 1628.\footnote{\,In contrast, \MullaFarid\ first joined the service of \KhanKhana---a prominent Mughal nobility
during the reigns of the Mughal emperors \Akbar\ and \Jahangir, `a man of the sword as well as of the pen' \parencite[75]{Lefevre}---in 1597 and remained in his service till he joined the court of \Shahjahan\ in 1628 \parencite[34]{Ghori}.} As I describe below (in  \S~\ref{zijshahjahani_mulla_farid}), a royal decree (\farman) was issued to bring Muslim and Hindu astronomers together to prepare the \ZijiShahJahani\ under \AsafKhanshort's supervision. This may have been the ticket for \Nityananda's entry to the Mughal court. His fluency in vernacular Hindi (as a resident of Delhi) and competency in Sanskrit astronomy (presumably, attested through testimony) might have brought him to \Shahjahan's court seeking patronage as a Sanskrit \jyotisa.
 
We do not know whether \Nityananda\ continued to remain at the Mughal court after composing the \Siddhantasindhu, or even if his association was ever exclusive to begin with. His second book, the \Sarvasiddhantaraja\ (1639), is a complex syncretism of Sanskrit \siddhantic\ astronomy and Islamicate theories. To my knowledge, there are no explicit references to any patrons in this work; however, judging by the scale and scope of the work, it seems very likely that he had continued access to intellectual and financial resources throughout its production \parencite[more on this in][]{MisraTD}.\footnote{\,A single extant manuscript of a text called the \Shahajahamganita, allegedly authored by \Nityananda, is currently held at the Anup Sanskrit Library in Bikaner (MS~5291, Serial \textnumero 787, 12 folia, injured, \vid\ \cite[CESS A3, p.\thinspace 174a]{PingreeCESS}; \cite[393]{AnupSanskritLibrary}). I have not been able to consult this manuscript to verify its professed authorship.} 

 

% \KhanKhanashort\ (1556--1627 \ce), better known as \Rahim\ in the world of Hindi-poetry and Nabbāba Khānakhānā in Sanskrit, was a polyglot man of letters and a patron  of Hindi, Sanskrit, and Persian literature. He composed the \Khetakautuka\ (\circa late sixteenth century \ce), a Sanskrit text of 124 verses explaining Persian \tajika\ astrology with transliterated Arabic/Persian technical terms \parencite[pp.\thinspace 79b--80a in Volume A2]{PingreeCESS}.}

% \Shahjahan's \jyotisa s like of \Nityananda, \Balabhadra\footnote{\,\Balabhadra\ (\fl 1629/53 \ce) composed his \Hayanaratna\ for \Shahjahan's second son \ShahShuja\ at Rāja\-mahala in 1629 \ce. The \Hayanaratna\ is a well-renowned Sanskrit digest (\textit{nabandha}) of excerpts from earlier Sanskrit writings on Arabic \tajika\ genethlialogy; edited by \textcite[forthcoming]{Ganstenhayanaratna}.}

% The Arabic divination method of geomancy (\khatt\ \alraml) appears as Sanskrit \ramala-texts from the beginning of the seventeenth-century, \eg \Ramalavaicitrya\ (\circa \ante\ 1657 \ce) of \Rama\ \parencite[pp.\thinspace 424a--425bin Volume A5]{PingreeCESS}. According to \textcite[79--80]{PingreeJyotihsastra}, these texts were introduced from Persian sources in the early Mughal period.

% A comprehensive analytical survey of Persian texts (originals and translations) authored between the thirteenth and nineteenth centuries in India has been undertaken by the \href{http://www.perso-indica.net/}{Perso-Indica Project}, an online peer-reviewed publication led by Fabrizio Speziale (EHESS, Paris) and Carl W.\@ Ernst (University of North Carolina, Chapel Hill) as editors-in-chief \parencite{PersoIndica}.


%%%%%%%%%%%%%%%%%%%%%%%%%%%%%%%%%%%%%%%%%%%%%%%%%%%%%%%%%%%%%%%%%%%%%%%%%%%%%%%%%%%%%%%%%
\subsection{Islamicate \zij es in Mughal India}\label{islamicate_zijes_mughal_india}
%%%%%%%%%%%%%%%%%%%%%%%%%%%%%%%%%%%%%%%%%%%%%%%%%%%%%%%%%%%%%%%%%%%%%%%%%%%%%%%%%%%%%%%%%

By the seventeenth century, Arabic and Persian astronomical texts were regularly studied at Islamic institutions of higher learning (\madrasa) in Mughal India; particularly, at those institutions that focused on teaching the rational sciences (\ulumalaqliyah).\footnote{\textcite[1--88]{Sufi} provides an extensive chronological study of the evolution of curriculum in Indian \madrasa s through the dynastic rule of the Ghaznavids, the Ghurids, the Delhi Sultanates, and eventually the Mughals.} \textcite[Table~1 on p.\thinspace 278]{Ansaritransmission} lists the names of prominent Islamicate scholars whose works are extant in several copies in Indian libraries. It includes the works of \abuNasr\ (d.\thinspace 1036), \Gilanifull\ (d.\thinspace 1029), \ibmHaytham\ (965--1041), \alBirunifull\ (973--1048), \alJaghmini\ (\circa early thirteenth century), \alTusifull\ (1201--1274), \alShirazifull\ (1236--1311), \alKashifull\ (d.\thinspace 1436), \UlughBegfull\ (1394--1499), and many others; also \vid\ \textcite[Section~III on pp.\thinspace 279--281 and Appendix~I on pp.\thinspace 288--294]{Ansaritransmission}. The large numbers of manuscript witnesses suggest the prevalence of these works in the repertoire of Muslim scholars (\ulama, \fudala) in early-modern India. 
% Of notable significance are the works of  \abuNasr\ (d.\thinspace 1036 \ce), \ibmHaytham\ (965--1041 \ce), \alBirunifull\ (973--1048 \ce), \alJaghmini\ (\circa early thirteenth century \ce), \alTusifull\ (1201--1274 \ce), \alShirazifull\ (1236--1311 \ce), \alKashifull\ (d.\thinspace 1436 \ce), \UlughBegfull\ (1394--1499 \ce), \alRumifull\ (\circa 1440 \ce), \aliQushjifull\ (d.\thinspace 1474 \ce), \alBirjandi\ (d.\thinspace 1525 \ce), and \alAmilifull\ (1547--1622 \ce), \vid\ \textcite[Section~III on pp.\thinspace 279--281 and Appendix~I on pp.\thinspace 288--294]{Ansaritransmission}.
% \footnote{\,\textcite[Section~III on pp.\thinspace 279--281]{Ansaritransmission} describes various astronomical works of Islamicate scholars that were studied in Indian \madrasah s. Also, \vid\  Appendix~I (\Ibid, pp.\thinspace 288--294) for a catalogue-list of astronomical works in India.}

The lists of \zij es enumerated in the \AiniAkbari\ of \AbulFadlAllami\ (1551--1602), the chronicler of \Akbar, and the \ZijiShahJahani\ of \MullaFarid\ composed during the rule of \Shahjahan\ provide further information on the astronomical tables in the imperial library (\kitabkhana) of early seventeenth-century Mughal India \parencite[Appendices~A and B on pp.\thinspace 45--48]{Ghori}. 
% \footnote{\,\textcite{Seyller} provides a comprehensive description and valuation of the manuscripts in the collections of the Mughal library. The paper begins by observing that `[s]hortly after the death of Emperor Akbar in 1605, an inventory of the vast holdings of the imperial Mughal library recorded a total of 24,000 volumes with a value of 6,463,731 rupees.' (p.\thinspace 243).}

An important \zij\ in these accounts is the \ZijSultani\ (alias \ZijUlughBeg\ or \ZijJadidiGurani) of \UlughBegfull\ composed by a collaborative team of astronomers (\alRumi, \alKashi, \UlughBeg, and \aliQushji) at the Observatory of Ulugh Beg in Samarqand in 1438/39.\footnote{\,\Vid\ \textcite[277--279]{RosenfeldandIhsanoglu}, \textcite[54]{KingSamso}, and \textcite[pp.\thinspace 127b--128a and pp.\thinspace 166b--167b]{Kennedysurvey} for a descriptive survey of the contents and manuscripts of the \ZijUlughBeg.} As \textcite[p.\thinspace 581a]{Ansarisurvey} notes, the \ZijUlughBeg\ was translated into Sanskrit (\JicaUlugbegi\thinspace\footnote{\,The \JicaUlugbegi\ (or \Ulakabegijica) is extant in a few fragmentary manuscripts containing only tables and star catalogues. The largest (and most complete) manuscript appears to be MS~Museum 45 held at the Maharaja Sawai Man Singh II Museum at the City Palace of Jaipur. According to \textcite[135]{PingreeJaipur} it contains 100 folia measuring  17 $\times$  28\textonehalf\ cm, contains only tables written in \Nagari\ numerals, and was acquired from S\={u}rata by Nandar\={a}ma Jo\'{s}\={i} for 20\textonehalf\ rupees. \label{jicaulughbeg}}) by a consortium of Muslim and Hindu scholars led by \ShahFathullahShirazi\ (d.\thinspace 1589) during the reign of \Akbar. From the sixteenth century, the preeminence of the \ZijUlughBeg\ in Mughal India made its structure the standard with which subsequent \zij es were composed. The \ZijiRahimi\ and the \ZijiShahJahani, both composed by \MullaFarid, are two such examples.\footnote{\,\textcite[Section~3.2 on pp.\thinspace 582--583]{Ansarisurvey} reviews the structure and contents of \MullaFarid's \ZijiRahimi\ (\circa 1615/17) dedicated to his patron \KhanKhana, a prominent Mughal nobility
during the reigns of the Mughal emperors \Akbar\ and \Jahangir.}

%%%%%%%%%%%%%%%%%%%%%%%%%%%%%%%%%%%%%%%%%%%%%%%%%%%%%%%%%%%%%%%%%%%%%%%%%%%%%%%%%%%%%%%%%
\subsection{The \ZijiShahJahani\ (\circa 1629/30) of \MullaFarid}\label{zijshahjahani_mulla_farid}
%%%%%%%%%%%%%%%%%%%%%%%%%%%%%%%%%%%%%%%%%%%%%%%%%%%%%%%%%%%%%%%%%%%%%%%%%%%%%%%%%%%%%%%%%

\MullaFarid's \ZijiShahJahanifull\thinspace\footnote{\,The \ZijiShahJahanifull\ `Grand Accomplishment of the Second Lord of the Conjunction, the \Zij\ of \Shahjahan' uses the royal epithet of \Shahjahan\ as the `Second Lord of the Conjunction' born on the auspicious conjunction (\qiran) of Jupiter and Venus at his natal hour on 5 January 1592 \ce. The use of `second' in the title is to establish a direct descent from the first Lord of Auspicious Conjunction, \Timur\ \parencite[1105--1106]{Chann}.} (\ZijiShahJahani\ for short) is a set of astronomical tables written at the behest of \AsafKhan\ (d.\thinspace 1641), the prime minister (\vazir\idafaconsonant\ \azam) and father-in-law to \Shahjahan. It was commissioned to institute a new calendar of \Shahjahan, the \Shahishani, beginning on the first day of Farvardin in his ascensional year 1037 \ah\ (21 March 1628). This was in keeping with previous regnal year calendar like \Jalali\ era (epoch~21 March 1079) of the Seljuk Sultan \MalikShah\ or the \Ilahi\ `Divine Era' (epoch~20 March 1555/56) of \Shahjahan's grand father \Akbar. \textcite[Section~3.2 on pp.583--585]{Ansarisurvey}, \textcite[34--36]{Ghori}, \textcite[357--358]{RosenfeldandIhsanoglu}, and \textcite[307]{STMI} survey the context, the structure, and manuscripts of the \ZijiShahJahani. For our present purpose, we note the following points from these surveys:
\begin{enumerate}[topsep=0pt]
    \item The \ZijiShahJahani\ (like the \ZijUlughBeg) consist of a detailed prolegomenon (\muqaddima) consisting of five sections (\qism, pl.\thinspace \aqsam) followed by four discourses (\maqala, pl.\thinspace\maqalat) on four different subjects, each containing several chapters (\bab, pl.\thinspace \biban) that are further divided into sections (\fasl, pl.\thinspace\fusul).
    \item \MullaFarid\ classifies the \ZijiShahJahani\ as a \zijiHisabi\ or a `computational table that revises and updates the parameters' of the \ZijUlughBeg\ (and different from a \zijiRasadi\ or an `observational table based on findings from direct observations').
    \item A lack of time to conduct newer observations, in part due to the advancement of age and the ailing health of \MullaFarid, meant that large parts of the \ZijiShahJahani\ were reproductions of corresponding parts of the \ZijUlughBeg. However, as \textcite[585]{Ansarisurvey} notes, the tables in the \ZijiShahJahani\ outnumber those in the \ZijUlughBeg. \MullaFarid\ includes the auxiliary tables for simplification (\tashil) so that the true longitudes (\taqvim) of celestial objects can be computed directly (without any interpolation). 
    \item By a royal decree of \Shahjahan, the \ZijiShahJahani\ was to be translated into `the language of Hindustan by Indian astronomers in consultation with Persian astronomers, for the sake of public utility'.\footnote{\,Excerpted from the \MulakhkhasiShahJahanNama, an abridged history of \Shahjahan\  written by his seventeenth-century court chronicler \TahirKhan, alias \InayatKhan, \vid\ \textcite{Ansarisurvey} for \TahirKhan's Persian text (Appendix~II.\thinspace A4 on p.\thinspace 597) and its English translation (p.\thinspace 584a)} \textcite[34]{Ghori} also observes that the \ZijiShahJahani\ was prepared by \MullaFarid\ in `collaboration of his brother \MullaTayyib\ and other scholars of Muslim and Hindu astronomy under the over-all supervision of the Vazir \AsifKhan\ [\sic]'.     
\end{enumerate}

\subsubsection{Manuscripts of the \ZijUlughBeg}\label{manuscripts_zij_shahjahani}
In this study, I outline the twenty-two chapters (\biban) in the second discourse (\maqala\idafaconsonant\ \duvum) of the \ZijiShahJahani, and among these, I examine the sixth chapter. To this end, I have consulted (parts of) the two manuscripts of the \ZijiShahJahani\ that were available to me. Table~\ref{mss_description_zij_shah_jahani} provides a description of these manuscripts and their assigned sigla.

\begin{table}[!htbp]
\centering
\renewcommand{\arraystretch}{1.5}
\renewcommand{\baselinestretch}{1.25}\selectfont
\begin{tabularx}{\textwidth}{c X}
\hline
Siglum & Manuscript description\\
\hline
\SjA & MS~2735 (MS Ind.\@ Inst.\@ Pers.\@ 12) from the Bodleian Library Oxford, entitled \textbf{Kâranâma i ṣâḥibkirân thânî zîj i shâhjahânî}, 380 folia (incomplete) with 25 lines per folio, 13\,¼ $\times$ 9\,⅜ inches, Persian \Nastaliq, written with red and black ink, \circa 17\textsuperscript{th} century \ce\ \parencite[p.\thinspace 61b]{Beeston}. \\
& Mr Alasdair Watson, the Bahari Curator of Persian Collections at the Bodleian Library, very kindly provided me with photographs of folia that include the sixth chapter of the second discourse. The folio numbers do not appear on the images; Mr Watson identified them as ff.\thinspace 21b--22a (personal communication). I have not had the opportunity to inspect the other folia of this manuscript. The summary of the Persian chapter-titles in \S~\ref{chapter_title_comparision_persian_sanskrit} rely entirely on the reading in MS~\SjB.\\
\SjB & MS~Or.\thinspace 372 (Sch 7909) from the British Library London, labelled \textbf{Farìd Ibráhím Zíj E Sháhjaháni Persian} (on the microfilm cover), entitled \tfarsi{کارنامه صاحبقران ثانی زیج شاه جهانی} %
\ZijiShahJahanifull, 419 folia with 31 lines per folio, 13\,¾ $\times$ 8\,½ inches, Persian \Nastaliq, treble-ruled text frame, \circa 17\textsuperscript{th} century \ce\ \parencite[pp.\thinspace 459b--460b]{Rieu}. \\
& Folio numbers are written at the top left corner of \textit{folium rectum} (b-side) in western Arabic numerals, (possibly) by a European owner/cataloguer/librarian.\\
& I am grateful to Dr Benno Van Dalen (from the \textit{Ptolemaeus Arabus et Latinus} project at the Bayerische Akademie der Wissenschaften München) for providing a digitised black-and-white photocopy of this manuscript to me. \\
\hline
\end{tabularx}
\caption{Description of the manuscripts of the \ZijiShahJahani}
\label{mss_description_zij_shah_jahani}
\end{table}

To my knowledge, the \ZijiShahJahani\ (or any part of it) has never been edited or translated into any  major European or Indian (vernacular) language in modern times. Historically, the only translation of the \ZijiShahJahani\ is \Nityananda's \Siddhantasindhu\ described in the next subsection.


%%%%%%%%%%%%%%%%%%%%%%%%%%%%%%%%%%%%%%%%%%%%%%%%%%%%%%%%%%%%%%%%%%%%%%%%%%%%%%%%%%%%%%%%%
\subsection{The \Siddhantasindhu\ (\circa early 1630) of \Nityananda}\label{siddhantasindhu_nityananda} 
%%%%%%%%%%%%%%%%%%%%%%%%%%%%%%%%%%%%%%%%%%%%%%%%%%%%%%%%%%%%%%%%%%%%%%%%%%%%%%%%%%%%%%%%%

\Pandita\ \Nityanandafull\ completed his translation of \MullaFarid's \ZijiShahJahani\ in the early 1630s and named it \Siddhantasindhu\ `the Ocean of \mbox{\Siddhanta s}'. This enormous work occupied around 440 folia measuring 45 $\times$ 33 cm (approximately). At the time, ten copies of this work were made and distributed among the Muslim nobility of northern Mughal India (more on this in \S~\ref{circualtion_siddhantasindhu}). Today, however, only a handful of near-complete manuscripts of this work survive. And among these, the four manuscripts at the Maharaja Sawai Man Singh II Museum at the City Palace of Jaipur (India) are the best preserved copies. One of these manuscripts (Khasmohor 4960, part of the \khasmohor\ or `special seal' collection) bears the royal insignia of \Shahjahan\ himself. \textcite[138--143]{PingreeJaipur} describes these four manuscripts at the City Palace Museum Library in Jaipur. The catalogue references of the other (fragmentary) manuscripts located elsewhere can be found in \textcites[CESS A3, p.\thinspace 173b, and CESS A5, p.\thinspace 184a]{PingreeCESS}.

\subsubsection{Manuscript of the \Siddhantasindhu} \label{manuscript_siddhantasindhu} 
Parallel to the selection from the \ZijiShahJahani, I outline the twenty-two chapters (\adhyaya s) of the second part (\dvitiya\ \kanda) of the \Siddhantasindhu, and among these, I focus on the sixth chapter. I have consulted the only copy of the \Siddhantasindhu\ made available to me for this purpose by the City Palace Museum Library in Jaipur. A description of the manuscript, and its assigned siglum, is given in Table~\ref{mss_description_siddhantasindhu}.

\begin{table}[!t]
\centering
\renewcommand{\arraystretch}{1.5}
\renewcommand{\baselinestretch}{1.25}\selectfont
\begin{tabularx}{\textwidth}{c X}
\hline
Siglum & Manuscript description\\
\hline
Kh & MS~4962 from the Khasmohor Collection at the City Palace Library of Jaipur, entitled (in Hindustānī) \tsans{pothi siddhaa.mtasi.mdhu kii} `Book (\textit{pothī}) of \Siddhantasindhu', 436 folia (incomplete: missing ff.\thinspace 1 and 3; tears and damages on f.\thinspace 2) with 21--30 lines per folio, 37 $\times$ 25 cm, Sanskrit \Nagari, written with red and black ink parallel to the shorter edge, double-ruled text frame, left-binding with side-sewing stitches, red-and-blue striped cloth-covered boards and book flap, belonging to Jagannātha Jośī and acquired for 100 rupees, \circa early 18\textsuperscript{th} century \ce\ \parencite[143]{PingreeJaipur}. \\
& Folio numbers are written at the bottom left corner of \textit{folium versum} in \Nagari\ numerals by the same hand as the scribe.\\
&\textbf{Remark}\quad The metrical verses are often introduced by the word \tsans{cha.mda.h} (\textit{chaṃḍaḥ}) `metre', \eg f.\thinspace 4v:\thinspace 4, 15~Kh. Its abbreviated form \tsans{cha.m} (\textit{chaṃ}) also appears in several places, \eg on f.\thinspace 5v:\thinspace 7~Kh. Overall, Kh uses the double-\textit{daṇḍa} `\tsans{||}' to indicate half-stanza breaks, and frequently places the verse number, each time beginning with one, between two sets of double-\textit{daṇḍa}s, \eg \tsans{||~1~||}. The prose passages are unnumbered.\\
& I gratefully acknowledge Dr Chandramani Singh, (retired) head curator of the Maharaja Sawai Man Singh II Museum Library, for her assistance in helping me get a digital copy of this rare and private manuscript.\\
\hline
\end{tabularx}
\caption{Description of the manuscript of the \Siddhantasindhu}
\label{mss_description_siddhantasindhu}
\end{table}

There are no published editions, translations, or studies of \Nityananda's \Siddhantasindhu\ to my knowledge.\footnote{\,\textcite[231--232]{PetersonCatalogue} provides an excerpt containing the first thirty-two verses (from the prolegomenon) and the colophon (from the end of the second part) of \Nityananda's \Siddhantasindhu. This is presumably transcribed from MS~RORI (Alwar) 2627 = MS~2014 Alwar, 441 folia, copied in 1855 \ce\ \parencite[CESS A5, p.\thinspace 184a]{PingreeCESS}; however, \citeauthor{PetersonCatalogue} does not identify the shelf mark of the manuscript. \textcite[128--129]{Minkowskilearnedbrahmin} makes a few remarks on \Nityananda's \Siddhantasindhu\ (based on \citeauthor{PetersonCatalogue}'s extract), while \textcite[269--27]{PingreeSSR} summarises (very briefly) his observations on the \Siddhantasindhu\ based on the manuscripts held at the Maharaja Sawai Man Singh II Museum at the City Palace of Jaipur.\label{peterson_catalogue_siddhantasindhu_reference}} At a very minimum, a comprehensive description of the structure and contents of the entire \Siddhantasindhu\ is certainly needed; however, such a task lies well beyond the scope of this study. Instead, I present below a few salient remarks from the \Siddhantasindhu\ that relate to the \ZijiShahJahani.

% \pagebreak %%%%%%%%%%%%%%%%%%%%% PAGEBREAK FOR FORMATTING ON PDF, CAN BE REMOVED LATER 

\begin{enumerate}[topsep=0pt]
    \item \textbf{On the content of MS~Kh}\quad \label{ms_kh_content_siddhantasindhu}The \Siddhantasindhu\ also begins with a detailed prolegomenon (\textit{granthārambha}, synonymous with \muqaddima) consisting of five sections (\prakara, synonymous with \qism) on ff.\thinspace 3r--11v~Kh\footnote{\,I cite manuscript references in the format $\langle$f.\thinspace folio$_\text{\#}$~Siglum$\rangle$ or $\langle$f.\thinspace folio$_\text{\#}$:\,line$_\text{\#}$~Siglum$\rangle$ throughout this paper. For instance, `f.\thinspace 22r: 9--11 Kh' indicates lines 9 to 11 on f.\thinspace 22r in Kh. \label{reference_folio_line_syntax}} (incomplete). Thereafter, it follows the structure of the \ZijiShahJahani\ with each part (\kanda, identified with \maqala) addressing a different subject and containing several chapters (\adhyaya, synonymous with \bab). However, 
    only the first three \maqalat\ of the \ZijiShahJahani\ appear to have been translated in the \Siddhantasindhu\ (as the three \kanda s); the fourth \maqala\ on miscellaneous astronomical calculations does not appear in Kh. A brief description of the content of Kh is as follows:
    \begin{itemize}
        \item The first part (\prathama-\kanda) with seven chapter describing the different calendrical eras (\textit{śāka}), \viz Arabic or Hijri (\textit{ārbīya}), \Shahjahan's (\textit{shāhjahām̐nīya}), Roman (\textit{raumīya}), Persian (\textit{phārasīya}), Malakī/Jalālī (\textit{ma\-lakīya}), Saṃvat (\textit{Hindukīya}), and Chinese-Uighūr Animal (\textit{khitāyīya-tur\-kīya}) on ff.\thinspace 12r--16v~Kh.
        \item The second part (\dvitiya\ \kanda) with twenty-two chapters describing various topics on finding the desired time (\textit{abhimata-samaya}) and the ascendant at that time (\textit{tātkālika-lagna}), as well as other topics related to it, on ff.\thinspace 17r--28v~Kh. More on this in \S~\ref{chapter_title_comparision_persian_sanskrit}.
        \item Tables (\kosthaka s) from ff.\thinspace 29r--97v~Kh.
        \item The third part (\trtiya-\kanda) with fifteen chapters describing the true (\sphuta) position and motion of celestial objects, and other topics related to it on ff.\thinspace 98v--111v~Kh.
        \item Tables (\kosthaka s) from ff.\thinspace 112r--436v~Kh.
    \end{itemize}
    %%%%
    \item \textbf{On \Shahjahan}\quad\label{shahjahan_epithet_siddhantasindhu} In the preamble, \Nityananda\ extols \Shahjahan\ with his encomiastic poetry \parencite[\eg \vid][129]{Minkowskilearnedbrahmin} and transliterates his Persian regnal epithet into \Nagari\ (on f.\thinspace 5v:\thinspace 15--16~Kh) as    
    \begin{multicols}{2}
    \tsans{abala-mujaphara-"saahibbadiina-\newline mahammada-saahiba-kiraana-saanii-\newline "saahajahaa-baadi"saaha-gaajii}
    \columnbreak
    
    \textit{abala-mujaphara-śāhibbadīna-mahammada-sāhiba-kirāna-sānī-śāhajahā-bādiśāha-gājī}
    \end{multicols}
        In his \Padshahnama, the seventeenth-century Mughal chronicler \AbdalHamidLahori\ states that upon ascending to the throne, Prince Shahāb al-Dīn Muḥammad Khurram (\Shahjahan) assumed the regnal name `Abū\Alif l-Muẓaffar Shahāb al-Dīn Muḥammad Ṣāhib-i Qirān-i Thānī' \parencite[6]{Elliot}. Among several other imperial epithets, \Shahjahan\ was called \tfarsi{صاحب قران ثاني} (\textit{ṣāhib-i qirān-i thānī}) `Second Lord of the Conjunction', \tfarsi{پادشاه غازی} (\textit{pādshāh-i ghāzī}) `Conqueror of Emperors' and \tfarsi{شاه جهان} (\textit{shāh-i jahān}) `King of the World'.
        
       \Nityananda\ traces the male ancestors of \Shahjahan\ from Tīmūr (\textit{taimūra}), Mīrān Shāh (\textit{mīrā-sāhā}), Sulṭān Muḥammad (\textit{sullā-mahaṃma}), Abū Sa\Ayn īd (\textit{abūsayīda}), Umar Shaykh (\textit{umara-śekha}), Bābur (\textit{bābara}), Humāyūn (\textit{humāū}), Akbar Shāh Jalāl al\=/Dīn (\textit{akabaraḥ śāha\-jallāladīna}), Jahāngīr (\textit{śrī-jahāṃgīra}), and finally, Shāh Jahān (\textit{śrīmān-sāhajahāṃ}) \parencite[vv.\thinspace 5--11 on p.\thinspace 230]{PetersonCatalogue}.\footnote{\,MS~Kh begins on f.\thinspace 2r in the middle of describing the genealogy of \Shahjahan. With the first folio missing and heavy damage to the second, I have relied on \citeauthor{PetersonCatalogue}'s transcription (\vid\ footnote~\ref{peterson_catalogue_siddhantasindhu_reference}) to fill in the missing parts.}
    %%%%    
     \item \textbf{On \AsafKhanshort}\label{asaf_khan_siddhantasindhu}\quad \AsafKhan, the prime minister (\vazir\idafaconsonant\ \azam) of \Shahjahan, is mentioned by name in the \Siddhantasindhu. On f.\thinspace 2v:\thinspace 10--11~Kh, we find the name \tsans{mantrii vaasaaphakhaa/} `Minister \AsafKhanshort\ (\textit{vāsapha-khām̐})'.
     
     Through vv.\thinspace 21--23 on the same folio (lines 8--18), \Nityananda\ generously praises him as
     \tsans{yo raajyaahvayama.n.dapasya sud.r.dha.h stambha.h} (line 15) `he who is the steadfast pillar (\textit{sudṛḍha-stambha}) of [this] pavilion called the Empire (\textit{rājyāhvaya-maṇḍapa})' and \tsans{var.naa"sramapaalayat} (lines 17--18) `[he who is] protecting the [Hindu social system of] \textit{varṇāśrama}'.  
    %%%%
    \item \textbf{On \MullaFarid}\quad \label{mulla_farid_sanskrit_sidhantasindhu}\Nityananda\ identifies \MullaFarid\ by his name. On f.\thinspace 6r:\thinspace 2~Kh, \Nityananda\ calls him \tsans{mullaaphariida.m ibaraahimaputra.m .dhilliinivaasina.m} `\MullaFarid\ (\textit{mullā-pharīda}), the son of Ibrāhīm (\textit{ibarāhima-putra}) [and] resident of Delhi (\textit{ḍhillī-nivasin})'. This description agrees with \textit{ibn} [\textit{Ḥāfiẓ}] \textit{Ibrāhīm} `son of Ibrāhīm, [a man who has memorised the Qur\Alif ān]' and \textit{Dehlavī} `resident of Delhi' in \MullaFarid's full name. 
    %%%%
     \item \textbf{On the \ZijiShahJahani}\quad \label{zij_shahjahani_sanskrit_siddhantasindhu}
    Further along, on f.\thinspace 6v:\thinspace 19--20~Kh, we find the name of \MullaFarid's text, transliterated into \Nagari\ from the Persian \ZijiShahJahanifull, as
    \begin{multicols}{2}
    \tsans{kaaranaamai-saahiba-kiraana-saanii \newline jiica-"saaha-jahaanii}
     \columnbreak
    
     \textit{kāranāmai-sāhiba-kirāna-sānī \newline jīca-śāha-jahānī}
    \end{multicols}
    %%%% 
    \item \textbf{On \zij es and their types}\label{zij_types_defintion_siddhantasindhu}\quad    \Nityananda\ first distinguishes between the terms \zij, \tashil, and \taqvim\ on f.\thinspace 7r:\thinspace 9--10~Kh. According to him,
    {\par
    \tsans{jiica iti siddhaanta.h | tasahiila iti saara.nii | takaviima iti grahasphu.tatvam |}\\
    `\zij\ (\textit{jīca}) is \Siddhanta\ (canon); \tashil\ (\textit{tasahīla}) is \sarani\ (table); \taqvim\ (\textit{takavīma}) is the true position of a celestial object (\textit{graha-sphuṭatva}) [in other words, an ephemeris].'\par}
    On f.\thinspace 8v:\thinspace 2--3~Kh, he defines a \zij\ as \tsans{yasmingranthe sthuulasuuk.smaga.nitaani bhavanti tasya naama jiica iti~|} `The book (\textit{grantha}) in which both gross (\textit{sthūla}) and subtle (\textit{sūkṣma}) computations (\textit{gaṇita}) are found, that is called a \zij\ (\textit{jīca})'.   
    
    \Nityananda, like \MullaFarid, also classifies \zij es into the two categories of \zijiRasadi\ and \zijiHisabi: 
    \begin{itemize}
            \item On f.\thinspace 7v:\thinspace4--5~Kh, he defines \tsans{jiica-rasadii} (\textit{jīca-rasadī}) as that work (\textit{tantra}) which is well-established (\textit{dṛḍhī-kṛtya}) by the rules of observations (\textit{rasada-vidhāna}) and state the motion of celestial objects (\textit{graha-bhukti}) with tables (\kosthaka s); and
            \item On f.\thinspace 7v:\thinspace 27--19~Kh, he defines \tsans{jiica-hisaabii} (\textit{jīca-hisābī}) as the work containing tables (\kosthaka s) that correct (\textit{śodhyate}) previous tables in the table-writing tradition (\textit{koṣṭhaka-lekhaka-paramparā}) or those that bring out the genuine result (\textit{vāstava-phala}) by simple procedures (\textit{sugama-prakāra}) of computations (\textit{gaṇita}). 
        \end{itemize}
     %%%%
% \pagebreak %%%%%%%%%%%%%%%%%%%%% PAGEBREAK FOR FORMATTING ON PDF, CAN BE REMOVED LATER 

     \item \textbf{On \UlughBeg\ and other Islamicate astronomers, and their \zij es}\label{ulugh_beg_others_zijes_siddhantasindhu}\quad   \Nityananda\ identifies \UlughBeg\ by name in several places in the preamble. For instance, on f.\thinspace 7r:\thinspace 26~Kh, he refers to \UlughBeg\ as \tsans{mirajaa ulaga-bega} (\textit{mirajā ulaga-bega}), and on the very next line, informs us of \UlughBeg's demise with \tsans{parame"svarastasya svargavaasa.m karotu} `May God (\textit{parameśvara}, \lit `supreme lord') grant him residence in heaven (\textit{svargavāsa})'. The \ZijUlughBeg\ is translated as \tsans{jiica-ulaga-begii} (\textit{jīca-ulaga-begi}) on f.\thinspace 7v:\thinspace5~Kh.
     
    Along with Ulugh Beg, \Nityananda\ also described the names and works of several Islamicate astronomers (on ff.\thinspace 7v--8v~Kh). For example:
     \begin{itemize}
         \item the \tsans{jiica-jaame} and \tsans{jiica-baaliga} of \tsans{go"siyaara} \\
         \alZijalJami\ (\textit{jīca-jāme}) and \alZijalBaligh\ (\textit{jīca-vāliga}) of \Gilanifull\ (\textit{gośiyāra}), or
         \item the \tsans{jiica-khaakaanii-takamiila-jiica-yiila\-khaanii} 
    of \tsans{maulaanaa jama"seda kaa"sii} \\
    \KhaqaniZijfull\ (\textit{jīca-khākānī-takamīla-jīca-yīlakhānī}) of \alKashifull\ (\textit{maulāna-jamaśeda-kāśī}).
     \end{itemize}
     The names of these \zij es, and the order in which they are listed, appear to be identical in both the \ZijiShahJahani\thinspace\footnote{\,\textcite[48]{Ghori} provides the list of \zij es in \MullaFarid's \ZijiShahJahani, \vid\ discussion in \S~\ref{islamicate_zijes_mughal_india}.}  and the \Siddhantasindhu.
    \end{enumerate}

\paragraph{Motivation and purpose of composition}\label{motivation_purpose_of_siddhantasindhu} 
The twenty-fourth verse from the prolegomenon of \Nityananda's \Siddhantasindhu\ is barely legible on the damaged second folio (verso) of MS~Kh; however, it is attested in \textcite[231]{PetersonCatalogue} from the Alwar manuscript. The verse (in the \sardulavikridita\ metre) describes how \AsafKhanshort, having derived this inspiration (\textit{preraṇā}) from \Shahjahan, ordered \Nityananda\ to compose a proper treatise (\textit{su-tantra-karaṇe}) for the benefit of people (\textit{loka-upakāra}).\footnote{\,\Nityananda\ eulogises \Shahjahan\ as the `crest jewel of the ornamental crown of kings' (\textit{nṛpāla-mukuṭa-ālaṅkāra-cūḍāmaṇi})---an epithet he repeats in the colophons of every part of the \Siddhantasindhu, \eg \vid\ Part~II colophon on p.\thinspace \pageref{partII_colophon_sanskrit}.} Therefore, as \Nityananda\ says, he endeavoured to compose (\textit{kartum samīhe}) to compose the \Siddhantasindhu, a pure (\textit{amala}) and clear (\textit{sphuṭa}) Ocean of \Siddhanta s (\textit{siddhānta-sindhu}), resembling the illustrious \ZijiShahJahani\ (\textit{śrīmat-sahājahāṃ-prakāśam}). 


\paragraph{Style of composition} \label{prose_poetry_mixed_form}   
\Nityananda's \Siddhantasindhu\ is a mixture of prose-sentences (\textit{gadya}) and metrical verses (\textit{padya}). The transition from prose to poetry is ubiquitous throughout the text. It is worth noting that most \textit{explicit} Sanskrit translations of Islamicate astronomical texts (\eg the texts listed in Table~\ref{sanskrit_translations_of_islamic_texts}) are entirely in prose. I discuss a few aspects of \Nityananda's choice of using prose and poetry in relation to Part~II.6 in \S~\ref{structure_content_zij_sindhu_chapter_six}.
    
\subsubsection{Circulation of the \Siddhantasindhu} \label{circualtion_siddhantasindhu}

According to two of the four manuscripts of \Nityananda's \Siddhantasindhu\ held at the Maharaja Sawai Man Singh II Museum at the City Palace of Jaipur, there were nine copies of the \Siddhantasindhu\ prepared for distributing among notable seventeenth-century Mughal elites (mostly, Mughal \Subadar s or Provincial Governors), and an author's copy for \Nityananda\ \parencite[142]{PingreeJaipur}. A note, in vernacular Hindi, on f.\thinspace 443v of MSS~Khasmohor 4960 and Museum 23,\footnote{\,MS~Khasmohor 4960 belonged to Pīthīnātha in 1717; the horoscope of his son (dated Thursday 10 October 1717) is written on f.\thinspace 443. It is said to have been previously purchased from Manasārāma in 1696 for 250\thinspace ½ rūpas (f.\thinspace 444). Both these notes appear in vernacular Hindi \parencite[138--142]{PingreeJaipur}. The provenance of MS~Museum 23 is previously described in footnote~\ref{ss_museum_23_description}.} begins by stating that the original Sanskrit text (\textit{jīcamūlakarī}) remains in the library (\textit{kitābkhāṃnā}) of the emperor (\textit{pātiśāha}, Pādishāh). Individual copies of the text were given to the following recipients:   
\begin{enumerate}[topsep=0pt]
    \item \AzamKhan\ (\textit{ājama khāṃ}) of Bengal (\textit{ba}[\textit{ṃ}]\textit{gāla}), the governor of Bengal from 1632 to 1635;
    \item \AbdallahKhanFiruzJung\ (\textit{avdullaha khaṃ}) of Patna (\textit{paṭaṇā}), the governor of Bihar from 1632 to 1639;
    \item Ṣāḥib Ṣuba[dār?] of Benaras (\textit{banārasī sahava} \textit{sūva}/\textit{mūva}), unidentified;
    \item \ItiqadKhan\ (\textit{itakada khāṃ}) of Delhi (\textit{dillī}), also known as \MirzaShapur\ (d.\thinspace 1650), the brother of \AsafKhanshort;
    \item \KhwajaSabirKhanDauran\ (\textit{khāṃ nadorā}) of Ujjain (\textit{ujjayaṇa}), also known as \NasiriKhan\ (\textit{navaśeri khāṃ}), governor of Malwa from
    1631 to 1638;
    \item \MahabatKhanKhaniKhanan\ (\textit{mahavata khā[ṃ] khāṃnakhāṃnā}) in Burhanpur (\textit{burahānapura}), \circa \post\ 1633;
     \item \VazirKhan\ (\textit{ujīra khāṃ}), also known as \HakimShaykhIlmalDinAnsari, of Lahore (\textit{lāhora}), governor of Lahore from 1628 to 1639;
     \item \ZafarKhanAhsan\ (\textit{japhara khāṃ}) of Kashmir (\textit{kaśmīra}), the governor of Kashmir from 1632 to 1639 and from 1642 to 1646; and
     \item an unnamed recipient in Multan (\textit{mulatāna}).
\end{enumerate}    


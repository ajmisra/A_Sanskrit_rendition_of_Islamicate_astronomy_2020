%%%%%%%%%%%%%%%%%%%%%%%%%%%%%%%%%%%%%%%%%%%%%%%%%%%%%%%%%%%%%%%%%%%%%%%%%%%%%%%%%%%%%%%%%
\subsection{Chronology and influence}\label{chronology_influence}
%%%%%%%%%%%%%%%%%%%%%%%%%%%%%%%%%%%%%%%%%%%%%%%%%%%%%%%%%%%%%%%%%%%%%%%%%%%%%%%%%%%%%%%%%

\Nityananda's \Siddhantasindhu\ is a translation of \MullaFarid's \ZijiShahJahani. It was completed in the early 1630s, making it one of the earliest Sanskrit translations of a Persian \zij. The \JicaUlugbegi, a Sanskrit translation of \ZijUlughBeg\ commissioned during \Akbar's reign (\ante\ 1605), is perhaps the earliest example. However, the extant fragmentary manuscripts only contain tables and star catalogues (\vid\ footnote~\ref{jicaulughbeg}) and not the canon (text).\footnote{\,The \JicaUlugbegi\ was prepared by a consortium of astronomers led by \AmirFathullahofShiraz, with the assistance of Sanskrit interpreters (\vid\  \S~\ref{Mughal_court}). As \textcite[367]{Sarmajyotisaraja} has pointed out, some of these interpreters were Sanskrit \jyotisa s: \eg Kishan Joshī is identified as  \Krsnadaivajna\ (\fl \circa 1600/25). If the authors' version of \JicaUlugbegi\ contained the canon of the \ZijUlughBeg, the following two reasons make the loss of the Sanskrit canon a notable (and regrettable) event.
\begin{itemize}
    \item \MullaFarid\ copied large parts of the \ZijUlughBeg\ (almost) verbatim in his \ZijiShahJahani; these parts have been translated by \Nityananda\ in his \Siddhantasindhu. The Sanskrit canon in the \JicaUlugbegi\ may have served as \Nityananda's model text to recopy just as easily as the \ZijUlughBeg\ served \MullaFarid.
    \item \MullaFarid\ was a direct pupil of \AmirFathullahofShiraz, and is believed to have `learned the rational sciences (\UlumiAqliyah) including astronomy and astrology' from him \parencite[720]{Ansarimedievalindia}. This suggests that, at the very least, \MullaFarid\ (and by association, \Nityananda) would have been familiar with a version of the \JicaUlugbegi\ closer to the original than what is currently extant.
\end{itemize}} % 

In contrast, as \S~\ref{siddhantasindhu_nityananda} describes, the \Siddhantasindhu\ includes tables as well as three (out of the four) distinct portions (\kanda s `parts' in Sanskrit or \maqalat\ `discourse' in Persian) of the canon of \ZijiShahJahani. The full extent of similarity between the content of these two texts can only be determined by an extensive future study; for our present discussion, however, we note that the \Siddhantasindhu\ expressly declares itself to be a Sanskrit version of the Persian original. Other Sanskrit texts in mathematical astronomy, \eg the three \siddhanta s mentioned in footnote~\ref{explicit_implicit_translation_sanskrit}, discuss or dismiss Islamicate ideas but they are not translations of any particular text. 

The only other Sanskrit `translation' that could predate \Siddhantasindhu\ (early 1630s) is the anonymously authored \Hayatagrantha\ written sometime before 1694 (\vid\ Table~\ref{sanskrit_translations_of_islamic_texts}). \textcite[327]{PingreeIslamAstSkt} correctly identifies the \Hayatagrantha\ as a translation of \aliQushji's Persian \Risaladarilmalhaya\ (dedicated to the Ottoman \MehmetII, r.\@ 1451--1481). \aliQushji's work was known to the Mughal court of \Humayun\ (r.\thinspace 1530--56) via a commentary on it, the \SharhilmalHaya\ written by \MuslihalDinMuhammadLari\ (d.\thinspace 1572) and dedicated to his patron \Humayun\ \parencite[296]{Pourjavady}. 

Following the \SharhilmalHaya, the \Hayatagrantha\ also discusses an assortment of topics (\prakarana) on planetary motion (\textit{graha-gati-nirupaṇa}) and the Earth-sphere (\textit{bhūgola-varṇana}) in two chapters (\adhyaya). In each topic, it explains Islamicate astronomical terms and provides equivalent Sanskrit expressions for them. \textcite[p.\thinspace 3 of the preface]{BhattacaryaHayata} thinks the \Hayatagrantha\ is either a translation (\textit{anuvāda}) of the Arabic text (\textit{arabī-deśasya grantha}) or a paraphrase of its summary (\textit{sāraṃ gṛhitvā likhita}); \textcite{PingreeIndianReception} adds to that by suggesting the author was `helped by a collaborator who was versed in Persian and Islamic astronomy, at least at a level sufficient for understanding the \textit{Risālah}' (p.\thinspace 475).\footnote{\,According to \textcite[p.\thinspace 57ab in Volume A4]{PingreeCESS}, there are seven extant manuscripts of the \Hayatagrantha, the earliest of which was copied in Oudh (Uttar Pradesh) in 1694. \citeauthor{BhattacaryaHayata}'s editions is based on three manuscripts held at the Sarasvatī Bhavana Granthālaya in Varanasi. In his preface (p.\thinspace 12), he cites internal evidence from the manuscripts to suggest (erroneously) that the \Hayatagrantha\ was composed in Kāśī in the eighteenth century \parencite[\vid][326--327]{PingreeIslamAstSkt}.} 

Beyond these speculations on the nature of the \Hayatagrantha, its influence on \Nityananda's \Siddhantasindhu\ is what concerns us here, and this is a difficult thing to ascertain at the present time. At the very outset, the anonymous authorship and vague timeline of the \Hayatagrantha\ make it problematic to situate it in relation to the \Siddhantasindhu. The Sanskrit words for Islamicate astronomical terms in both these texts are quite similar; however, their presentations often differ. 
The \Hayatagrantha\ includes transliterations of Arabic/Persian technical words along with their equivalent expressions in Sanskrit, whereas the \Siddhantasindhu\ translates Islamicate terms into Sanskrit without (always) transliterating the original Arabic/Persian word.\footnote{\,It is worth noting that \Nityananda's \Siddhantasindhu\ is not sanitised of all Arabic or Persian words. There are several instances where Islamicate names of authors, works, calendrical elements, \etcp\ are transliterated in \Nagari\ (\eg the excerpts in \S~\ref{siddhantasindhu_nityananda}). The text from the second part (\dvitiya\ \kanda) indicates that \Nityananda\ translates Arabic/Persian astronomical terms into equivalent (or original) Sanskrit expressions without referring to the original words.} 

For example, the \Hayatagrantha\ glosses the Persian word \tsans{baaadkokiba} (\textit{bāad-kokiba}) and the Sanskrit word \tsans{spa.s.ta-kraanti} (\textit{spaṣṭa-krānti}) as the `true declination' of a celestial object \parencite[p.\thinspace 19, lines 2--4]{BhattacaryaHayata}. \Nityananda\ also 
refer to the `true declination' (of celestial objects) with the Sanskrit word \tsans{spa.s.ta-kraanti} (\textit{spaṣṭa-krānti}); however, he does not transliterate the corresponding Persian technical expression used by \MullaFarid, namely \tfarsi{بعد كواكب از معدّل النهار} (\bud\idafaconsonant\ \kawakib\ \az\ \muaddil\ \alnahar) the `distance of a celestial object from the celestial equator'. (\Vid\ the chapter-titles of Discourse \hyperlink{Pii6}{II.6} and Part \hyperlink{Sii6}{II.6} in \S~\ref{chapter_title_comparision_persian_sanskrit}, page~\pageref{discourse_part_6_chapter_title_example}).


Nevertheless, the use of the same Sanskrit words to translate Islamicate astronomy does not establish an interdependence between these texts per se. It could also indicate a common written source (\eg a bilingual technical lexicon) or point towards a more institutionalised setup (like at the scriptoriums or \maktabkhana s) where these translations were produced. The emergence of a common technical vocabulary is then another aspect of cross-traditional discourses that occurred in the seventeenth century, making this a topic of exploration for future studies.   

%%%%%%%%%%%%%%%%%%%%%%%%%%%%%%%%%%%%%%%%%%%%%%%%%%%%%%%%%%%%%%%%%%%%%%%%%%%%%%%%%%%%%%%%%
\subsection{The \Siddhantasindhu, Part II: content and context}
\label{comparative_overview_chapters_zij_sindhu}
%%%%%%%%%%%%%%%%%%%%%%%%%%%%%%%%%%%%%%%%%%%%%%%%%%%%%%%%%%%%%%%%%%%%%%%%%%%%%%%%%%%%%%%%%

The second part (\dvitiya\ \kanda) of \Nityananda's \Siddhantasindhu\ includes twenty-two chapters on various topics that help determine the time of rising (\textit{udaya-samaya}) and degrees of ascension (\textit{udaya-aṃśa}) of the zodiacal signs at one's local latitude. The arrangement of the chapters in the \Siddhantasindhu\ follows the order of the twenty-two chapters in the second discourse (\maqala\idafaconsonant\ \duvum) of the \MullaFarid's \ZijiShahJahani\ identically. Table~\ref{table_contents_part_two} provides a chapter-wise list of topics covered in the \ZijiShahJahani\ Discourse II and \Siddhantasindhu\ Part II. 

The Persian and Sanskrit chapter-titles from these two texts, along with my English translations of the titular text, are presented in parallel in \S~\ref{chapter_title_comparision_persian_sanskrit}.  A full list of Persian and Sanskrit technical expressions in the respective chapter-titles of \ZijiShahJahani\ Discourse~II and \Siddhantasindhu\ Part~II is included in the glossary (beginning on page~\pageref{main}) grouped under their common English translations. I offer below a few general remarks on the language and scope of these chapters.

%%%%%%%%%%%%%%%
\begin{table}[!htbp]
\centering
\renewcommand{\arraystretch}{1.25}
\renewcommand{\baselinestretch}{1.25}\selectfont
\begin{tabularx}{\linewidth}{cX}
    \hline
   Chapter & List of topics\\
    \hline   
    \hyperlink{Pii1}{II.1} & Sexagesimal place-values of digits\\
    \hyperlink{Pii2}{II.2} & Method of interpolation between successive entries in a table\\
    \hyperlink{Pii3}{II.3} & Calculating Sine and Versed Sine values\\
    \hyperlink{Pii4}{II.4} & Calculating the shadow of a gnomon (\ie Cotangent values)\\
    \hyperlink{Pii5}{II.5} & Declination of the zodiacal signs\\
    \hyperlink{Pii6}{II.6} & Calculating the true declination of a celestial object\\
    \hyperlink{Pii7}{II.7} & Calculating the maximum elevation and depression of a celestial object \\
    \hyperlink{Pii8}{II.8} & Right ascensions of the zodiacal signs at the terrestrial equator\\
    \hyperlink{Pii9}{II.9} & Calculating the equation and hours of daylights at a terrestrial location\\
    \hyperlink{Pii10}{II.10} & Calculating the oblique ascensions of the zodiacal signs\\
    \hyperlink{Pii11}{II.11} & Inverse calculation of the right ascensions of the zodiacal signs from their oblique ascensions\\
    \hyperlink{Pii12}{II.12} & Calculating the right ascension and ecliptic longitude of a zodiacal sign culminating at the time of rising of a celestial object\\
    \hyperlink{Pii13}{II.13} & Calculating the right ascension of celestial objects at the time of their rising and setting (at the local horizon)\\
    \hyperlink{Pii14}{II.14} & Calculating the azimuth from the altitude of a celestial object\\
    \hyperlink{Pii15}{II.15}& Calculating the altitude from the azimuth of a celestial object\\
    \hyperlink{Pii16}{II.16} & Determining the line of the local meridian\\
    \hyperlink{Pii17}{II.17} & Determining the latitude and longitude of a terrestrial location\\
    \hyperlink{Pii18}{II.18} & Calculating the zenith-distance of the nonagesimal point\\
    \hyperlink{Pii19}{II.19} & Calculating the distance (in degrees) between two celestial objects \\
    \hyperlink{Pii20}{II.20} & Determining the direction of Mecca/\Kashi\ from a terrestrial location\\
    \hyperlink{Pii21}{II.21} & Determining the ascendant zodiacal sign (at the local horizon) corresponding to the altitude of a celestial object \\
    \hyperlink{Pii22}{II.22} & Determining the altitude of a celestial object corresponding to an ascending zodiacal sign (at the local horizon)\\
\hline    
\end{tabularx}
\caption{List of topics commonly discussed in the twenty-chapters of the \ZijiShahJahani\ Discourse~II and the \Siddhantasindhu\ Part~II}
\label{table_contents_part_two}    
\end{table} 

%%%%%%%%%%%%%%%

\begin{enumerate}[topsep=0pt]
    \item \MullaFarid's Persian text describes the computations without necessarily defining the terms first. In contrast, \Nityananda's discussions often begin with definitions (\textit{lakṣaṇa}s) of technical terms before describing the computational methods.  
    The \ZijiShahJahani\ is written in the format of a traditional \zij, and accordingly, it assumes its readers are familiar with technical expressions in Arabic/Persian. \Nityananda's \Siddhantasindhu, however,  describes Islamicate astronomy in Sanskrit to readers largely unfamiliar with the form or the language of the text. The prefatory definitions, along with the use of typical Sanskrit deictic words (like \textit{atha} `now', \textit{tat} `its/their', \textit{tatra} `there', \etcp) to introduce them, suggest an emphasis on communicating ideas effectively rather than simply translating the Persian text.  
    %%
    \item The ability to translate Arabic/Persian technical terms into Sanskrit requires a conceptual understanding of both Islamicate and Sanskrit astronomy, as well as a linguistic competence in navigating between these languages. \Nityananda's translations, presumably mediated through vernacular Hindi, reflect his command on the language of Sanskrit astronomy.
    
    In some instances, his expressions are literal translations (\textit{śabda-anuvāda}) of Arabic/Persian words; for example, \gls{ascendant} \tfarsi{طالع} (\tali) as \tsans{lagna} (\textit{lagna}), 
    \gls{genus} [of digits] \tfarsi{جنس} (\jins) as \tsans{jaatii} (\textit{jātī}),  or \gls{line_midday} \tfarsi{خطّ نصف النهار} (\khatt\ \nisf\ \alnahar) as \tsans{madhyaahna-rekhaa} (\textit{madhyāhna-rekhā}). In other instances, they appear to be figurative translations (\textit{bhāva-anuvāda}) based on an implied equivalence of technical meaning; for example,
    \begin{itemize}
    \item \gls{distance_celestial_object} \tfarsi{بعد كواكب} (\bud\idafaconsonant\ \kawakib) as \gls{true_declination} \tsans{spa.s.ta-kraanti} (\textit{spaṣṭa-krānti}),
    \item \gls{equation_of_daylight} \tfarsi{تعديل النهار} (\tadil\ \alnahar) as \gls{ascensional_difference} \tsans{cara} (\textit{cara}), or 
    \item \gls{latitude_visible_climate} \tfarsi{عرض اقلیم رؤیت} (\ard\idafaconsonant\ \iqlim\idafaconsonant\ \ruyat) as \gls{ecliptic_zenith_distance} \tsans{d.rkk.sepa} (\textit{dṛkkṣepa}). 

    \end{itemize}
    

    %%
    \item A salient aspect of \Nityananda's translation is \textit{localisation}; in other words, adapting the foreign content to suit the local context. The passages in his Sanskrit translation (of the second part) explain technical terms in greater detail, while his Sanskrit vocabulary preserves the meaning of Arabic/Persian words without referring to them expressly. Beyond these  communicative and semantic measures, \Nityananda\ also changes the context in which these foreign computational methods are applied.
    
    For example, the twentieth chapter in the second discourse of the \ZijiShahJahani\ describes the method to determine the azimuth (and inclination) of \qibla; in essence, the direction of Mecca. \Nityananda\ translates this chapter as the method to determine the direction of \Kashi. Having first discussed the mathematics of finding the direction of \Kashi\ from one's own location, he then goes on to apply the method to find the directions of other cities like Agra (\textit{argalapura}) and Mecca (\textit{makkaapura}) as illustrative examples (\textit{udāharaṇa}). His translation not only captures the mathematical essence of the chapter from the \ZijiShahJahani\ but also translates the cultural context of locating sacred and imperial cities. 
    %%
    \item \label{triprasna_remark} In the colophon, \Nityananda\ states that the second part contains discussions `accompanied by many statements and rationales on the ``three questions"' (\textit{tripraśna-pracura-ukti-yukti-sahita}). The \triprasnadhikara\ is a separate chapter (\adhyaya\ or \adhikara) in Sanskrit \siddhanta s that discuss methods to find the cardinal directions (\textit{diś}), the local latitude (\textit{deśa}), and the times (\textit{kāla}) of various celestial and terrestrial phenomena. By referring to the \triprasna, \Nityananda\ again bring a familiar context to situate an otherwise curious collection of chapters. In fact, the practice of writing benediction verses (\mangalacarana) at the beginning of every part of the book and a closing colophon at the end of each part is a Sanskrit \siddhantic\ trait not seen in the \ZijiShahJahani. 
\end{enumerate}

%%%%%%%%%%%%%%%%%%%%%%%%%%%%%%%%%%%%%%%%%%%%%%%%%%%%%%%%%%%%%%%%%%%%%%%%%%%%%%%%%%%%%%%%%
\subsection{The \Siddhantasindhu, Part II.6: structure and language}
\label{chapter_six_zij_sindhu_comparision}
%%%%%%%%%%%%%%%%%%%%%%%%%%%%%%%%%%%%%%%%%%%%%%%%%%%%%%%%%%%%%%%%%%%%%%%%%%%%%%%%%%%%%%%%%

The sixth chapter (\textit{saṣṭhādhyāya}) from the second part of \Nityananda's \Siddhantasindhu\
describes three methods to compute the true declination of a celestial object. A celestial object is variously understood as a planet, a star, or an asterism that moves in the celestial sphere. Typically, this object posses a non-zero ecliptic latitude and hence its declination is different from the Sun that moves on the ecliptic. \Nityananda\ methods to compute the true declination of such an object are identical to those stated by \MullaFarid\ in the sixth chapter of the second discourse of his \ZijiShahJahani. An edition of the original text from \MullaFarid's \ZijiShahJahani\ Discourse~II.6 and \Nityananda's \Siddhantasindhu\ Part II.6, along with my English translations of corresponding Persian and Sanskrit passages, are included in \S\S~\ref{zijshahjahan_persian_english} and \ref{siddhantasindhu_sanskrit_english} separately. The technical glossary (beginning on page~\pageref{main}) also includes a list of technical expressions found in the Persian and Sanskrit passages of this chapter.

These methods rely on astronomical quantities that are distinctly Islamicate in their origin, \eg the \glslink{second_declination_parent}{second declination} of a celestial object or the \glslink{arc_max_declination}{arc of maximum argument of the distance}. However, \Nityananda's exposition of these Islamicate methods is uniquely original in its presentation and style. In the following paragraphs, I present a few remarks on the structure and the language of \Nityananda's Sanskrit text in comparison with \MullaFarid's Persian. The mathematical aspects of these methods is to appear in \textcite[forthcoming]{MisraTD}.


% The translated English expressions are emboldened for emphasis in the following paragraphs.     

\subsubsection{Structure of the text} \label{structure_content_zij_sindhu_chapter_six}
The methods to compute the \gls{distance_celestial_object} (\bud\idafaconsonant\ \kawkab\ \az\ \muaddil\ \alnahar) in the \ZijiShahJahani\ Discourse~II.6 are translated in the \Siddhantasindhu, Part II.6 as methods to compute the \gls{true_declination} (\spasta-\kranti). In \S\S~\ref{zijshahjahan_persian_english} and \ref{siddhantasindhu_sanskrit_english}, I have grouped the Persian and Sanskrit text of this chapter into comparable passages (numbered `[1]', `[2]', \etcp) to highlight their grammatical and mathematical likeness. (\Vid\ my editorial conventions in \S~\ref{chapter_vi_zij_sindhu}.) Table~\ref{comparision_passage_zij_sindhu} provides an outline of the passages in the two texts with a brief description of their content. 

As Table~\ref{comparision_passage_zij_sindhu} shows, \Nityananda's third method includes four additional passages [\hyperlink{SpassA}{α}--\hyperlink{SpassD}{δ}] in metrical verses that are not found in \MullaFarid's Persian text. The mathematics of the third method requires a knowledge of several astronomical concepts that are not very commonly known in Sanskrit astronomy. \Nityananda\ defines these Sanskrit terms in the four passages before using them in his third method of computation. \Nityananda's \Siddhantasindhu\ is among the earliest texts to Sanskritise Islamicate computational methods in astronomy; and in a few instances, coin original Sanskrit expressions for Islamicate technical terms.

\begin{table}[!htbp]
\centering
\renewcommand{\arraystretch}{1.2}
\renewcommand{\baselinestretch}{1.2}\selectfont
\begin{tabularx}{\textwidth}{ccX}
    \hline
   \multicolumn{2}{c}{Passages in II.6} & \multirow{2}{*}{Description}\\
    \cline{1-2}   
\ZijiShahJahani & \Siddhantasindhu &\\
\hline
\multicolumn{3}{c}{First Method}\\
{[\hyperlink{Ppass1}{1}]}$\,_\text{prose}$ & [\hyperlink{Spass1}{1}]$\,_\text{verse}$ & Definition of \glslink{argument_of_distance}{argument of the distance} \\
{[\hyperlink{Ppass2}{2}]}$\,_\text{prose}$ & [\hyperlink{Spass2}{2}]$\,_\text{verse}$ & First method to compute the \glslink{sine_true_declination}{Sine of the true declination}\\
\multicolumn{3}{c}{Second Method}\\
{[\hyperlink{Ppass3}{3}]}$\,_\text{prose}$ & [\hyperlink{Spass3}{3}]$\,_\text{verse}$ & Second method to compute the \glslink{sine_true_declination}{Sine of the true declination}\\
{[\hyperlink{Ppass4}{4}]}$\,_\text{prose}$ & [\hyperlink{Spass4}{4}]$\,_\text{prose}$ & Alternative second method to compute the \glslink{sine_true_declination}{Sine of the true declination} (using tables)\\
{[\hyperlink{Ppass5}{5}]}$\,_\text{prose}$ & [\hyperlink{Spass5}{5}]$\,_\text{prose}$ & Case one: no [ecliptic] \gls{latitude}\\
{[\hyperlink{Ppass6}{6}]}$\,_\text{prose}$ & [\hyperlink{Spass6}{6}]$\,_\text{prose}$ & Case two: [ecliptic] \gls{latitude} with no [first] \glslink{declination_parent}{declination}\\
{[\hyperlink{Ppass7}{7}]}$\,_\text{prose}$ & [\hyperlink{Spass7}{7}]$\,_\text{prose}$ & Case three: [first] \glslink{declination_parent}{declination} equals the \glslink{maximum_declination_parent}{obliquity of the ecliptic} \\
\multicolumn{3}{c}{Third Method}\\
-- & [\hyperlink{SpassA}{α}]$\,_\text{verse}$ & Definitions of the \glslink{solstitial_colure}{solstitial colure} and the \gls{circle_congruent_ecliptic}\\
-- & [\hyperlink{SpassB}{β}]$\,_\text{verse}$ & Definition of the arc of \gls{maximum_true_declination}\\
-- & [\hyperlink{SpassC}{γ}]$\,_\text{verse}$ & Definition of the arc of \gls{maximum_latitude}\\
-- & [\hyperlink{SpassD}{δ}]$\,_\text{verse}$ & Definitions of the \gls{congruent_bhuja} and the \gls{congruent_koti}\\
{[\hyperlink{Ppass8}{8}]}$\,_\text{prose}$ & [\hyperlink{Spass8}{8}]$\,_\text{verse}$ & Calculating the \glslink{Sine_distance_solstitial_colure_parent}{Sine of the distance along the `circle congruent to the ecliptic' from the solstice}\\
{[\hyperlink{Ppass9}{9}]}$\,_\text{prose}$ & [\hyperlink{Spass9}{9}]$\,_\text{verse}$ & Calculating the \glslink{arc_max_latitude_parent}{arc of maximum latitude}\\
{[\hyperlink{Ppass10}{10}]}$\,_\text{prose}$ & [\hyperlink{Spass10}{10}]$\,_\text{verse}$ & Calculating the \glslink{arc_max_declination}{arc of maximum argument of the distance}\\
{[\hyperlink{Ppass11}{11}]}$\,_\text{prose}$ & [\hyperlink{Spass11}{11}]$\,_\text{verse}$ & Third method to compute the \glslink{sine_true_declination}{Sine of the true declination}\\
\hline
\end{tabularx}
\caption{Description of the passages in \ZijiShahJahani\ Discourse~II.6 (in \S~\ref{zijshahjahan_persian_english}) and \Siddhantasindhu\ Part II.6 (in \S~\ref{siddhantasindhu_sanskrit_english}).}
\label{comparision_passage_zij_sindhu}
\end{table}


\paragraph{Mixture of prose and poetry}\label{prose_poetry_siddhantasindhu_chapter_six}
\Nityananda's Sanskrit translation of the sixth chapter is a mixture of prose sentences and metrical verses. As Table~\ref{comparision_passage_zij_sindhu} indicates, \MullaFarid's Persian passages are written entirely in prose while the Sanskrit text includes eleven passages in meter [\hyperlink{Spass1}{1}--\hyperlink{Spass3}{3}, \hyperlink{SpassA}{α}--\hyperlink{SpassD}{δ}, and \hyperlink{Spass8}{8}--\hyperlink{Spass11}{11}] and four passages in prose [\hyperlink{Spass4}{4}--\hyperlink{Spass7}{7}]. The metrical verses are numbered while the prose passages are are not.\footnote{\,The numbering scheme restarts at one for every metrical part following a prose interlude, \vid\ Remark in Table~\ref{mss_description_siddhantasindhu}.} These poetic verses are composed in a range Sanskrit meters:
\begin{itemize}[topsep=0pt]
    \item Passages [\hyperlink{Spass2}{2}, \hyperlink{Spass8}{8}]: eight-syllabled \textit{pramāṇikā}
    \item Passage [\hyperlink{Spass3}{3}]: eight-syllabled \textit{anuṣṭubh}
    \item Passage [\hyperlink{Spass9}{9}, \hyperlink{Spass10}{10}]: eleven-syllabled \textit{rathoddhatā}
    \item Passages [\hyperlink{Spass1}{1}, \hyperlink{Spass11}{11}]: twelve-syllabled \textit{vaṃśasthavila}
    \item Passages [\hyperlink{SpassB}{β}, \hyperlink{SpassC}{γ}]: twelve-syllabled \textit{drutavilambita}
    \item Passage [\hyperlink{SpassA}{α}]: seventeen-syllabled \textit{pṛthvī}
    \item Passage [\hyperlink{SpassD}{δ}]: \textit{āryā} \textit{jāti}-meter
\end{itemize}

The \Siddhantasindhu\ is the first \textit{explicit} translation of an Islamicate astronomical text to use metrical Sanskrit (\vid\ remark on page~\pageref{prose_poetry_mixed_form}). Its intention to Sanskritise the content for local \jyotisa s may explain the use of metrical verses; however, it is not entirely clear why \Nityananda\ chooses to then translate certain Persian passages in prose. I list below my observations on these prose passages. 
\begin{enumerate}[topsep=0pt]
    \item All three methods described in the text are prescriptive: they outline the constituent terms and then suggest a computational formula using these terms. There are no mathematical derivation or explanations given. The prose passages include an interpolative method of computation (in passage [\hyperlink{Spass4}{4}]) and three special cases (in passages  [\hyperlink{Spass5}{5}--\hyperlink{Spass7}{7}]), both based on the formula of the second method (in passage [\hyperlink{Spass3}{3}]). They appear to illustrate the (use of the) formula rather than being ancillary to it.
    \item The interpolative method in passage [\hyperlink{Spass4}{4}], and again in passage [\hyperlink{Spass6}{6}], refers to the \gls{table_Cosine_maximum_declination}. MS~Kh of the \Siddhantasindhu\ includes the text of these passages but not the table itself. In comparison, MSS~\SjA\ (f.\thinspace 21b) and \SjB\ (f.\thinspace 16a) of the \ZijiShahJahani\ present the table alongside the Persian text of the chapter.
    \item The Sanskrit verses from the \Siddhantasindhu, Part II.6 are also found in the \Sarvasiddhantaraja, \ganitadhyaya\ `chapter on computations'. \Nityananda\ copies these metrical verses verbatim into the \spastakrantyadhikara\ `section on true declination' of his \Sarvasiddhantaraja\ but leaves out the four prose passages from the \Siddhantasindhu. A critical edition and technical translation of the \spastakrantyadhikara\ is to appear in \textcite[forthcoming]{MisraTD}. 
\end{enumerate}
 
\subsubsection{Language of the text} \label{language_content_zij_sindhu_chapter_six}
\Nityananda's Sanskrit passages of the sixth chapter follow \MullaFarid's  Persian text in more than their mathematical content. In the following remarks, I note some of the linguistic features of \Nityananda's compositions in comparison to \MullaFarid's Persian sentences. All grammatical terms are abbreviated in these remarks, with an expanded list of abbreviations included on page~\pageref{acronym}. 
\begin{enumerate}[topsep=0pt]
    \item \textbf{Grammatical similarity}
    \begin{enumerate}[topsep=0pt]
    \item  \label{subject_fronting_passage_1} \textbf{Subject-fronting}\quad  The inflected grammar of Sanskrit allows a flexible word-ordering in most prose sentences, and even more so, in metrical verses. \Nityananda\ utilises this syntactic freedom and composes some of his verses to resemble \MullaFarid's Persian statements quite closely. For example,
    the Persian text in passage~[\hyperlink{Ppass1}{1}] is a typical conditional sentence where the subjects precede the conditional proposition, \sic\ 
    {\par\centering
    $\underbrace{\text{[Given]\enskip X \& Y}}_{\text{subject-fronted events}}$\quad:\quad $\underbrace{\text{if\enskip $\mathcal{C}$\,(X, Y)}}_{\text{if-clause}}$\quad $\longrightarrow$\quad $\underbrace{\text{then\enskip $\mathcal{A}_\text{1}$\,(X, Y),\enskip else\enskip $\mathcal{A}_\text{2}$\,(X, Y)}}_{\text{then-clause}}$\par}    
    where $\mathcal{C}$\,(X, Y) is the condition involving events `X' and `Y' in the if-clause, $\mathcal{A}_\text{1}$\,(X, Y) is the first action of the two events in the then-clause, and $\mathcal{A}_\text{2}$\,(X, Y) is the second alternative action of the two events in the then-clause. 
    
    Persian grammar specifies various kinds of conditional sentences, and accordingly, the verbs in the if-clause (protasis) are in different verbal moods, \eg if the condition is a proposition to be fulfilled, like in passage~[\hyperlink{Ppass1}{1}], the verb in the protasis is the subjunctive mood: \tfarsi{باشند} (\textit{bāshand})  \acrshort{present}-\acrshort{subjunctive} `should be'. The consequent then-clause (apodasis) can chose between indicative or subjunctive moods of verbs according to the context of the sentence. \Nityananda's Sanskrit verse in passage~[\hyperlink{Spass1}{1}] mirrors this subject-fronted conditional construction with the verb in the protasis in the optative mood: \tsans{bhavet} (\textit{bhavet}) \acrshort{optative}-\acrshort{active} `should be' (conditional possibility). The apodasis that follows states the consequent actions as statements to be understood as implicit instructions, \sic then [we take/do] $\mathcal{A}_\text{1}$\,(X, Y), otherwise [we take/do] $\mathcal{A}_\text{2}$\,(X, Y).
    
    \item \label{implied_modality}\textbf{Implied modality}\quad The realis mood of a present indicative statement in Persian also implies an irrealis future potential. For example, passage~[\hyperlink{Ppass2}{2}] includes two statements in the following form:
    {\par\centering
    \begin{tabular}{c} First (instructive)\\ statement \end{tabular}%
    \quad$\overset{\text{implied}}{\underset{\text{consequence}}{\longrightarrow}}$\quad%
    \begin{tabular}{c} Second (declarative)\\ statement. \end{tabular}\par}
    The second (declarative) statement uses the verb \tfarsi{بود} (\textit{buvad}) 
    \acrshort{present}-\acrshort{indicative} `is' to indicate `the result \textit{is} [something]'; this, in effect, also convey the meaning `the result \textit{will be} [something]'. \Nityananda's Sanskrit translation of this passage retains the form, and uses the verb \tsans{bhavet} (\textit{bhavet}) \acrshort{optative}-\acrshort{active} `will be' (future probability) to indicate a similar meaning: `the result \textit{will be} [something]'.
    \end{enumerate}

    \item \textbf{Semantic equivalents}\quad \Nityananda\ translates Islamicate astronomical terms using equivalent Sanskrit expressions, some of which, are literal translations; for example,
    \begin{itemize}
    \item  \gls{circle_four_poles} \tfarsi{دایرهٔ ماره باقطاب اربعه} (\textit{\dayiri\idafavowel\ \marri\ \biaqtab\idafaconsonant\ \arbai}) as \tsans{dhruva-catu.ska-yaata-v.rtta} (\textit{dhruva-catuṣka-yāta-vṛtta})
    \item \gls{latitude_celestial_object} \tfarsi{عرض کوکب} (\ard\idafaconsonant\ \kawkab) as \tsans{khagasya baa.na} (\textit{khagasya bāṇa}), or
    \item \gls{one_direction} \tfarsi{یک جهت}  (\yik\ \jahat) as  \tsans{eka\=/di"s} (\textit{eka\=/diś}).
        % \item \tfarsi{میل کلّی} (\mayl\idafaconsonant\ \kulli) `total/greatest declination' as \tsans{parama-kraanti} (\textit{parama-kraanti}),
    \end{itemize}
    In other instances, his translations employ novel Sanskrit expressions to capture the implied mathematical meaning of Islamicate terms; for example,
    \begin{itemize}
        \item \gls{Cosine_inverse_declination_degree_celestial_object}\linebreak \tfarsi{جیب تمام میل منکوس درجه كوكب}  (\jayb\idafaconsonant\ \tamam\idafaconsonant\ \mayl\idafaconsonant\ \mankus\idafaconsonant\ \daraji\idafavowel\ \kawkab)\footnote{\,The use of the term `inverse declination' (\almayl\ \almakus) to indicate the [first] declination of the ecliptic longitude of a celestial object increased by ninety first appears in the works of thirteenth century \Maragha\ astronomers, \eg \ZijIlkhani\ of \alTusi\ \parencite[188]{HamadanialTusi} or \Tajalazyaj\ of \alMaghribi\ \parencite[196]{Dorce}.} as \gls{day_Sine_increased_by_three_signs} \tsans{sa-bha-traya-dyujiivaa} (\textit{sa-bha-traya-dyujīvā}),
        \item \gls{distance_celestial_object_solstice} \tfarsi{بعد کوکب از
        \tfarsib{دایرهٔ ماره باقطاب اربعه}} (\textit{\bud\idafaconsonant\ \kawkab\ \az\ \guillemotleft\dayiri\idafavowel\ \marri\ \biaqtab\idafaconsonant\ \arbai\guillemotright})  as \gls{congruent_koti}  \tsans{sad.r"s-ko.ti} (\textit{sadṛś-koṭi}),
        \item \gls{first_arc} \tfarsi{قوس اوّل} (\qaws\idafaconsonant\ \avval) as \gls{maximum_latitude} \tsans{para-i.su} (\textit{para-iṣu}), or
        \item \gls{second_arc} \tfarsi{قوس دوم} (\qaws\idafaconsonant\ \duvum) as \gls{maximum_true_declination} \tsans{para-sphu.ta-apama} (\textit{para-sphuṭa-apama}).
    \end{itemize}
    
    \item \textbf{Hybrid translations}\quad Certain Sanskrit words appear to be a mix of literal and figurative translations, \eg 
    the `argument of the distance', understood as the arc of the great circle passing through the two ecliptic poles and a celestial object, and lying between a celestial object and the celestial equator, is called \tfarsi{حصّهٔ بعد} (\hissi\idafavowel\ \bud) `share of the distance' in Persian. \Nityananda\ translates this Persian expression as \tsans{sphu.ta-apama-a.m"sa} (\textit{sphuṭa-apama-aṃśa}) `share of true declination' in passage~[\hyperlink{SEpass1}{1}], but then goes on to translate it as \tsans{sphu.ta-apama-a"nka} (\textit{sphuṭa-apama-aṅka}) `curve of the true declination'\footnote{\,Quite typically, the word \tsans{a"nka} (\textit{aṅka}) refers to a `digit/ number', or more literally, a `mark/sign'. \Nityananda\ uses the word to signify a geometrical `arc' or `curve' of a great circle. This interpretation is validated by the use of the word \textit{aṅka} to mean `arc/curve' in several geometrical explanations in the \Siddhantasindhu\ as well as the \Sarvasiddhantaraja\ \parencite[\eg \vid][279]{Misrathesis}.} in subsequent passages~[\hyperlink{SEpass2}{2}--\hyperlink{SEpass4}{4}].

    \item \textbf{Original expressions}\quad  As noted earlier, \Nityananda\ composes four passages~[\hyperlink{SpassA}{α}--\hyperlink{SpassD}{δ}] in his Sanskrit text to explain the
    terms involved in the third computational method. These terms are not directly expressed in the Persian text, and hence, \Nityananda's expressions are original in their language and their substance. For example, passage~[\hyperlink{SEpassA}{α}] defines the \gls{circle_congruent_ecliptic} \tsans{bhacakra-sad.r"sa-v.rtta} (\textit{bhacakra-sadṛśa-vṛtta}), a great circle passing through the two equinoctial points and a celestial object, and resembling the ecliptic. For planetary objects, this is the orbit of the planet with the longitude of its node being zero; or as \Nityananda\ later indicates in passage~[\hyperlink{SEpassC}{γ}], when the \gls{conjunction_equinox_node} \tsans{vi.suva-paata-yuga} (\textit{viṣuva-pāta-yuga}) is being assumed.
    
    \item \textbf{Synonymy and ambiguity}\quad In rendering \MullaFarid's Persian sentences into Sanskrit, \Nityananda\ employs a variety of Sanskrit synonyms (or near-synonyms) to translate Persian technical terms. For example, a celestial object is called \tfarsi{کوکب} (\kawkab) in Persian, whereas \Nityananda\ variously uses the words  \tsans{khaga} (\textit{khaga}), \tsans{graha} (\textit{graha}),  \tsans{dyucara} (\textit{dyucara}), \tsans{nak.satra} (\textit{nakṣatra}), \tsans{nabhoga} (\textit{nabhoga}), and \tsans{bha} (\textit{bha}) to describe such an object. The abundance of Sanskrit synonyms allows him to choose words that suits the meter of his verse; however, the metrical constraints also makes his translations a little vague in some instances. For example, the Persian text in passage~[\hyperlink{PEpass1}{1}] refers to the \tfarsi{جهت حصّهٔ بعد} (\jahat\idafaconsonant\ \hissi\idafavowel\ \bud) `direction of the share of the distance' as being in the \tfarsi{جهت مجموع}  (\jahat\idafaconsonant\ \majmu) `direction of the sum' or the \tfarsi{جهت فضل} (\jahat\idafaconsonant\ \fadla) `direction of the difference/residue'; in other words, dependant on the directions of the two quantities that make up the `share of the distance'. In his 
    Sanskrit translation in passage~[\hyperlink{SEpass1}{1}], \Nityananda\ simply states that the `share of the true declination' is in its \tsans{sva-di"s} (\textit{sva-diś}) `own direction'.
    
    \item \textbf{Lowering a sexagesimal number}\quad Of particular note is an Islamicate arithmetic operation that, to my knowledge, does appear in any Sanskrit astronomical or mathematical text till the \Siddhantasindhu.  
    
    Islamicate texts often include the operation \tfarsi{منحطّ} \munhatt\ `lowering' a sexagesimal number before multiplying (\gls{low_multiplication}) or before dividing (\gls{low_division}). In effect, shifting the fractional point leftwards to \textit{lower} the value of the sexagesimal number before operating on it; or in other words, dividing a number by the Radius (\textit{sinus totus}) of 60.\footnote{\,For instance, \alKashi\ uses the word \munhatt\ `to depress' a sexagesimal number, \ie divide it by the Radius of 60, in relation to his Sine computations in his \KhaqaniZij\ (\circa 1413/1414) \parencite[40]{Hamadani_Zadeh_Khaqani}.} \Nityananda\ uses the verb √\tsans{adharii-k.r} (√\textit{adharī-kṛ}) `to make [something] low' or `lower' to indicate a division by 60.\footnote{\,MS Benares (1963) 37079 of the \Sarvasiddhantaraja\ from the Sarasvatī Bhavana Granthālaya (Varanasi) parses the word \tsans{adhara} \textit{adhara} in relation to the divisor (in the last \pada\ of verse 6) as \begin{quote}\tsans{adhara-sa-.sa.s.ti-bhakta-bhaajaka-bhajana.m atraadhara-bhajana-sa.m{\tsnb{ज्ञं}} ucyate} (f.\thinspace 63r:~11)\newline
    \textit{adhara} is the sixtieth part [\lit with sixty divided] of the divisor of the division; here it is referred to as the \textit{adhara}-division by name.
    \end{quote}}
    His translation of the term is more literal compared to later authors like \Nayanasukhopadhyaya\ who, in his \Sarahatajakiravirajandi\ (1729), translates \munhatt\ from \alBirjandi's \SharhalTadhkirah\ (1507) as \textit{ṣaṣtyāpta}/\textit{ṣaṣṭibhakta} `divided by sixty'  \parencite[265]{KusubaPingree}.
\end{enumerate}

%%%%%%%%%%%%%%%%%%%%%%%%%%%%%%%%%%%%%%%%%%%%%%%%%%%%%%%%%%%%%%%%%%%%%%%%%%%%%%%%%%%%%%%%%
\subsection{Grammatical notes on translation} \label{grammatical_notes_translation}
%%%%%%%%%%%%%%%%%%%%%%%%%%%%%%%%%%%%%%%%%%%%%%%%%%%%%%%%%%%%%%%%%%%%%%%%%%%%%%%%%%%%%%%%%
In translating the Persian text from \MullaFarid's \ZijiShahJahani, Discourse~II.6 and \Nityananda's \Siddhantasindhu, Part~II.6 into English, I have interpreted the verbs (and verbal derivatives) based on their implied modality in the sentence. Hence, a particular form of a verb (\eg an optative active in Sanskrit) is sometimes translated differently in different sentences. I list below some of the main aspects of my translations. 

\paragraph{Persian to English} 
\begin{enumerate}[topsep=0pt]
    \item The Persian subjunctive mood expresses a variety of meanings based on the context of the sentence. \MullaFarid's Persian text of the sixth chapter uses the present subjunctive form of verbs quite commonly. Translating these verbs using English indicative forms does not fully capture the subjunctive mood of the original sentences. Hence, I translate the Persian subjunctive verbs in my English translations with the modal verb `should', \eg \tfarsi{باشد} (\textit{bāshad})  \acrshort{present}-\acrshort{subjunctive}-\acrshort{singular}·\acrshort{third} `[he/she/it] should be' instead of `[he/she/it] is' or `[he/she/it] is to be'. This helps distinguish between the Persian indicatives (realis) and subjunctives (irrealis) in the passages, particularly, in the case of conditional clauses.
                
    \item An impersonal passive sentence may be constructed in Persian with a third-person plural conjugation of the verb and a dismissive (or vague) subject. For example, the Persian text in passage~[\hyperlink{Ppass4}{4}] uses the verbs \tfarsi{درآرند} (\textit{dar ārand}) \acrshort{present}-\acrshort{indicative}-\acrshort{plural}·\acrshort{third} `[they] extract', along with a direct object (identified by the \tfarsi{را} (\textit{rā}) marker) and no specific subject. The only indication of the subject is found in the enclitic conjugation of the verb. A syntax-preserving translation of this sentence reads `they extract [the object]'; however, it can also understood as the impersonal passive sentence `[the object] is extracted'. As grammatical opinions on passive constructions in Persian vary  \parencite[\eg \vid][261--264]{NematiPersianpassive}, I choose to retain the syntax-preserving form in my English translation in passage~[\hyperlink{PEpass4}{4}].
\end{enumerate}

\paragraph{Sanskrit to English} 
\begin{enumerate}[topsep=0pt]
\item The subjunctive mood is obsolete in classical Sanskrit and is replaced by the use of the  optative. Like the subjunctive, the optative mood also indicates various meanings depending on the context of the verb. \Nityananda's Sanskrit passages from the sixth chapter use several verbs in their optative active form. I translate these forms according to their syntactic location (\ie whether they occur in principal realis sentences or subordinating irrealis clauses) and their modal intention. For example, as I previously alluded in remarks~\ref{subject_fronting_passage_1} and \ref{implied_modality} in \S~\ref{language_content_zij_sindhu_chapter_six}, the verb \tsans{bhavet} (\textit{bhavet}) \acrshort{optative}-\acrshort{active}-\acrshort{singular}·\acrshort{third} implies both `[he/she/it] should be' (in conditional clauses) and `[he/she/it] will be' (in potential statements) depending on the context. 

\item More generally, I use the the English modal verb `should' to translate most Sanskrit optative forms to distinguish them from the indicative forms in English \eg \tsans{syaat} (\textit{syāt}) \acrshort{optative}-\acrshort{active}-\acrshort{singular}·\acrshort{third} `[he/she/it] should be/exist' instead of `[he/she/it] is/exists'. In such translations, the English modal verb `should' is more epistemic than deontic in conveying the irrealis mood (in other words, \textit{should} conveys the sentiment of possibility or inference and not a directive or exhortation). Also, the use `should' to mark the Persian subjunctive and the Sanskrit optative in the English translations reveal a syntactic similarity in the Persian and Sanskrit passages of the sixth chapter.
\end{enumerate}

A complete list of the Persian verbs in the \ZijiShahJahani, Discourse~II.6 and the Sanskrit verbs (and verbal derivatives) in \Siddhantasindhu, Part~II.6 can be found in Appendices~\ref{persian_verbs} and \ref{sanskrit_verbs} respectively.   
    

   
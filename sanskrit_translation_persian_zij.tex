%%%%%%%%%%%%%%%%%%%%%%%%%%%%%%%%%%%%%%%%%%%%%%%%%%%%%%%%%%%%%%%%%%%%%%%%%%%%%%%%%%%%%%%%%
\subsection{Chronology and influence}\label{chronology_influence}
%%%%%%%%%%%%%%%%%%%%%%%%%%%%%%%%%%%%%%%%%%%%%%%%%%%%%%%%%%%%%%%%%%%%%%%%%%%%%%%%%%%%%%%%%

\Nityananda's \Siddhantasindhu\ is a translation of \MullaFarid's \ZijiShahJahani. It was completed in the early 1630s, making it one of the earliest Sanskrit translations of a Persian \zij. The \JicaUlugbegi, a Sanskrit translation of \ZijUlughBeg\ commissioned during \Akbar's reign (\ante\ 1605), is perhaps the earliest example. However, the extant fragmentary manuscripts only contain tables and star catalogues (\vid\ footnote~\ref{jicaulughbeg}) and not the canon (text).\footnote{\,The \JicaUlugbegi\ was prepared by a consortium of astronomers led by \AmirFathullahofShiraz, with the assistance of Sanskrit interpreters (\vid\  \S~\ref{Mughal_court}). As \textcite[367]{Sarmajyotisaraja} has pointed out, some of these interpreters were Sanskrit \jyotisa s: \eg Kishan Joshī is identified as  \Krsnadaivajna\ (\fl \circa 1600/25). If the authors' version of \JicaUlugbegi\ contained the canon of the \ZijUlughBeg, the following two reasons make the loss of the Sanskrit canon a notable (and regrettable) event.
\begin{itemize}
    \item \MullaFarid\ copied large parts of the \ZijUlughBeg\ (almost) verbatim in his \ZijiShahJahani; these parts have been translated by \Nityananda\ in his \Siddhantasindhu. The Sanskrit canon in the \JicaUlugbegi\ may have served as \Nityananda's model text to recopy just as easily as the \ZijUlughBeg\ served \MullaFarid.
    \item \MullaFarid\ was a direct pupil of \AmirFathullahofShiraz, and is believed to have `learned the rational sciences (\UlumiAqliyah) including astronomy and astrology' from him \parencite[720]{Ansarimedievalindia}. This suggests that, at the very least, \MullaFarid\ (and by association, \Nityananda) would have been familiar with a version of the \JicaUlugbegi\ closer to the original than what is currently extant.
\end{itemize}} % 

In contrast, as \S~\ref{siddhantasindhu_nityananda} describes, the \Siddhantasindhu\ includes tables as well as three (out of the four) distinct portions (\kanda s `parts' in Sanskrit or \maqalat\ `discourse' in Persian) of the canon of \ZijiShahJahani. The full extent of similarity between the content of these two texts can only be determined by an extensive future study; for our present discussion, however, we note that the \Siddhantasindhu\ expressly declares itself to be a Sanskrit version of the Persian original. Other Sanskrit texts in mathematical astronomy, \eg the three \siddhanta s mentioned in footnote~\ref{explicit_implicit_translation_sanskrit}, discuss or dismiss Islamicate ideas but they are not translations of any particular text. 

The only other Sanskrit `translation' that could predate \Siddhantasindhu\ (early 1630s) is the anonymously authored \Hayatagrantha\ written sometime before 1694 (\vid\ Table~\ref{sanskrit_translations_of_islamic_texts}). \textcite[327]{PingreeIslamAstSkt} correctly identifies the \Hayatagrantha\ as a translation of \aliQushji's Persian \Risaladarilmalhaya\ (dedicated to the Ottoman \MehmetII, r.\@ 1451--1481). \aliQushji's work was known to the Mughal court of \Humayun\ (r.\thinspace 1530--56) via a commentary on it, the \SharhilmalHaya\ written by \MuslihalDinMuhammadLari\ (d.\thinspace 1572) and dedicated to his patron \Humayun\ \parencite[296]{Pourjavady}. 

Following the \SharhilmalHaya, the \Hayatagrantha\ also discusses an assortment of topics (\prakarana) on planetary motion (\textit{graha-gati-nirupaṇa}) and the Earth-sphere (\textit{bhūgola-varṇana}) in two chapters (\adhyaya). In each topic, it explains Islamicate astronomical terms and provides equivalent Sanskrit expressions for them. \textcite[p.\thinspace 3 of the preface]{BhattacaryaHayata} thinks the \Hayatagrantha\ is either a translation (\textit{anuvāda}) of the Arabic text (\textit{arabī-deśasya grantha}) or a paraphrase of its summary (\textit{sāraṃ gṛhitvā likhita}); \textcite{PingreeIndianReception} adds to that by suggesting the author was `helped by a collaborator who was versed in Persian and Islamic astronomy, at least at a level sufficient for understanding the \textit{Risālah}' (p.\thinspace 475).\footnote{\,According to \textcite[p.\thinspace 57ab in Volume A4]{PingreeCESS}, there are seven extant manuscripts of the \Hayatagrantha, the earliest of which was copied in Oudh (Uttar Pradesh) in 1694. \citeauthor{BhattacaryaHayata}'s editions is based on three manuscripts held at the Sarasvatī Bhavana Granthālaya in Varanasi. In his preface (p.\thinspace 12), he cites internal evidence from the manuscripts to suggest (erroneously) that the \Hayatagrantha\ was composed in Kāśī in the eighteenth century \parencite[\vid][326--327]{PingreeIslamAstSkt}.} 

Beyond these speculations on the nature of the \Hayatagrantha, its influence on \Nityananda's \Siddhantasindhu\ is what concerns us here, and this is a difficult thing to ascertain at the present time. At the very outset, the anonymous authorship and vague timeline of the \Hayatagrantha\ make it problematic to situate it in relation to the \Siddhantasindhu. The Sanskrit words for Islamicate astronomical terms in both these texts are quite similar; however, their presentations often differ. 
The \Hayatagrantha\ includes transliterations of Arabic/Persian technical words along with their equivalent expressions in Sanskrit, whereas the \Siddhantasindhu\ translates Islamicate terms into Sanskrit without (always) transliterating the original Arabic/Persian word.\footnote{\,It is worth noting that \Nityananda's \Siddhantasindhu\ is not sanitised of all Arabic or Persian words. There are several instances where Islamicate names of authors, works, calendrical elements, \etcp\ are transliterated in \Nagari\ (\eg the excerpts in \S~\ref{siddhantasindhu_nityananda}). The text from the second part (\dvitiya\ \kanda) indicates that \Nityananda\ translates Arabic/Persian astronomical terms into equivalent (or original) Sanskrit expressions without referring to the original words.} 

For example, the \Hayatagrantha\ glosses the Persian word \tsans{baaadkokiba} (\textit{bāad-kokiba}) and the Sanskrit word \tsans{spa.s.ta-kraanti} (\textit{spaṣṭa-krānti}) as the `true declination' of a celestial object \parencite[p.\thinspace 19, lines 2--4]{BhattacaryaHayata}. \Nityananda\ also 
refer to the `true declination' (of celestial objects) with the Sanskrit word \tsans{spa.s.ta-kraanti} (\textit{spaṣṭa-krānti}); however, he does not transliterate the corresponding Persian technical expression used by \MullaFarid, namely \tfarsi{بعد كواكب از معدّل النهار} (\bud\idafaconsonant\ \kawakib\ \az\ \muaddil\ \alnahar) the `distance of a celestial object from the celestial equator'. (\Vid\ the chapter-titles of Discourse \hyperlink{Pii6}{II.6} and Part \hyperlink{Sii6}{II.6} in \S~\ref{chapter_title_comparision_persian_sanskrit}, page~\pageref{discourse_part_6_chapter_title_example}).


Nevertheless, the use of the same Sanskrit words to translate Islamicate astronomy does not establish an interdependence between these texts per se. It could also indicate a common written source (\eg a bilingual technical lexicon) or point towards a more institutionalised setup (like at the scriptoriums or \maktabkhana s) where these translations were produced. The emergence of a common technical vocabulary is then another aspect of cross-traditional discourses that occurred in the seventeenth century, making this a topic of exploration for future studies.   

%%%%%%%%%%%%%%%%%%%%%%%%%%%%%%%%%%%%%%%%%%%%%%%%%%%%%%%%%%%%%%%%%%%%%%%%%%%%%%%%%%%%%%%%%
\subsection{Persian content in a Sanskrit context}
\label{comparative_overview_chapters_zij_sindhu}
%%%%%%%%%%%%%%%%%%%%%%%%%%%%%%%%%%%%%%%%%%%%%%%%%%%%%%%%%%%%%%%%%%%%%%%%%%%%%%%%%%%%%%%%%

The second part (\dvitiya\ \kanda) of \Nityananda's \Siddhantasindhu\ includes twenty-two chapters on various topics that help determine the time of rising (\textit{udaya-samaya}) and degrees of ascension (\textit{udaya-aṃśa}) of the zodiacal signs at one's local latitude. The arrangement of the chapters in the \Siddhantasindhu\ follows the order of the twenty-two chapters in the second discourse (\maqala\idafaconsonant\ \duvum) of the \MullaFarid's \ZijiShahJahani\ identically. Table~\ref{table_contents_part_two} provides a chapter-wise list of topics covered in the \ZijiShahJahani\ Discourse II and \Siddhantasindhu\ Part II. The Persian and Sanskrit chapter-titles from these two texts, along with my English translations of the titular text, are presented in parallel in \S~\ref{chapter_title_comparision_persian_sanskrit}. I offer below a few general remarks on the language and scope of these chapters.

%%%%%%%%%%%%%%%
\begin{table}[!htbp]
\centering
\renewcommand{\arraystretch}{1.25}
\renewcommand{\baselinestretch}{1.25}\selectfont
\begin{tabularx}{\linewidth}{cX}
    \hline
   Chapter & List of topics\\
    \hline   
    \hyperlink{Pii1}{II.1} & Sexagesimal place-values of digits\\
    \hyperlink{Pii2}{II.2} & Method of interpolation between successive entries in a table\\
    \hyperlink{Pii3}{II.3} & Calculating Sine and Versed Sine values\\
    \hyperlink{Pii4}{II.4} & Calculating the shadow of a gnomon (\ie Cotangent values)\\
    \hyperlink{Pii5}{II.5} & Declination of the zodiacal signs\\
    \hyperlink{Pii6}{II.6} & Calculating the true declination of a celestial object\\
    \hyperlink{Pii7}{II.7} & Calculating the maximum elevation and depression of a celestial object \\
    \hyperlink{Pii8}{II.8} & Right ascensions of the zodiacal signs at the terrestrial equator\\
    \hyperlink{Pii9}{II.9} & Calculating the equation and hours of daylights at a terrestrial location\\
    \hyperlink{Pii10}{II.10} & Calculating the oblique ascensions of the zodiacal signs\\
    \hyperlink{Pii11}{II.11} & Inverse calculation of the right ascensions of the zodiacal signs from their oblique ascensions\\
    \hyperlink{Pii12}{II.12} & Calculating the right ascension and ecliptic longitude of a zodiacal sign culminating at the time of rising of a celestial object\\
    \hyperlink{Pii13}{II.13} & Calculating the right ascension of celestial objects at the time of their rising and setting (at the local horizon)\\
    \hyperlink{Pii14}{II.14} & Calculating the azimuth from the altitude of a celestial object\\
    \hyperlink{Pii15}{II.15}& Calculating the altitude from the azimuth of a celestial object\\
    \hyperlink{Pii16}{II.16} & Determining the line of the local meridian\\
    \hyperlink{Pii17}{II.17} & Determining the latitude and longitude of a terrestrial location\\
    \hyperlink{Pii18}{II.18} & Calculating the zenith-distance of the nonagesimal point\\
    \hyperlink{Pii19}{II.19} & Calculating the distance (in degrees) between two celestial objects \\
    \hyperlink{Pii20}{II.20} & Determining the direction of Mecca/\Kashi\ from a terrestrial location\\
    \hyperlink{Pii21}{II.21} & Determining the ascendant zodiacal sign (at the local horizon) corresponding to the altitude of a celestial object \\
    \hyperlink{Pii22}{II.22} & Determining the altitude of a celestial object corresponding to an ascending zodiacal sign (at the local horizon)\\
\hline    
\end{tabularx}
\caption{List of topics commonly discussed in the twenty-chapters of the \ZijiShahJahani\ Discourse~II and the \Siddhantasindhu\ Part~II}
\label{table_contents_part_two}    
\end{table} 

%%%%%%%%%%%%%%%

\begin{enumerate}[topsep=0pt]
    \item \MullaFarid's Persian text describes the computations without necessarily defining the terms first. In contrast, \Nityananda's discussions often begin with definitions (\textit{lakṣaṇa}s) of technical terms before describing the computational methods.  
    The \ZijiShahJahani\ is written in the format of a traditional \zij, and accordingly, it assumes its readers are familiar with technical expressions in Arabic/Persian. \Nityananda's \Siddhantasindhu, however,  describes Islamicate astronomy in Sanskrit to readers largely unfamiliar with the form or the language of the text. The prefatory definitions, along with the use of typical Sanskrit deictic words (like \textit{atha} `now', \textit{tat} `its/their', \textit{tatra} `there', \etcp) to introduce them, suggest an emphasis on communicating ideas effectively rather than simply translating the Persian text.  
    %%
    \item The ability to translate Arabic/Persian technical terms into Sanskrit requires a conceptual understanding of both Islamicate and Sanskrit astronomy, as well as a linguistic competence in navigating between these languages. \Nityananda's translations, presumably mediated through vernacular Hindi, reflect his command over the language of Sanskrit astronomy. His expressions are literal translations (\textit{śabda-anuvāda}) of Arabic/Persian words in a few places; for example, 
    {\par
    \tfarsi{جنس} (\jins) `genus' [of digits] as \tsans{jaatii} (\textit{jātī}), \tfarsi{خطّ نصف النهار} (\khatt\ \nisf\ \alnahar) `line of midday' as \tsans{madhyaahna-rekhaa} (\textit{madhyāhna-rekhā}), or  \tfarsi{طالع} (\tali) `ascendant' as \tsans{lagna} (\textit{lagna}).\par}
    Whereas in other instances, they appear to be figurative translations (\textit{bhāva-anuvāda}) based on an implied equivalence of technical meaning; for example,
    
    {\renewcommand{\arraystretch}{1.2}
    \begin{tabularx}{\linewidth}{lXX}
    --&\tfarsi{بعد كواكب} (\bud\idafaconsonant\ \kawakib) \newline`distance of a celestial object from the celestial equator' &\tsans{spa.s.ta-kraanti} (\textit{spaṣṭa-krānti})\newline `true/correct declination'\\
    --&\tfarsi{عرض اقلیم رؤیت} (\ard\idafaconsonant\ \iqlim\idafaconsonant\ \ruyat) `latitude of the visible climate' &  \tsans{d.rkk.sepa} (\textit{dṛkkṣepa})   `zenith distance of the nonagesimal point'\\
    --&\tfarsi{تعديل النهار} (\tadil\ \alnahar) \newline`equation of daylight' &
    \tsans{cara} (\textit{cara}) \newline `ascensional difference'
    \end{tabularx}}
    %%
    \item A salient aspect of \Nityananda's translation is \textit{localisation}; in other words, adapting the foreign content to suit the local context. The passages in his Sanskrit translation (of the second part) explain technical terms in greater detail, while his Sanskrit vocabulary preserves the meaning of Arabic/Persian words without referring to them expressly. Beyond these  communicative and semantic measures, \Nityananda\ also changes the context in which these foreign computational methods are applied.
    
    For example, the twentieth chapter in the second discourse of the \ZijiShahJahani\ describes the method to determine the azimuth (and inclination) of \qibla; in essence, the direction of Mecca. \Nityananda\ translates this chapter as the method to determine the direction of \Kashi. Having first discussed the mathematics of finding the direction of \Kashi\ from one's own location, he then goes on to apply the method to find the directions of other cities like Agra (\textit{argalapura}) and Mecca (\textit{makkaapura}) as illustrative examples (\textit{udāharaṇa}). His translation not only captures the mathematical essence of the chapter from the \ZijiShahJahani\ but also translates the cultural context of locating sacred and imperial cities. 
    %%
    \item \label{triprasna_remark} In the colophon, \Nityananda\ states that the second part contains discussions `accompanied by many statements and rationales on the ``three questions"' (\textit{tripraśna-pracura-ukti-yukti-sahita}). The \triprasnadhikara\ is a separate chapter (\adhyaya\ or \adhikara) in Sanskrit \siddhanta s that discuss methods to find the cardinal directions (\textit{diś}), the local latitude (\textit{deśa}), and the times (\textit{kāla}) of various celestial and terrestrial phenomena. By referring to the \triprasna, \Nityananda\ again bring a familiar context to situate an otherwise curious collection of chapters. In fact, the practice of writing benediction verses (\mangalacarana) at the beginning of every part of the book and a closing colophon at the end of each part is a Sanskrit \siddhantic\ trait not seen in the \ZijiShahJahani. 
\end{enumerate}

%%%%%%%%%%%%%%%%%%%%%%%%%%%%%%%%%%%%%%%%%%%%%%%%%%%%%%%%%%%%%%%%%%%%%%%%%%%%%%%%%%%%%%%%%
\subsection{The \Siddhantasindhu, Part II.6}
\label{chapter_six_zij_sindhu_comparision}
%%%%%%%%%%%%%%%%%%%%%%%%%%%%%%%%%%%%%%%%%%%%%%%%%%%%%%%%%%%%%%%%%%%%%%%%%%%%%%%%%%%%%%%%%



The common Sanskrit words for translation are 
interpretation (\textit{anuvāda}, \lit explanatory oral repetition); translation (\textit{bhāṣāntara} \lit different language); rewriting (\textit{purnarlekhana}); explanation (\textit{vivaraṇa})


The performative aspect of a rendition is seen in \Nityananda's metrical compositions, which along with his prosaic interludes, allow him to translate and localise \MullaFarid's Persian prose passages in Sanskrit.   

The \Siddhantasindhu\ is a hybrid mix of prose and poetry. 

It is interesting to note that \Nityananda\ repeats parts of his his \Siddhantasindhu\ in his \Sarvasiddhantaraja\ verbatim. The last section on `true declinations' (\spastakrantyadhikara) in the \Sarvasiddhantaraja\ (\ganitadhyaya, 'chapter on computations') includes the metrical verses from Part II.6 of the \Siddhantasindhu\  but leaves out the non-metrical prose-sentences. I discuss the mathematics of this choice in \textcite[forthcoming]{MisraTD}; instead, I explore here the milieu in which these technical translations were effected.   
 
 
Sanskrit translations of Islamicate astronomical text were almost always rendered in prose (with a few inceptive benedictory verses (\textit{maṅgalācaraṇa}) in meter).

\Nityananda's \Sarvasiddhantaraja\ (1639) is a prime example of the latter kind, and following the style of a Sanskrit \siddhanta, it is composed entirely in metrical verses.

\subsubsection{}\label{structure_chapter_six}

\gls{other_declination} 
adharikṛtā \label{adharikrta_discussion}
low multiplication
multiplication of sexagesimal numbers, and the division of the result (of the multiplication) by 60 (Radius); in other words, shifting the fractional point leftwards to \textit{lower} the value of the sexagesimal number

lowering
\textit{lowering} the value of the sexagesimal number, \ie dividing it by 60 (Radius); in effect, shifting the fractional point leftwards.\\[5pt]

low diving
division of sexagesimal numbers, with the divisor first divided by 60 (Radius); in other words, \textit{lowering} the sexagesimal value of the divisor by first shifting its fractional point leftwards.\\[5pt]


cosine of max declination table
 table of the product of the $\mathcal{R}$\thinspace cosine of the \protect\gls{maximum_declination} [\ie $\mathcal{R}$\thinspace cosine of the \textbf{obliquity of the ecliptic}] with sixty sexagesimal numbers from 1 to 60.

daysine of longitude increased by three signs \\
\tsans{sa-bha-traya-dyujiivaa} (\textit{sa-bha-traya-dyujīvā}), \lit `the day-radius corresponding to an increase of three zodiacal signs'\\
the radius ($\mathcal{R}$\thinspace cosine) of the parallel of ecliptic declination (δ\textsubscript{1}) of a celestial object corresponding to its ecliptic longitude (λ\degree) increased by three zodiacal signs (90\degree), \ie $\mathcal{R}$\thinspace cos$\Big[$δ\textsubscript{1}(λ\degree + 90\degree)$\Big]$.\\[5pt]
        
\tfarsi{میل کلّی} (\mayl\idafaconsonant\ \kulli), \lit `total declination'; \tsans{parama-kraanti} (\textit{parama-kraanti}), \lit `greatest declination'        
The \ZijiShahJahani, following the \ZijUlughBeg, and the \Siddhantasindhu\ use an ecliptic obliquity of 23;\thinspace 30,\thinspace17 degrees.

first declinaton
the \textit{first} declination (δ\textsubscript{1}) of a point on the ecliptic with ecliptic longitude (λ\degree), measured in degrees, corresponding to the position of a celestial object; in other words, the arc of the great circle passing through the two celestial poles and a point on the ecliptic (with longitude λ\degree), and lying between the ecliptic and the celestial equator.
        
inverse declination\label{inverse_declination_discussion}
the first \protect\glslink{declination_degree}{declination of the degrees} (δ\textsubscript{1}) of a point on the ecliptic with longitude equal to the ecliptic longitude (λ\degree) of a celestial object increased by 90\degree, \ie δ\textsubscript{1}(λ\degree~+~90\degree).\footnote{\,The use of the term `inverse declination' (\almayl\ \almakus) to mean δ\textsubscript{1}(λ\degree\ + 90\degree) first appears in the works of thirteenth century \ce\ Mar\={a}gha astronomers: for example, \ZijIlkhani\ of \alTusi, \vid\ \textcite[188]{HamadanialTusi}, and \Tajalazyaj\ of \alMaghribi, \vid\ \textcite[196]{Dorce}.}

bhuja defintion
\tsans{khagasya bhujaa} (\textit{khagasya bhujā})\\[5pt]
        the arc of ecliptic longitude of a celestial object, taken as a \textit{bhujā} measure, \vid\ \textsc{remark} in \protect\gls{congruent_bhuja}.

owndirection svadiś
\textit{Tacitly}, the \protect\gls{direction_sum} or the \protect\gls{direction_residue}, according to orientation of a celestial object. \Vid\ \S\thinspace\ref{siddhantasindhu_sanskrit_english}:~[1].

distance from the ecliptic points 
\tfarsi{بعد درجه کوکب از اعتدال اقرب} (\textit{\bud\idafaconsonant\ \daraji\idafavowel\ \kawkab\ \az\ \itidal\idafaconsonant\ \aqrab})\\[5pt]
        the arc of the ecliptic between the celestial object and the nearest equinoctial points (`first point of Aries' 0\degree\ or `first point of Libra' 180\degree).\\[5pt]
        
table of sine
The \ZijiShahJahani\ and the \Siddhantasindhu\ tabulate the Sine values for every minute of arc from 0;\thinspace 0 degree to 360;\thinspace 0 degrees. The maximum value of the Sine (\textit{sinus totus}), identified with the Radius ($\mathcal{R}$), is 60;\thinspace 0.

suvrtta
a great circle (orthodrome) of a sphere, \ie a circle on a sphere that is concentric with the centre of the sphere and passes through two antipodal points on the sphere's surface.


circle congruent to the ecliptic
a great circle passing through the two equinoctial points and a celestial object, and resembling the ecliptic, \ie having its degrees of arc measured from the vernal equinoctial point  (\textit{meṣādi} `first point of Aries' 0\degree). \Vid\ \S\thinspace\ref{siddhantasindhu_sanskrit_english}:~{\footnotesize \P}\thinspace(α).

congruent arc
the arc of the \protect\gls{circle_congruent_ecliptic} that lies between the equinoctial point and the celestial object. 
\textsc{remark}:  More generally, \tsans{bhujaa} (\textit{bhujā}) [\lit `hand/arm', aliases: \tsans{baahu} (\textit{bāhu}) or \tsans{dos} (\textit{dos})] of an angle is a technical term in Sanskrit mathematical sciences denoting the amount of degrees \textit{already elapsed in odd quadrants} and the amount of degrees \textit{to be elapsed in even quadrants} of any circle.

complement of bhuja
complement of the \protect\gls{congruent_bhuja} to 90\degree, in other words, the arc of the \protect\gls{circle_congruent_ecliptic} lying between the celestial object and the solstital~colure. \Vid\ \S\thinspace\ref{siddhantasindhu_sanskrit_english}:~{\footnotesize \P}\thinspace(δ).\\[5pt]
       \textsc{remark}: More generally, \tsans{ko.ti} (\textit{koṭi}) [\lit `extremity'] of an angle is a technical term in Sanskrit mathematical sciences denoting the amount of degrees \textit{to be elapsed in odd quadrants} and the amount of degrees \textit{already elapsed in even quadrants} of any circle.\\[5pt]

distance of the degree of a celestial object from the nearest solstice
\tfarsi{بعد درجه کوکب از انقلاب اقرب}  (\textit{\bud\idafaconsonant\ \daraji\idafavowel\ \kawkab\ \az\ \inqilab\idafaconsonant\ \aqrab})
the arc of the ecliptic between the celestial object and the nearest solstitial points (`first point of Capricorn' 90\degree\ or `first point of Cancer' 270\degree).\\[5pt]

lowered sine of congruent arc
$\mathcal{R}$\thinspace sine of \protect\glslink{congruent_bhuja}{congruent arc} divided by 60 (Radius).

krantisutra
\Nityananda\ defines the \gls{circle_of_declination} in the \Siddhantasindhu, Part II.5, verse~1 (f.\thinspace 19: 19--20~Kh):\\[5pt]
        \tsans{svagavi.suvadhruvayugmoparigatamiha bhavati yadv.rttam || tatkraantisuutrasa.m}\tsnb{ज्ञं} \tsans{puurvaacaaryairvinirdi.s.tam} ||

wholesum
tacitly, the additional amount added to the sum (of arcs) to bring the total up to a one-quarter (90\degree), one-half (180\degree), three-quarters (270\degree), or one-whole (360\degree) turn of a circle.

gola
\textit{tacitly}, the declination or orientation of a celestial object in the northern or southern halves of the celestial sphere.

revolution
\textit{tacitly}, the place values higher than degrees in a sexagesimal number; in other words, 
        place values of the order of 60\textsuperscript{\textit{n}} with \textit{n} =1,\,2,\,3\dots. Also identified with the position of integer-revolutions (in sexagesimal numbers) for angular measures of arc.
        
anka
 \textit{polysemously} a digit/number, a mark/sign, a curve/arc, \etcp.

first arc
the arc of the solstitial colure between the ecliptic and the great circle passing through the two equinoctial points and a celestial object [\ie the \protect\gls{circle_congruent_ecliptic}]
COMPARE WITH maximum latitude (Sanskrit)
the arc of the solstitial colure between the ecliptic and the \protect\gls{circle_congruent_ecliptic} [\ie the great circle passing through the two equinoctial points and a celestial object].


second arc
the arc of the solstitial colure between the celestial equator and the great circle passing through the two equinoctial points and a celestial object [\ie the \protect\gls{circle_congruent_ecliptic}]. 
COMPARE WITH maximum declination (Sanskrit)      
        
\textcite[232]{Truschke} rightly observes that `characterizing Mughal cross-cultural interests as ongoing encounters along a cultural frontier usefully highlights the processes whereby members of largely discrete traditions came into contact with one another and worked out dynamic relationships between their cultural and political worlds'.


\newpage
description of compound words (\textit{samāsa}), their segmentation (\textit{samāsa-padaccheda}), clustering of syntactic pairs of adjacent segments iteratively (\textit{sāmarthya-nirdhāraṇa}, identifying the type of compound for each cluster (\textit{samāsa-bheda-nirdhāraṇa}).

% In another place though, \Nityananda\ discusses Arabic and Persian orthography in Sanskrit. In the first section (\textit{prathama-prakāra}) of the prolegomenon (\textit{granthārambha}) of the \Siddhantasindhu\  (on f.\thinspace 8r:\thinspace 11--16~Kh) 
% \Nityananda\ goes into a full discussion, in Sanskrit, on  . According to him, the word \zij\ is well known in Persian (\textit{phārasī}) as \textit{jīga} but in the Arab world (\textit{āraba-deśa}) it becomes \textit{jīja} with linguistic corruption (\textit{apabhraṃśa}) brought on by the absence of the letter `\textit{ga}' (\textit{gakāra-abhāvāt}) in Arabic. People living in the Arab country then read the said word with the letter `\textit{ja}' instead of the letter `\textit{ga}'. I have not been able to locate this statement in the first section (\qism) of the prolegomenon (\muqaddima) of \MullaFarid's \ZijiShahJahani.}

% \tsans{makkaa} as the Sanskrit phonological calque of the Persian word \tfarsi{مكّه} (\makkah) `the city of Mecca'. This is made evident in several instances in this chapter, \eg \tsans{makkaapure} (\textit{makkāpure}) `in the city of Mecca' (f.\thinspace 27r:\thinspace 13~Kh) or (\textit{makkānagaraṃ}) `the city of Mecca' (f.\thinspace 27r:\thinspace 21~Kh).     
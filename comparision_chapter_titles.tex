
% I include corresponding chapter titles (in Persian) from the second discourse (\maqala\idafaconsonant\ \duvum)\footnote{\,In this study, I follow the EI3 standard of phonetic transcription in Brill's \textit{Encyclopedia of Islam}, third edition, to render Arabic and Persian text into Latin characters, \vid\ \textcite{EIslam}.} of \ZijUlughBeg\ for comparison. These are taken from the Persian edition of the \ZijUlughBeg\ by Louis P.\thinspace E. Amélie Sédillot, \vid\ \textcite[Persian edition on pp.\thinspace 341--386\thinspace(\tfarsi{۳۸۶--۳۴۱})]{Sedillotedition}. The orthography of the Persian text in Sédillot's 1847 edition is modified to match modern spelling conventions. I provide English translations of the Persian titular text for readers unfamiliar with Persian with technical terms.


% Folia 17r--28v of MS~Kh contain Part II (\dvitiya\ \kanda) of \Nityananda's \Siddhantasindhu. The Sanskrit chapter-titles of the twenty-two chapters (\adhyaya) in Part II are presented below (in \Nagari) along with their English translations. At the end of every titular text, the folio and line numbers corresponding to its location in MS~Kh is included in parenthesis. Suspected scribal errors are corrected with explanations in footnotes. Technical terms and phrases are emboldened in the English translation for comparison with similar constructions in the Persian text from \ZijiShahJahani. These are accompanied by corresponding Roman transliterations of Sanskrit expressions in parenthesis.

\setcounter{footnote}{0}
\renewcommand{\thefootnote}{\arabic{footnote}} % for restarting the footnote numbering 
\setlength{\footnotesep}{\baselineskip} % for space between footnotes

\begin{multicols}{2}
\noindent\reversemarginpar\marginnote{\hypertarget{PiiInc}{Discourse~II\\\textit{incipit}}}
\tfarsi{مقاله دوم \hfill}\\
\tfarsi{در معرفت اوقات و طالع هر وقت و آنچه تعلق بدان دارد مشتمل بر بیست و دو باب.%
\te{(f.\thinspace 14a:\thinspace 1~\SjB)}\hfill}
\columnbreak

\noindent\normalmarginpar\marginnote{\hypertarget{SiiInc}{Part~II\\\textit{incipit}}}%
\tsans{nityaanandasvaruupaaya saccidvayamurtaye~||\\ advitiiyaaya vibhave.anantaaya brahma.ne nama.h~||}\\[5pt]
\tsnb{अथ द्वितीयकाण्डे द्वाविंशत्यध्यायैरभिमतसमयस्तात्कालिकलग्नं च तदुपयोगीन्यपि \\ ज्ञायन्ते~॥}%
\enskip (f.\thinspace 17r:\thinspace1--3~Kh)
\end{multicols}%
%
\begin{multicols}{2}
Second Discourse\\
On the \gls{knowledge} (\marifat) of [finding] the \glspl{time} (\avqat) and the \gls{ascendant_each_time} (\tali\ \har\ \vaqt), and whatever belongs to it [\ie all things related to this topic], including twenty-two chapters.
\columnbreak

Obeisance to Brahman (\textit{brahman}) who is the embodiment of eternal bliss (\textit{nitya-ānanda-svarūpa}), who is the form of both existence and thought (\textit{sat-cit-dvaya-mūrti}), who is [the One] without a second (\textit{advitīya}), who is omnipresent (\textit{vibhū}), [and] who is infinite (\textit{ananta}).\\[5pt] 
Now, in the second part, [finding] the \gls{desired_time} (\textit{abhimata-samaya}) and the \gls{ascendant_that_time} (\textit{tātkālika-lagna}), as well as [things] using that, are understood with twenty-two chapters. 
\end{multicols}

\newpage %%%%________________________NEWPAGE

%%%%%%%%%%%%%%%%%%%%%%%%%%%%%%%%%%%
\begin{multicols}{2}
\noindent\reversemarginpar\marginnote{\hypertarget{Pii1}{Discourse~II.1}}
\tfarsi{باب اوّل\\
در بیان معرفت جنس هر یک از حاصل ضرب و خارج قسمت و جذر، یعنی دانستن آنکه حاصل ضرب یا خارج قسمت یا جذر از کدام مرتبه است از مراتب مرفوعات و درج و اجزاء درج مثل دقایق و ثواني و غیر آن.% 
\te{(f.\thinspace 14a:\thinspace 3--1~\SjB)}\hfill}
\columnbreak

\noindent\normalmarginpar\marginnote{\hypertarget{Sii1}{Part~II.1}}%
\tsnb{तत्र प्रथमाध्याये\te{%
\footnote{~\tsans{prathamaadhyaaye} {$\Big ]$} \tsans{prathamadhyaaye}\enskip Kh. A regular sandhi of the words \tsans{prathama}\,$_\text{\acrshort{modifier}}$ + \tsans{adhyaaye}\,$_\text{\acrshort{locative}-\acrshort{singular}}$ generates the compound \tsans{pratha\underline{maa}dhyaaye}. The omission of the long vowel (\textit{ā}-diacritic in \tsans{maa} looks like a scribal oversight.}} %
गुणनभजनफले मूलं च परिवर्तादिस्थानैः कलादिस्थानैर्वा किं\linebreak जातीयं स्यादिति ज्ञायते~॥
तत्र गणितसौकर्यार्थं\te{%
\footnote{~\tsans{ga.nita\selip}{$\Big ]$} \tsans{ga.niita\selip}\enskip Kh. There are no morphological divisions of \tsans{ga.niita} that are syntactically admissible in the larger compound \tsans{ga.niitasaukaryaartha.m}, whereas the word \tsans{ga\underline{.ni}ta} is both grammatically well-formed and contextually apposite. I suspect the use of the long vowel (the \textit{ī}-diacritic in \tsans{.nii}) is either a scribal hypercorrection or an inadvertent misspelling.\label{grammatical_abbreviation_example}}} %
यवनप्रसिद्धप्रकारेणाङ्कस्थानानां संस्कृतशब्दैः संज्ञा कल्प्यते~॥} \\
(f.\thinspace 17r:\thinspace3--5~Kh)
\end{multicols}%
%
\begin{multicols}{2}
First Chapter\\
On the expression of the \gls{knowledge} (\marifat) of each \gls{genus} (\jins) [of digits] from the \gls{result_multiplication} (\hasil\idafaconsonant\ \darb), and the \gls{quotient_division} (\kharij\idafaconsonant\ \qismat), and the \gls{square_root} (\jadr). In other words, to know what is the \gls{position} (\martaba) of [the digits in] the \gls{result_multiplication} (\mbox{\hasil\idafaconsonant} \darb), or the \gls{quotient_division} (\mbox{\kharij\idafaconsonant} \qismat), or the \gls{square_root} (\jadr), from the \glspl{position} (\maratib) of \glspl{elevation_sexagesimal} (\marfuat) [\scl integer number of revolutions], and the \gls{degree} (\daraj), and the \glslink{parts_of_degree}{fractional parts of a degree} (\ajza\idafaconsonant\ \daraj) like \glspl{minute} (\daqaiq) and \glspl{second} (\thawani) and so on.
\columnbreak

There, in the first chapter, with [digits in] the \glspl{position} (\textit{sthāna}) of \gls{revolution} (\textit{parivarta}) \etc or with [digits in] the \glspl{position} (\textit{sthāna}) of \gls{minute} (\textit{kalā}) \etc in the \gls{result_multiplication_division} (\textit{guṇana-bhajana-phala}), and the \gls{square_root} (\textit{mūla}), what [digits] should \gls{belonging_genus} (\textit{jātīya}): this is understood. Therein, for the purpose of facilitating ease in computations (\textit{gaṇita-saukarya-artha}) with the method famous amongst the foreigner (\textit{yavana-prasiddha}), the name of the \glspl{position} (\textit{sthāna}) of \glspl{digit} (\textit{aṅka}) is considered with Sanskrit words.
\end{multicols}
\medskip

\newpage %%%%________________________NEWPAGE

%%%%%%%%%%%%%%%%%%%%%%%%%%%%%%%%%%%
\begin{multicols}{2}
\label{chapter_number_example}
\noindent\reversemarginpar\marginnote{\hypertarget{Pii2}{Discourse~II.2}}%
\tfarsi{باب دوم\\
در عمل تعدیل ما بين السطرين. \\\te{(f.\thinspace 14b\thinspace 6~\SjB)}\hfill}
\columnbreak

\noindent\normalmarginpar\marginnote{\hypertarget{Sii2}{Part~II.2}}%
\tsnb{अथ द्वितीयाध्याये द्विकोष्ठान्तरोत्थफल\-साधनम् ॥} %
\tsans{tasya muula.m trairaa"sikam || atra yavanaa.h parasparasambandhicatuuraa"siinga.nayanti~|| tallak.sa.na.m ca ||}\enskip (f.\thinspace 18r:\thinspace2--4~Kh)
\end{multicols}%
%
\begin{multicols}{2}
Second chapter\\
On the \gls{method_of_interpolation} (\amal\idafaconsonant\ \tadil) 
\gls{between_two_lines} (\ma\ \bayn\ \alsatrayn) [of a table].
\columnbreak

Now, in the second chapter, the \gls{demonstration} (\textit{sādhana}) of the \gls{result} (\textit{phala}) derived from the \gls{difference_between_two_cells} (\textit{dvi-koṣṭha-antara}). The basis of this is the \gls{rule_of_three} (\textit{trai-rāśika}). Here, the foreigner (\textit{yavana}) take into account four \gls{correlated_numbers} (\textit{paraspara-sambandhi-rāśi}). And their \glspl{definition} (\textit{lakṣaṇa}) [are first stated]. 
\end{multicols}
\medskip

%%%%%%%%%%%%%%%%%%%%%%%%%%%%%%%%%%%

\begin{multicols}{2}
\noindent\reversemarginpar\marginnote{\hypertarget{Pii3}{Discourse~II.3}}%
\tfarsi{باب سيوم\\
در معرفت جيب و سهم.%
\te{(f.\thinspace 15a:\thinspace 8~\SjB)}\hfill}
\columnbreak

\noindent\normalmarginpar\marginnote{\hypertarget{Sii3}{Part~II.3}}%
\tsnb{अथ तृतीयाध्याये ज्याशरज्ञानम्~॥}\\
(f.\thinspace 18r:\thinspace23--24~Kh)
\end{multicols}%
%
\begin{multicols}{2}
Third chapter\\
On the \gls{knowledge} (\marifat) of the \gls{Sine} (\jayb) and the \gls{Sagitta} (\sahm) [\ie the Versed Sine].
\columnbreak

Now, in the third chapter, the \gls{knowledge} (\textit{jñāna}) of the \gls{Sine} (\textit{jyā}) and the \gls{Versed_Sine} (\textit{śara}).
\end{multicols}
\medskip

%%%%%%%%%%%%%%%%%%%%%%%%%%%%%%%%%%%

\begin{multicols}{2}
\noindent\reversemarginpar\marginnote{\hypertarget{Pii4}{Discourse~II.4}}%
\tfarsi{باب چهارم\\
در معرفت ظلّ.%
\te{(f.\thinspace 15b:\thinspace 1~\SjB)}\hfill}
\columnbreak

\noindent\normalmarginpar\marginnote{\hypertarget{Sii4}{Part~II.4}}%
\tsnb{अथ चतुर्थाध्याये छायाज्ञानम्~॥}\\ 
(f.\thinspace 19r:\thinspace13~Kh)
\end{multicols}%
%
\begin{multicols}{2}
Fourth chapter\\
On the \gls{knowledge} (\marifat) of the \glslink{shadow}{shadow} (\zill) [of a gnomon].
\columnbreak

Now, in the fourth chapter, the \gls{knowledge} (\textit{jñāna}) of the \glslink{shadow}{shadow} (\textit{chāyā}) [of a gnomon].
\end{multicols}

\newpage %%%%________________________NEWPAGE

%%%%%%%%%%%%%%%%%%%%%%%%%%%%%%%%%%%

\begin{multicols}{2}
\noindent\reversemarginpar\marginnote{\hypertarget{Pii5}{Discourse~II.5}}%
\tfarsi{باب پنجم\\
در معرفت ميل اجزاء فلك البروج از معدّل النهار.%
\te{(f.\thinspace 16a:\thinspace 4~\SjB)}\hfill}
\columnbreak

\noindent\normalmarginpar\marginnote{\hypertarget{Sii5}{Part~II.5}}%
\tsnb{अथ पञ्चमाध्याये क्रान्तिज्ञानम् ॥ \\तत्र तावत्क्रान्तिसूत्रादिसंज्ञोच्यते~॥}\\
(f.\thinspace 19v:\thinspace18--19~Kh)
\end{multicols}%
%
\begin{multicols}{2}
Fifth chapter\\
On the \gls{knowledge} (\marifat) of the \gls{declination_parts_ecliptic} (\mayl\idafaconsonant\ \ajza\idafaconsonant\ \falak\ \alburuj) from the \gls{celestial_equator} (\muaddil\ \alnahar).
\columnbreak

Now, in the fifth chapter, the \gls{knowledge} (\textit{jñāna}) of the \gls{declination} (\textit{krānti}). There, firstly, a [technical] term beginning with \gls{circle_of_declination} (\textit{krānti-sūtra}) is stated.
\end{multicols}
\medskip
%%%%%%%%%%%%%%%%%%%%%%%%%%%%%%%%%%%

\begin{multicols}{2}\label{discourse_part_6_chapter_title_example}
\noindent\reversemarginpar\marginnote{\hypertarget{Pii6}{Discourse~II.6}}%
\tfarsi{باب ششم\\
در معرفت بعد كواكب از معدّل النهار.\\ \te{(f.\thinspace 16a:\thinspace 26--25~\SjB)}\hfill}
\columnbreak

\noindent\normalmarginpar\marginnote{\hypertarget{Sii6}{Part~II.6}}%
\tsnb{अथ षष्ठाध्यायस्पष्टक्रान्तिज्ञानम्~॥}\\
(f.\thinspace 20r:\thinspace16~Kh)
\end{multicols}%
%
\begin{multicols}{2}
On the \gls{knowledge} (\marifat) of the \gls{distance_celestial_object} (\bud\idafaconsonant\ \kawakib\ \az\ \muaddil\ \alnahar).
\columnbreak

Sixth chapter\\
Now, the \gls{knowledge} (\textit{jñāna}) of the \gls{true_declination} (\textit{spaṣṭa-krānti}) in the sixth chapter.
\end{multicols}
\medskip
%%%%%%%%%%%%%%%%%%%%%%%%%%%%%%%%%%%

\begin{multicols}{2}
\noindent\reversemarginpar\marginnote{\hypertarget{Pii7}{Discourse~II.7}}%
\tfarsi{باب هفتم\\
در معرفت غایت ارتفاع و انخفاض كواكب.%
\te{(f.\thinspace 16b:\thinspace 11--10~\SjB)}\hfill}
\columnbreak

\noindent\normalmarginpar\marginnote{\hypertarget{Sii7}{Part~II.7}}%
\tsnb{अथ सप्तमाध्याये ग्रहस्य परमोन्नतांशाना\-मधःस्थपरमभागानां च ज्ञानम्~॥}\\
(f.\thinspace 20v:\thinspace12--13~Kh)
\end{multicols}%
%
\begin{multicols}{2}
Seventh chapter\\
On the \gls{knowledge} (\marifat) of the \gls{maximum_elevation_depression_celestial_object} (\ghayat\idafaconsonant\ \irtifa\ \va\ \inkhifad\idafaconsonant\ \kawakib).
\columnbreak

Now, in the seventh chapter, the \gls{knowledge} (\textit{jñāna}) of the \gls{degrees_maximum_elevation} (\textit{parama-unnata-aṃśa}) and the \gls{degrees_maximum_depression} (\textit{adhaḥstha-parama-bhāga}) of a \gls{celestial_object} (\textit{graha}).
\end{multicols}

\newpage %%%%________________________NEWPAGE

%%%%%%%%%%%%%%%%%%%%%%%%%%%%%%%%%%%

\begin{multicols}{2}
\noindent\reversemarginpar\marginnote{\hypertarget{Pii8}{Discourse~II.8}}%
\tfarsi{باب هشتم\\
در معرفت مطالع خطّ استوا و آنرا مطالع فلك مستقیم نیز گویند.%
\te{(f.\thinspace 16b:\thinspace 17~\SjB)}\hfill}
\columnbreak

\noindent\normalmarginpar\marginnote{\hypertarget{Sii8}{Part~II.8}}%
\tsnb{अथ अष्टमाध्याये व्यक्षोदयांशज्ञानम् ॥\\ तेषां लङ्कोदयांशसंज्ञाप्युच्यते~॥}~%
\tsans{tallak.sa.namaaha~||}\enskip(f.\thinspace 20v:\thinspace24--25~Kh)
\end{multicols}%
%
\begin{multicols}{2}
Eighth chapter\\
On the \gls{knowledge} (\marifat) of \gls{ascension_line_terrestrial_observer} (\matali\ \khatt\idafaconsonant\ \istiva) [\ie the right ascensions of the zodiacal signs]. And that is also called the \gls{ascension_right_sphere} (\matali\ \falak\idafaconsonant\ \mustaqim).
\columnbreak

Now, in the eighth chapter, the \gls{knowledge} (\textit{jñāna}) of the \gls{rising_terrestrial_equator_degrees} (\textit{vyakṣa-udaya-aṃśa}) [\ie the right ascensions of the degrees of the ecliptic]. All of them are also called the \gls{rising_Lanka_degrees} (\textit{laṅkā-udaya-aṃśa}) by name. Their \glspl{definition} (\textit{lakṣaṇa}) state [as follows].
\end{multicols}
\medskip
%%%%%%%%%%%%%%%%%%%%%%%%%%%%%%%%%%%

\begin{multicols}{2}
\noindent\reversemarginpar\marginnote{\hypertarget{Pii9}{Discourse~II.9}}%
\tfarsi{باب نهم\\
در معرفت تعديل النهار و قوس النهار و قوس الليل و ساعات النهار و ساعات الليل.%
\te{(f.\thinspace 16b:\thinspace 25~\SjB)}\hfill}
\columnbreak

\noindent\normalmarginpar\marginnote{\hypertarget{Sii9}{Part~II.9}}%
\tsnb{अथ नवमाध्याये चरदिनरात्रिवामानां दिनरात्रिहोरादीनां च ज्ञानम्~॥} %
\tsans{tatra taavatte.saa.m lak.sa.nam~||}%
\enskip (f.\thinspace 21r:\thinspace5--6~Kh)
\end{multicols}%
%
\begin{multicols}{2}
Ninth chapter\\
On the \gls{knowledge} (\marifat) of the \gls{equation_of_daylight} (\tadil\ \alnahar); and the \gls{arc_of_daylight} (\qaws\ \alnahar) and the \gls{arc_of_night} (\qaws\ \allayl); and the 
\gls{hours_of_daylight} (\saat\ \alnahar) and \gls{hours_of_night} (\saat\ \allayl).
\columnbreak

Now, in the ninth chapter, the \gls{knowledge} (\textit{jñāna}) of the \gls{ascensional_difference} (\textit{cara}) of the \gls{oblique_diurnal_circle} (\textit{dina-rātri-vāma}[-\textit{vrtta}]) and of the \gls{hours_of_day_and_night} (\textit{dina-rātri-horā}) \etcp\ There, firstly, the \glspl{definition} (\textit{lakṣaṇa}) of those [are stated].
\end{multicols}
\medskip
%%%%%%%%%%%%%%%%%%%%%%%%%%%%%%%%%%%

\begin{multicols}{2}
\noindent\reversemarginpar\marginnote{\hypertarget{Pii10}{Discourse~II.10}}%
\tfarsi{باب دهم\\
در معرفت مطالع بلد.\te{(f.\thinspace 17a\thinspace 26~\SjB)}\hfill}
\columnbreak

\noindent\normalmarginpar\marginnote{\hypertarget{Sii10}{Part~II.10}}%
\tsnb{अथ दशमाध्याये निजोदयांशज्ञानम्~॥}\\
\tsans{tallak.sa.na.m ca~||}\enskip (f.\thinspace 21v:\thinspace23~Kh)
\end{multicols}%
%
\begin{multicols}{2}
Tenth chapter\\
On the \gls{knowledge} (\marifat) of the \gls{ascension_ecliptic_locality} (\matali\idafaconsonant\ \balad) [\ie the oblique ascensions of the zodiacal signs].
\columnbreak

Now, in the tenth chapter, the \gls{knowledge} (\textit{jñāna}) of the \gls{rising_location_degree} (\textit{nija-udaya-aṃśa}) [\ie the oblique ascensions of the degrees of the ecliptic]. And their \glspl{definition} (\textit{lakṣaṇa}) [are first stated]. 
\end{multicols}

\newpage %%%%________________________NEWPAGE

%%%%%%%%%%%%%%%%%%%%%%%%%%%%%%%%%%%

\begin{multicols}{2}
\noindent\reversemarginpar\marginnote{\hypertarget{Pii11}{Discourse~II.11}}%
\tfarsi{باب یازدهم \\
در عمل عکس مطالع يعنی معرفت طوالع از مطالع\te{\footnote{~The words \tfarsi{از مطالع} are a marginal addition. They appears in the exterior (left) margin of f.\thinspace 17a~\SjB\ alongside line 5 of the text. The main text has an interlinear insertion mark `$\vee$' at the end of the preceding \tfarsi{طوالع}. The marginal text ends with a terminal \textit{number-like} mark \tfarsi{۴} which, according to \textcite[117]{Gacek}, is an abbreviation for \tfarsi{تمام شد} \textit{tamām shud} `ended/finished' often seen in manuscripts of Indian/Iranian origins.}} بعمل.%
\te{(f.\thinspace 17b:\thinspace 5~\SjB)}\hfill}
\columnbreak

\noindent\normalmarginpar\marginnote{\hypertarget{Sii11}{Part~II.11}}%
\tsnb{अथैकादशाध्याये स्वोदयांशेभ्यो विनैव कोष्ठ\-कैर्विलोमक्रियातो} \mbox{\tsnb{विलग्नांशकज्ञानम्\thinspace॥}}
\tsans{vilomakriyaalak.sa.nam~||}\enskip (f.\thinspace 22r:\thinspace9--11~Kh)
\end{multicols}%
%
\begin{multicols}{2}
Eleventh chapter\\
On the \gls{inverse_method} (\amal\idafaconsonant \aks) [of] \glsuseri{ascension_measure} (\matali); in other words, the knowledge of the [ecliptic degrees of the] \glspl{ascendant} (\tawali) from the [local] \glsuseri{ascension_measure} (\matali) [\ie  from the oblique ascensions of the ascendants] by direct calculation. 
\columnbreak

Now, in the eleventh chapter, the \gls{knowledge} (\textit{jñāna}) of the \glslink{ascendant_ecliptic_degrees}{[ecliptic] degrees of the ascendants} (\textit{vilagna-aṃśaka}) from the \gls{rising_location_degree} (\textit{sva-udaya-aṃśa}) [\ie from the oblique ascensions of the ascendants] without [using] the \glspl{table} (\textit{koṣṭhaka}) [and] by using the \gls{inverse_procedure} (\textit{viloma-kriyā}). The \gls{definition} (\textit{lakṣaṇa}) of the \gls{inverse_procedure} (\textit{viloma-kriyā}) [is first stated].
\end{multicols}
\medskip
%%%%%%%%%%%%%%%%%%%%%%%%%%%%%%%%%%%

\begin{multicols}{2}
\noindent\reversemarginpar\marginnote{\hypertarget{Pii12}{Discourse~II.12}}%
\tfarsi{باب دوازدهم \\
در معرفت مطالع  ممرّ و درجهٔ ممرّ كوكب.%
\te{(f.\thinspace 17b:\thinspace30--29~\SjB)}\hfill}
\columnbreak

\noindent\normalmarginpar\marginnote{\hypertarget{Sii12}{Part~II.12}}%
\tsnb{अथ द्वादशाध्याये नक्षत्रस्य लङ्कायामुदये जाते सति भोदयलग्नव्यक्षोदयांशभोदय-\linebreak लग्नांशयोर्ज्ञानम्~॥} ~\tsans{tallak.sa.nam~||}\\
(f.\thinspace 22v:\thinspace15--17~Kh)
\end{multicols}%
%
\begin{multicols}{2}
Twelfth chapter\\
On the \gls{knowledge} (\marifat) of the \gls{ascension_transit} (\matali\idafaconsonant\ \mamarr) [\ie the right ascension of the zodiacal sign culminating with a celestial object] and the \gls{degree_transit_celestial_object} (\daraji\idafavowel\ \mamarr\idafaconsonant\ \kawkab) [\ie the ecliptic longitude of the zodiacal sign culminating with a celestial object].
\columnbreak

Now, in the twelfth chapter, when a \gls{celestial_object} (\textit{nakṣatra}) \glslink{rising}{rises} (\textit{udaya}) at \Lanka\ (the terrestrial equator), the \gls{knowledge} (\textit{jñāna}) of the \gls{degree_equatorial_ascension} (\textit{bha-udaya-lagna-vyakṣa-udaya-aṃśa}) [\ie the right ascension of the zodiacal sign culminating with the celestial object] and the \gls{degree_meridian_ecliptic_rising_time} (\textit{bha-udaya-lagna-aṃśa}) [\ie the ecliptic longitude of the zodiacal sign culminating with the celestial object]. Their \glspl{definition} (\textit{lakṣaṇa}) [are first stated]. 
\end{multicols}

\newpage %%%%________________________NEWPAGE

%%%%%%%%%%%%%%%%%%%%%%%%%%%%%%%%%%%


\begin{multicols}{2}
\noindent\reversemarginpar\marginnote{\hypertarget{Pii13}{Discourse~II.13}}%
\tfarsi{باب سیزدهم \\
در مطالع طالع و غروب كواكب.%
\te{(f.\thinspace 18a:\thinspace12--11~\SjB)}\hfill}
\columnbreak

\noindent\normalmarginpar\marginnote{\hypertarget{Sii13}{Part~II.13}}%
\tsnb{अथ त्रयोदशाध्याये नक्षत्रस्योदयसमये ऽस्तसमये च निजोदयांशकज्ञानम्~॥}\\
\tsans{tallak.sa.na.m puurvaardhamadhye\te{\footnote{~\tsans{puurvaardhamadhye} {$\Big ]$} \tsans{puurvaadhamaye}\enskip Kh. The compound \tsans{puurvaadhamaye} in Kh can be  segmented as \tsans{puurva}\,$_\text{\acrshort{modifier}}$ + \tsans{adha}\,$_\text{\acrshort{modifier}}$ + \tsans{maye}\,$_\text{\acrshort{locative}-\acrshort{singular}}$; however, this reading is neither syntactically nor contextually coherent with the rest of the sentence. The omission of over-letter \textit{r}-diacritic (\textit{repha}) in \tsans{rdha} and confusing the glyph \tsans{ye} for the ligature \tsans{dhye} are fairly common scribal mistakes.}} %
proktameva~||}\\ (f.\thinspace 23r:\thinspace6--7~Kh)
\end{multicols}%
%
\begin{multicols}{2}
Thirteenth chapter\\
On the [right] \glsuseri{ascension_measure} (\matali) of the \gls{rising} (\tali) and \gls{setting} (\ghurub) of \glspl{celestial_object} (\kawakib).
\columnbreak

Now, in the thirteenth chapter, at the \gls{time_rising} (\textit{udaya-samaya}) and \gls{time_setting} (\textit{asta-samaya}) of a \gls{celestial_object} (\textit{nakṣatra}), the \gls{knowledge} (\textit{jñāna}) of the \gls{rising_location_degree} (\textit{nija-udaya-aṃśaka}) [\ie the oblique ascensions of the degrees of the ecliptic]. The \gls{definition} (\textit{lakṣaṇa}) of that has already been declared in the first half [of Part~II].
\end{multicols}
\medskip
%%%%%%%%%%%%%%%%%%%%%%%%%%%%%%%%%%%

\begin{multicols}{2}
\noindent\reversemarginpar\marginnote{\hypertarget{Pii14}{Discourse~II.14}}%
\tfarsi{باب چهاردهم \\
در معرفت سمت از ارتفاع یا انخفاض.%
\te{(f.\thinspace 18a:\thinspace20~\SjB)}\hfill}
\columnbreak

\noindent\normalmarginpar\marginnote{\hypertarget{Sii14}{Part~II.14}}%
\tsnb{अथ चतुर्दशाध्याये ऽभीप्सितोन्नतांशाधरांशेभ्यः\te{\footnote{~\tsans{.abhiipsitonnataa.m"saadharaa.m"sebhya.h} {$\Big ]$} \tsans{.abhiipsitonnata.m"saadhaaraa"sebhya}\enskip Kh. A regular sandhi of the words \tsans{.abhiipsita}\,$_\text{\acrshort{modifier}}$ + \tsans{unnata}\,$_\text{\acrshort{modifier}}$ + \tsans{a.m"sa}\,$_\text{\acrshort{modifier}}$ + \tsans{adhara}\,$_\text{\acrshort{modifier}}$  + \tsans{a.m"sebhya.h}\,$_\text{\acrshort{dative}/\acrshort{ablative}-\acrshort{plural}}$ generates the contextually apposite compound \tsans{.abhiipsitonna\underline{taa.m}"saa\underline{dha}raa.m"sebhya.h}. Kh attests \tsans{aadhaara}\,$_\text{\acrshort{modifier}}$ `support/base' instead of \tsans{adhara}\,$_\text{\acrshort{modifier}}$ `lower' in the chapter-title, but then uses \tsans{adhara}\,$_\text{\acrshort{modifier}}$ in several other places in this chapter. I suspect the irregular vowel-marks (the \textit{a}-diacritic in \tsans{ta.m} and \textit{ā}-diacritic in \tsans{dhaa}) in \tsans{.abhiipsitonnata.m"saadhaaraa"sebhya} are scribal mistakes (just like the missing \textit{anusvāra} over \tsans{raa}  or the missing \textit{visarga} in \tsans{bhya}).}} %
स्वदिगंशज्ञानम्~॥}\enskip
(f.\thinspace 23r:\thinspace24~Kh)
\end{multicols}%
%
\begin{multicols}{2}
Fourteenth chapter\\
On the \gls{knowledge} (\marifat) of the \gls{azimuth} (\samt) from the \gls{elevation} (\irtifa) or the \gls{depression} (\inkhifad) [of a celestial object].
\columnbreak

Now, in the fourteenth chapter, the \gls{knowledge} (\textit{jñāna}) of the \gls{azimuth_own_location} (\textit{sva-diś-aṃśa}) from the \gls{desired_degree_elevation} (\textit{abhīpsita-unnata-aṃśa}) and the \gls{degrees_depression} (\textit{adharā-aṃśa}) [of a celestial object].
\end{multicols}

\newpage %%%%________________________NEWPAGE

%%%%%%%%%%%%%%%%%%%%%%%%%%%%%%%%%%%

\begin{multicols}{2}
\noindent\reversemarginpar\marginnote{\hypertarget{Pii15}{Discourse~II.15}}%
\tfarsi{باب پانزدهم \\
در معرفت ارتفاع از سمت. \\
\te{(f.\thinspace 18b:\thinspace8--7~\SjB)}\hfill}
\columnbreak

\noindent\normalmarginpar\marginnote{\hypertarget{Sii15}{Part~II.15}}%
\tsnb{अथ पञ्चदशाध्याये दिगंशेभ्यो ऽभीष्टोन्नतांशाधरांशज्ञानम्~॥} %
\tsans{tatraananyatvaprakaaro\-papatti.h~||}
\enskip
(f.\thinspace 23v:\thinspace21--22~Kh)
\end{multicols}%
%
\begin{multicols}{2}
Fifteenth chapter\\
On the \gls{knowledge} (\marifat) of the \gls{elevation} (\irtifa) [of a celestial object] from [its] \gls{azimuth} (\samt).
\columnbreak

Now, in the fifteenth chapter, the \gls{knowledge} (\textit{jñāna}) of the \gls{desired_degree_elevation} (\textit{abhīṣṭa-unnata-aṃśa}) and the \gls{degrees_depression} (\textit{adharā-aṃśa}) [of a celestial object] from the \gls{azimuth_own_location} (\textit{sva-diś-aṃśa}). There, a \gls{demonstration} (\textit{upapatti}) by \gls{method_of_identity} (\textit{ananyatva-prakāra}) [is stated].
\end{multicols}
\medskip

%%%%%%%%%%%%%%%%%%%%%%%%%%%%%%%%%%%

\begin{multicols}{2}
\noindent\reversemarginpar\marginnote{\hypertarget{Pii16}{Discourse~II.16}}%
\tfarsi{باب شانزدهم \\
در معرفت خطّ نصف النهار.\\
\te{(f.\thinspace 18b:\thinspace16--15~\SjB)}\hfill}
\columnbreak

\noindent\normalmarginpar\marginnote{\hypertarget{Sii16}{Part~II.16}}%
\tsnb{अथ षोडशाध्याये याम्योतररेखाज्ञानम्~॥} %
\tsans{tasya eva naama madhyaahnarekheti\te{\footnote{~\tsans{naama madhyaahnarekheti} {$\Big ]$} \tsans{naama\thinspace dhyaahnarekheti}\enskip Kh. The technical word \tsans{madhyaahnarekhaa} is grammatically well-formed and contextually apposite to the discussions in this chapter. The words \tsans{naama} and \tsans{dhyaahnarekheti} occurs across a line break (lines 14 and 15) in Kh. This appears to be a haplography: the scribe inadvertently left out the second \tsans{ma} while copying.\label{emeneded_attested_sanskrit_example}}}~||}\\
(f.\thinspace 24r:\thinspace14--15~Kh)
\end{multicols}%
%
\begin{multicols}{2}
Sixteenth chapter\\
On the \gls{knowledge} (\marifat) of the \gls{line_midday} (\khatt\ \nisf\ \alnahar) [\ie the local meridian line].
\columnbreak

Now, in the sixteenth chapter, the \gls{knowledge} (\textit{jñāna}) of the \gls{meridian_line} (\textit{yāmya-uttara-rekhā}).~It is even called the \gls{line_midday} (\textit{madhyāhna-rekhā}). 
\end{multicols}
\medskip

%%%%%%%%%%%%%%%%%%%%%%%%%%%%%%%%%%%

\begin{multicols}{2}
\noindent\reversemarginpar\marginnote{\hypertarget{Pii17}{Discourse~II.17}}%
\tfarsi{باب هفدهم \\
در معرفت طول و عرض بلد. \\
\te{(f.\thinspace 18b:\thinspace29~SjB)}\hfill}
\columnbreak

\noindent\normalmarginpar\marginnote{\hypertarget{Sii17}{Part~II.17}}%
\tsnb{अथ सप्तदशाध्याये देशान्तराक्षांशज्ञानम्~॥} %
\tsans{tallak.sa.na.m ca~||}\enskip
(f.\thinspace 24v:\thinspace4~Kh)
\end{multicols}%
%
\begin{multicols}{2}
Seventeenth chapter\\
On the \gls{knowledge} (\marifat) of the [terrestrial] \gls{longitude_and_latitude_terrestrial} (\tul\ \va\ \ard\idafaconsonant\ \balad).
\columnbreak

Now, in the seventeenth chapter, the \gls{knowledge} (\textit{jñāna}) of \gls{degrees_terrestrial_long_latitude} (\textit{deśāntara-akṣa-aṃśa}) [in one's own location]. And their \glspl{definition} (\textit{lakṣaṇa}) [are first stated]. 
\end{multicols}

\newpage %%%%________________________NEWPAGE

%%%%%%%%%%%%%%%%%%%%%%%%%%%%%%%%%%%

\begin{multicols}{2}
\noindent\reversemarginpar\marginnote{\hypertarget{Pii18}{Discourse~II.18}}%
\tfarsi{باب هژدهم \\
در معرفت عرض اقلیم رؤیت.\\
\te{(f.\thinspace 19a:\thinspace20--19~SjB)}\hfill}
\columnbreak

\noindent\normalmarginpar\marginnote{\hypertarget{Sii18}{Part~II.18}}%
\tsnb{अथाष्टादशाध्याये दृक्क्षेपदृग्गतिज्ञानम्~॥} %
\tsans{tallak.sa.na.m ca~||}\enskip
(f.\thinspace 25r:\thinspace12--13~Kh)
\end{multicols}%
%
\begin{multicols}{2}
Eighteenth chapter\\
On the \gls{knowledge} (\marifat) of the \gls{latitude_visible_climate}  (\mbox{\ard\idafaconsonant\ \iqlim\idafaconsonant} \ruyat) [\ie the zenith distance of the nonagesimal point].
\columnbreak

Now, in the eighteenth chapter, the \gls{knowledge} (\textit{jñāna}) of \gls{ecliptic_zenith_distance} (\textit{dṛkkṣepa}) and the \gls{ecliptic_pole_zenith_distance} (\textit{drggati}). And their \glspl{definition} (\textit{lakṣaṇa}) [are first stated]. 
\end{multicols}
\medskip

%%%%%%%%%%%%%%%%%%%%%%%%%%%%%%%%%%%

\begin{multicols}{2}
\noindent\reversemarginpar\marginnote{\hypertarget{Pii19}{Discourse~II.19}}%
\tfarsi{باب نوزدهم \\
در استخراج بعد میان دو کوکب. \\
\te{(f.\thinspace 19a:\thinspace28~SjB)}\hfill}
\columnbreak

\noindent\normalmarginpar\marginnote{\hypertarget{Sii19}{Part~II.19}}%
\tsnb{अथैकोनविंशाध्याये\nobreak\te{\footnote{~%
\tsans{athaikonavi.m"saadhyaaye} {$\Big ]$} \tsans{athaikonavi.m"sodhyaaye}\enskip Kh. The locative adverbial phrase \tsans{athaikonavi.m"sodhyaaye} in Kh can be segmented as 
\tsans{atha}\,$_\text{\acrshort{indeclinable}}$ + \tsans{ekonavi.m"sa.h}\,$_\text{\acrshort{nominative}-\acrshort{singular}}$ + \tsans{adhyaaye}\,$_\text{\acrshort{locative}-\acrshort{singular}}$. However, the word \tsans{ekonavi.m"sa.h} `nineteen' is a cardinal number, and if used as an ordinal adjective, it should be in concord with the substantive \tsans{adhyaaye}. The meaning of the phrase `in the nineteenth chapter' is preserved in the compound \tsans{ekonavi.m\underline{"saa}dhyaaye} as well as the words \tsans{ekonavi.m"se .adhyaaye} (with a locative concord). I select the compounded form as it is consistent with the previous chapter-titles in Part II.}} %
द्विनक्षत्रान्तरांशक\-ज्ञानम्~॥}%
\tsans{ tallak.sa.nam~||}\enskip (f.\thinspace 25v:\thinspace6--7~Kh)
\end{multicols}%
%
\begin{multicols}{2}
Nineteenth chapter\\
On the \gls{determination} (\istikhraj) of the \gls{distance_between_celestial_object} (\bud\idafaconsonant\ \miyan\idafaconsonant\ \duvum\idafaconsonant\ \kawkab).
\columnbreak

Now, in the nineteenth chapter, the \gls{knowledge} (\textit{jñāna}) of \gls{degrees_separation_two_celestial_objects} (\textit{dvi-nakṣatra-antara-aṃśaka}). The \gls{definition} (\textit{lakṣaṇa}) of that [is first stated].  
\end{multicols}

\newpage %%%%________________________NEWPAGE

%%%%%%%%%%%%%%%%%%%%%%%%%%%%%%%%%%%

\begin{multicols}{2}
\noindent\reversemarginpar\marginnote{\hypertarget{Pii20}{Discourse~II.20}}%
\tfarsi{باب بیستم \\
در معرفت سمت قبله و انحراف او.\\
\te{(f.\thinspace 19b:\thinspace22--21~SjB)}\hfill}
\columnbreak

\noindent\normalmarginpar\marginnote{\hypertarget{Sii20}{Part~II.20}}%
\tsans{atha vi.m"satime .adhyaaye\te{\footnote{~\tsans{vi.m"satime .adhyaaye}\enskip{$\Big ]$}\enskip\tsans{vi.m"satidhyaaye}\enskip Kh. The attested form is morphologically defective. A regular sandhi of the words \tsans{vi.m"sati}\,$_\text{\acrshort{modifier}}$ + \tsans{adhyaaye}\,$_\text{\acrshort{locative}-\acrshort{singular}}$ generates \tsans{vi.m"sa\underline{tya}dhyaaye}, a locative adverbial phrase meaning  `in twenty chapters'. 
I correct this to \tsans{vi.m"satime .adhyaaye} `in the twentieth chapter' (using the ordinal form \tsans{vi.m"satima} `twentieth' instead of the cardinal number \tsans{vi.m"sati} `twenty') as it is consistent with the next two chapter-titles in Part~II.}} %
 svapure saumyayaamyadigbhyaa.m diga.m"sai.h kaa"sii kvaastiiti%
\te{\footnote{~\tsans{kaa"sii kvaastiiti} {$\Big ]$} \tsans{kaa"siikvaa\char"A8FBmii}$\overset{\text{?}}{\text{/}}$\tsans{sii\char"A8FBti}\,\enskip Kh, (\textit{conjecture}). There are no visible signs of scribal corrections or lacunae, but the writing (in red ink) is partially faded making it difficult to identify the letters with certainty. Nevertheless, there are no combinations of these letters that provide a grammatically valid and contextually apposite reading. I emend the  words to \tsans{kaa"sii kvaa\underline{stii}ti}, \lit the question ``Where is \Kashi?", that serves as the subject of the main sentence. (In other words: ``Where is \Kashi?", \textit{this} is [understood with \dots]). I suspect the scribe unwittingly copied the  glyph \tsans{mii}/\tsans{sii} for the ligature \tsans{stii} as they
often appear very similar in \Nagari\ palaeography.}} %
{\dev{ज्ञा}}yate~||}\enskip(f.\thinspace 26v:\thinspace17--18~Kh)
\end{multicols}%
%
\begin{multicols}{2}
Twentieth chapter\\
On the \gls{knowledge} (\marifat) of the \gls{azimuth_qibla} (\samt\idafaconsonant\ \qibla) and its \glslink{inclination_azimuth_qibla}{inclination} (\inhiraf[\idafaconsonant\ \samt\idafaconsonant\ \qibla]).
\columnbreak

Now, in the twentieth chapter, [the direction of] \Kashi\ is understood with \gls{degree_azimuth} (\textit{diś-aṃśa}) [measured] from both the \glsuseri{north_south_direction} (\textit{saumya-yāmya-diś}) in one's own city.
\end{multicols}
\medskip
%%%%%%%%%%%%%%%%%%%%%%%%%%%%%%%%%%%

\begin{multicols}{2}
\noindent\reversemarginpar\marginnote{\hypertarget{Pii21}{Discourse~II.21}}%
\tfarsi{باب بیست و یکم \\
در معرفت طالع از ارتفاع.%
\te{(f.\thinspace 20a:\thinspace31~SjB)}\hfill}
\columnbreak

\noindent\normalmarginpar\marginnote{\hypertarget{Sii21}{Part~II.21}}%
\tsans{atha ekavi.m"satime .adhyaaye .abhii.s.tonnataa.m\-"sebhyo\nobreak\te{\footnote{~\tsans{.abhii.s.tonnataa.m"sebhyo} {$\Big ]$} \tsans{.abhii.s.tonnavaa.m"sebhyo}\enskip Kh. The compound \tsans{.abhii.s.tonnavaa.m"sebhyo} in Kh can be segmented as \tsans{.abhii.s.ta}\,$_\text{\acrshort{modifier}}$ + \tsans{unnava}\,$_\text{?}$ + \tsans{a.m"sebhyo}\,$_\text{\acrshort{dative}/\acrshort{ablative}-\acrshort{plural}}$; however,  \tsans{unnava} is neither a valid morphophonemic compound nor a standard lexical entry. The word \tsans{unna\underline{ta}} is contextually relevant and also variously attested in this chapter, \eg \tsans{unnatajyaayaa.h} (f.\thinspace 27r:\thinspace27~Kh) or  \tsans{samunnatajyaa} (f.\thinspace 27v:\thinspace14~Kh). I suspect the scribe inattentively copied the glyph \tsans{vaa.m} for \tsans{taa.m} in the chapter-title.}} %
lagna{\dev{ज्ञा}}nam~||}\enskip 
(f.\thinspace 27r:\thinspace26--27~Kh)
\end{multicols}%
%
\begin{multicols}{2}
Twenty-first chapter\\
On the \gls{knowledge} (\marifat) of the \gls{ascendant} (\tali) from the \gls{elevation} (\irtifa). 
\columnbreak

Now, in the twenty-first chapter, the \gls{knowledge} (\textit{jñāna}) of the \gls{ascendant} (\textit{lagna}) from the \gls{desired_degree_elevation} (\textit{abhīṣṭa-unnata-aṃśa}).
\end{multicols}

\newpage %%%%________________________NEWPAGE

%%%%%%%%%%%%%%%%%%%%%%%%%%%%%%%%%%%

\begin{multicols}{2}
\noindent\reversemarginpar\marginnote{\hypertarget{Pii22}{Discourse~II.22}}%
\tfarsi{باب بیست و دوم \\
در معرفت ارتفاع یا انخفاض كواكب از طالع.%
\te{(f.\thinspace 20b:\thinspace25--24~SjB)}\hfill}
\columnbreak

\noindent\normalmarginpar\marginnote{\hypertarget{Sii22}{Part~II.22}}%
\tsnb{अथ द्वाविंशतिमे ऽध्याये खगस्य स्वोदयांशेभ्यो\te{\footnote{~\tsans{svodayaa.m"sebho} {$\Big ]$} \tsans{svodaya.m"sobhyo}\enskip Kh. A regular sandhi of the words \tsans{sva}\,$_\text{\acrshort{modifier}}$ + \tsans{udaya}\,$_\text{\acrshort{modifier}}$ + \tsans{a.m"sebhyo}\,$_\text{\acrshort{dative}/\acrshort{ablative}-\acrshort{plural}}$ generates
\tsans{svodayaa.m\underline{"se}bhyo} where the terminal consonant -\tsans{"sa} (of \tsans{a.m"sa})
changes to -\tsans{"se} (and not -\tsans{"so}) before the dative/ablative case ending -\tsans{bhya.h}. The \textit{o}-diacritic in \tsans{"so} appears to be a scribal hypercorrection.}} %
ऽभीष्टोन्नतांशानामधरांशकानां च ज्ञानम्~॥}%
\tsans{ etallak.sa.na.m puurvame.soktam~||}\enskip (f.\thinspace 28r:\thinspace16--17~Kh)
\end{multicols}%
%
\begin{multicols}{2}
Twenty-second chapter\\
On the \gls{knowledge} (\marifat) of the \gls{elevation} (\irtifa) or \gls{depression} (\inkhifad) of \glspl{celestial_object} (\kawakib) from the \gls{ascendant} (\tali). 
\columnbreak

Now, in the twenty-second chapter, the \gls{knowledge} (\textit{jñāna}) of the \gls{desired_degree_elevation} (\textit{abhīṣṭa-unnata-aṃśa}) and of the \gls{degrees_depression} (\textit{adhara-aṃśaka}) from the \gls{rising_location_degree} (\textit{sva-udaya-aṃśa}) of a \gls{celestial_object} (\textit{khaga}). The \gls{definition} (\textit{lakṣaṇa}) of this has already been declared in the first half [of Part~II].
\end{multicols}

\newpage %%%%________________________NEWPAGE

%%%%%%%%%%%%%%%%%%%%%%%%%%%%%%%%%%%

\begin{multicols}{2}\label{partII_colophon_sanskrit}
The \ZijiShahJahani: Discourse~II does not have a colophon. F~21r:\thinspace  21~\SjB\ ends with the last line of chapter twenty-two: \tfarsi{\dots بهین موامره همین مطلوب حاصل آید.} 
\columnbreak

\noindent\normalmarginpar\marginnote{\hypertarget{SiiCol}{Part~II\\\textit{colophon}}}%
\tsans{ya.h "srii"saahajahaa/\te{\footnote{~\tsans{"srii"saahajahaa/} {$\Big ]$} \tsans{"srii"saahahaa/}\enskip Kh. I suspect the scribe unwittingly left out the letter \tsans{ja} while copying; most other occurrences of \Shahjahan's name in the text read \tsans{"srii"saahajahaa/} (sometimes without the terminal nasal diacritic \textit{candrabindu}), \eg folia 3v:\thinspace1, 5r:\thinspace15, or 6v:\thinspace22 of Kh. Also, this verse is in the nineteen-syllabled \sardulavikridita\ meter which would require a five-syllabled word like \tsans{"srii"saahajahaa/\selip} for metrical completion (\textit{pada-pūrti}).}} %
n.rpaalamuku.taala"nkaara\-cuu.daama.nistasyaa{\dev{ज्ञा}}%
mavalambya dustaramamu.m siddhaanta\-sindhu.m taran~|| nityaananda iti dvijottamak.rpa.h\te{\footnote{~The appositive nominal compound \tsans{dvijottamak.rpa.h} is grammatically irregular. The terminal word \tsans{k.rpa.h}\,$_\text{\acrshort{nominative}-\acrshort{singular}}$ is attested as the sage Kṛpācārya (from the \textit{Mahābhārata}) in Sanskrit lexicons. In this case, however, I suspect \tsans{dvijottamak.rpa.h} is a metrical contraction (\textit{pada-anatireka-karaṇa}) that can be parsed as \tsans{dvijottamaa.naa.m k.rpaapaatra.h} `[the one] worthy of the mercy (\textit{kṛpā}) of the best \Brahmana s'. In the colophon of \Nityananda's \Sarvasiddhantaraja\ (1639), we find a related phrase \tsans{dvijaanaamaa{\tsnb{ज्ञा}}kaarii} %
 `one who executes the commands of the \Brahmana s' (\ie obedient of the authority of \Brahmana s) as an epithet of \Nityananda\ (\vid\ \cite[228]{PetersonCatalogue}; \cite[102]{DvivediGanakaTarangini}).}} %
"sriidevadattaatmajastripra"snapracuroktiyuktisahita.m kaa.n.da.m dvitiiya.m hyagaat~||}%
 \enskip (f.\thinspace 28v:\thinspace15--18~Kh)
\medskip

\Nityananda, who crosses over this unconquerable `Ocean of the \Siddhanta s [\scl composes the \Siddhantasindhu] [by] holding onto the command of Śri \Shahjahan\ who is the crest jewel of the ornamental crown of kings, [the man who is worthy of] the mercy of the best \Brahmana s, the son of Śri \Devadatta, has just finished the second part accompanied by many statements and rationales
on the \triprasna.\footnote{~\Vid\ \S~\ref{comparative_overview_chapters_zij_sindhu}, remark~\ref{triprasna_remark} on page~\pageref{triprasna_remark}.}
\end{multicols}

%%%%%%%%%%%%%%%%%%%%%%%%%%%%%%%%%%%
%%%%%%%%%%%%%%%%%%%%%%%%%%%%%%%%%%%



% In another place though, \Nityananda\ discusses Arabic and Persian orthography in Sanskrit. In the first section (\textit{prathama-prakāra}) of the prolegomenon (\textit{granthārambha}) of the \Siddhantasindhu\  (on f.\thinspace 8r:\thinspace 11--16~Kh) 
% \Nityananda\ goes into a full discussion, in Sanskrit, on  . According to him, the word \zij\ is well known in Persian (\textit{phārasī}) as \textit{jīga} but in the Arab world (\textit{āraba-deśa}) it becomes \textit{jīja} with linguistic corruption (\textit{apabhraṃśa}) brought on by the absence of the letter `\textit{ga}' (\textit{gakāra-abhāvāt}) in Arabic. People living in the Arab country then read the said word with the letter `\textit{ja}' instead of the letter `\textit{ga}'. I have not been able to locate this statement in the first section (\qism) of the prolegomenon (\muqaddima) of \MullaFarid's \ZijiShahJahani.}

% \tsans{makkaa} as the Sanskrit phonological calque of the Persian word \tfarsi{مكّه} (\makkah) `the city of Mecca'. This is made evident in several instances in this chapter, \eg \tsans{makkaapure} (\textit{makkāpure}) `in the city of Mecca' (f.\thinspace 27r:\thinspace 13~Kh) or (\textit{makkānagaraṃ}) `the city of Mecca' (f.\thinspace 27r:\thinspace 21~Kh).  
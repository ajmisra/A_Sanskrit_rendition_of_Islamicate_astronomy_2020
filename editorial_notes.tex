\renewcommand{\thefootnote}{\arabic{footnote}} % Hindu-Arabic footnote numbering

I discuss below the orthographic standards and transcription/transliteration schemes I have adopted in editing the Persian text (from \SjA\ \& \SjB) and the Sanskrit text (from Kh) in \S\S~\ref{persian_orthography}--\ref{transcription_transliteration_schemes}. A description of the typographic conventions in \S\S\ \ref{chapter_title_comparision_persian_sanskrit}, \ref{zijshahjahan_persian_english}, and \ref{siddhantasindhu_sanskrit_english}  of this study follows that in \S~\ref{typographic_conventions}. Therein, I explain with examples the various symbols and abbreviations used in the critical footnotes of these sections. Lastly, \S~\ref{glossary_appendix_format} describes the format of the Glossary (on page~\pageref{main}) and the Appendix (on page~\pageref{Appendix_verbs}).

%%%%%%%%%%%%%%%%%%%%%%%%%%%%%%%%%%%%%%%%%%%%%%%%%%%%%%%%%%%%%%%%%%%%%%%%%%%%%%%%
\subsection{Remarks on Persian orthography} \label{persian_orthography}
%%%%%%%%%%%%%%%%%%%%%%%%%%%%%%%%%%%%%%%%%%%%%%%%%%%%%%%%%%%%%%%%%%%%%%%%%%%%%%%%
The Persian text presented in this study follows the orthography of Classical Persian in which the manuscripts were written. It does, however, use modern punctuation marks for added clarity. 

Arabic loanwords are transcribed with their original spelling when they are attested as such, \eg the Arabic letter \tfarsi{ـي} (\textit{yā}\Ayn) is retained in the spelling of the word \tfarsi{ثاني} (\thani) following \SjA\ (f.\thinspace 21b:\thinspace21). In other instances, Persian spellings are used, \eg \tfarsi{درجه} (\daraji) instead of the Arabic \tfarsi{درجة} with the Arabic letter \tfarsi{ـة} (\textit{tā}\Ayn \textit{marbūṭah}). Persian words are presented in their unligated forms: \eg \tfarsi{آنرا} on f.21b:\thinspace 23~\SjA\ is transcribed as \tfarsi{آن را}. 

A few minor orthographic irregularities are seen in the Persian text of \SjA\ and \SjB. I note these below and emend them silently. However, when the reading of the text is affected by a grammatical ambiguity, I discuss my interpretation in corresponding footnotes. Scribal alterations, cancellations, copying errors, and marginalia are also noted in the footnotes.
\begin{enumerate}[topsep=0pt]
    \item Vocalisation marks are often omitted, in particular, on the syllable-initial \tfarsi{آ} (\textit{alif madda}). For example, \tfarsi{آن} (\textit{ān}) is simply written as \tfarsi{ان} on f.\thinspace 22a:\thinspace 2~\SjA\ and f.\thinspace 16b:\thinspace 1~\SjB.
    
    \item Arabic loan words are generally written without any diacritics. For example, the words \tfarsi{معدّل}  (\muaddil) (on f.\thinspace 21b:\thinspace 21~\SjA) and \tfarsi{اوّل} (\avval) (on f.\thinspace 14a:\thinspace 1~\SjB) are written without the over-letter diacritic \tfarsi{{\msf{◌}}\char"0651} (\textit{shadda}/\textit{tashdīd}).
    
    \SjA\ sometimes uses supplementary diacritics (\tashkil) to differentiate homographic words, \eg on f.\thinspace 22a:\thinspace8, the word \tfarsi{دور}, understood as `distant/remote', is explicitly written as \tfarsi{دُور}  (\textit{dūr}, \textsc{ipa} /duːɾ/) with the over-letter diacritic \tfarsi{{\msf{◌}}\char"064F} (\textit{ḍamma}/\textit{pish}) to differentiate it from \tfarsi{دَور} (\textit{dawr}, \textsc{ipa} /dawɾ/) `cycle/revolution'.   
    
    \item \SjA\ also has occasional diacritic points that are misplaced, for instance, the word \tfarsi{جهت} (\jahat) is spelt as \tfarsi{ـهـت}\.{\tfarsi{جـ}} on f.\thinspace 21b:\thinspace 23. The overdot over the word-initial letter \tfarsi {جـ} (\textit{jim}) is meaningless. 
    
    \item \SjB\ sometimes lacks diacritic points to indicate the phonetic distinction of consonants (\ijam). For example, the word-initial letter \tfarsi{بـ} (\textit{be}) in \tfarsi{بر} (\textit{bar}) appears without the underdot (on f.\thinspace 16a:\thinspace 30~\SjB). 
\end{enumerate}

%%%%%%%%%%%%%%%%%%%%%%%%%%%%%%%%%%%%%%%%%%%%%%%%%%%%%%%%%%%%%%%%%%%%%%%%%%%%%%%%
\subsection{Remarks on Sanskrit orthography}\label{sanskrit_orthography} 
%%%%%%%%%%%%%%%%%%%%%%%%%%%%%%%%%%%%%%%%%%%%%%%%%%%%%%%%%%%%%%%%%%%%%%%%%%%%%%%%
The Sanskrit text of Kh is fairly regular with occasional orthographic irregularities. Most of these are common scribal oversights seen in \Nagari\ palaeography, and hence, I emend these silently. They include 
\begin{itemize}[topsep=0pt]
    \item using the over-letter diacritic \tsans{\char"0902} (\textit{anusvāra}) for all conjoined nasal consonants;
    \item omitting the diacritical marks \tsans{\char"0903} (\textit{visarga}) for the terminal aspirate, \tsans{्} (\textit{virāma}) for inherent-vowel suppression, and \tsans{ऽ} (\textit{avagraha}) for prodelision of \textit{a/ā};
    \item using/omitting punctuation marks like the \tsans{||} (\textit{double-daṇḍa}) irregularly;
    \item using irregular (vernacular?) \Nagari\ letters (\eg {\tsv{\char"95F}} for \tsans{ya}) for Sanskrit, and retaining ill-formed vocalic signs (\eg \tsans{di\tsv{\char"940}});
    \item using doubled consonant irregularly, \eg  \tsans{ddha} in \tsans{arddha}  (after a vowel-suppressed \textit{r}-consonant) or across line (\textit{pāda}) breaks in a stanza;
    \item reversing conjunct-consonant pairs (\eg \tsans{dhba} for \tsans{bdha} or \tsans{nha} for \tsans{hna}); and
    \item confusing consonants like \tsans{ba} and \tsans{va}, \tsans{pa} and \tsans{ya}, \tsans{ma} and \tsans{sa}, \tsans{.sa} and \tsans{kha}, \etcp, and certain ligatures like \tsans{.s.ta} for \tsans{.s.tha}, \tsans{kta} for \tsans{tkta}, \etc
\end{itemize}

However, I discuss in footnotes the structure of those orthographic irregularities (mainly, morphosyntactic errors) that affect the reading of the text even as I emend the words accordingly. Other scribal errors like haplography (inadvertent omission) and dittography (inadvertent repetition) are also described in footnotes. 

%%%%%%%%%%%%%%%%%%%%%%%%%%%%%%%%%%%%%%%%%%%%%%%%%%%%%%%%%%%%%%%%%%%%%%%%%%%%%%%%
\subsection{Transcription and transliteration schemes}\label{transcription_transliteration_schemes}
%%%%%%%%%%%%%%%%%%%%%%%%%%%%%%%%%%%%%%%%%%%%%%%%%%%%%%%%%%%%%%%%%%%%%%%%%%%%%%%%
I adopt the following transcription/transliteration schemes to render Arabic, Persian, Sanskrit, and Hindi characters into the Roman (Latin) script.     
\begin{itemize}[topsep=0pt]
    \item Arabic and Persian text are transcribed with the EI3 standard of phonetic transcription in Brill's \textit{Encyclopedia of Islam}, third edition \parencite[]{EIslam}.
    \item Sanskrit and Hindi text is transliterated following the International Alphabet of Sanskrit Transliteration (IAST) scheme. For vernacular Hindi words, as well as \Devanagari-spellings of transliterated Persian words, I use the International Organisation for Standardisation (ISO) 15919 extension to transliterate certain characters, \eg {\tsnb{ड़ी}} is rendered as \textit{ṛī}, \tsans{khaa/} as \textit{khām̐}, \etc Commonly attested words of Indian origin (\eg Hindu, Brahmin, Mughal, Varanasi, \etcp) are presented without diacritics.
\end{itemize}

%%%%%%%%%%%%%%%%%%%%%%%%%%%%%%%%%%%%%%%%%%%%%%%%%%%%%%%%%%%%%%%%%%%%%%%%%%%%%%%%
\subsection{Typographic conventions} \label{typographic_conventions} 
%%%%%%%%%%%%%%%%%%%%%%%%%%%%%%%%%%%%%%%%%%%%%%%%%%%%%%%%%%%%%%%%%%%%%%%%%%%%%%%%
\subsubsection{Chapter-titles in \S~\ref{chapter_title_comparision_persian_sanskrit}}\label{chapter_titles_in_zij_sindhu}

\begin{enumerate}[topsep=0pt]
    \item \textbf{Layout}\quad The chapter-titles from Discourse II (\maqala\idafaconsonant\ \duvum) of the \ZijiShahJahani\ and Part II (\dvitiya\ \kanda) of the \Siddhantasindhu\ are placed in parallel columns in \S~\ref{chapter_title_comparision_persian_sanskrit}. The Persian title-text (to the left) and the Sanskrit title-text (to the right) can be identified by their corresponding chapter numbers in the left and right margins respectively, \eg `Discourse~II.2' (in the left margin) and `Book~II.2' (in the right margin) corresponding to the second chapter at the top of page~\pageref{chapter_number_example}. 
    I include the folia and line numbers of the 
    manuscripts at the end of the text in parentheses. 
    
    \item \textbf{Format of the translations}\quad My English translations of the chapter-titles are placed right below the corresponding Persian and Sanskrit text, parallel to each other. The technical terms in the translations are typeset in bold, and are accompanied by a Roman transcription/transliteration (in parentheses) of corresponding Persian and Sanskrit expressions. I indicate any additional words/expressions supplied for grammatical clarity by enclosing them in square brackets `[\,]' in my translations.
\end{enumerate}

\subsubsection{Chapter VI in \S\S~\ref{zijshahjahan_persian_english} \& \ref{siddhantasindhu_sanskrit_english}}\label{chapter_vi_zij_sindhu}

\begin{enumerate}[topsep=0pt]
     \item \textbf{Layout}\quad The Persian and Sanskrit text of the sixth chapter from Discourse II of the \ZijiShahJahani\ and from Part II of the \Siddhantasindhu\ are presented in \S~\ref{zijshahjahan_persian_english} and \S~\ref{siddhantasindhu_sanskrit_english} respectively. In both these sections, I place corresponding passages from the original text and their English translations on successive pages; \vid\ <INSERT PAGE REFERENCE NOTE> for my division of the text into comparable passages.
    \item \textbf{Ordering of the passages}\quad The passage-markers are enclosed in square brackets, \eg `[2]' or `[α]', and placed at the beginning of the passages. They appear in the right margin for the Persian text, and in the left margin for the Sanskrit text and English translations. 
    \item \textbf{Roman transcription/transliteration}\quad The technical expressions in the English translations are typeset in bold, and are accompanied by a Roman transcription/transliteration (in parentheses) of corresponding Persian and Sanskrit expressions. The Roman transcriptions of Persian compound verbs that indicate arithmetic operations are indicated in their infinitive form and are prefixed with an asterisk. For example, \gls{sum} (*\jam\ \kardan) `to sum' in passage~[\hyperlink{PEpass1}{1}] on page~\pageref{passage_1_english_persian}. \Vid\ footnote~\ref{compound_action_verb_persian} in the Appendix (on page~\pageref{Appendix_verbs}).
    \item \textbf{Verse numbering in \S~\ref{siddhantasindhu_sanskrit_english}}\quad The numbering of the metrical Sanskrit verses is \textit{different} from the ordering of the passages. I follow the Sanskrit text in placing the verse numbers at the end of the stanzas in my translations. For instance, the verse-number `\tsans{|| 1 ||}' at the end of a stanza in passage [\hyperlink{SpassA}{α}] (on page~\pageref{verse_1_label_sans_example}) also appears as `1' at the end of my English translation (on page~\pageref{verse_1_label_eng_example}).  
    \item \textbf{Poetic meters in \S~\ref{siddhantasindhu_sanskrit_english}}\quad The Sanskrit names of the poetic meters are indicated (in Roman transliteration) in the right margin alongside the Sanskrit verses, \eg verse number `\tsans{|| 2 ||}' (in passage~[\hyperlink{Spass2}{2}]) is in the \textit{pramāṇikā}-meter indicated in the right margin margin (in-line with the verse) on page~\pageref{sanskrit_meter_typography_example}.
    \item \textbf{Folio breaks}
    \begin{itemize}[topsep=0pt]
        \item For the Persian text in \S~\ref{zijshahjahan_persian_english}, I indicate a folio break with `$\rceil$' (in-line with the text) and state the manuscript reference in the left margin. For instance, passage [\hyperlink{Ppass2}{2}] on page~\pageref{folio_break_persian_example} has `\tfarsi{\dots جيب~\te{$\rceil$}بعد\dots}' on its second line and corresponding to it, `f.\thinspace 22a:\thinspace 1~\SjA~$\rceil$' is written in the left margin. This indicates the words `\tfarsi{بعد\dots}' begin on line 1 of f.\thinspace 22a in \SjA.
        \item For the Sanskrit text in \S~\ref{siddhantasindhu_sanskrit_english}, folio breaks are shown with `$\lceil$' (in-line with the text) and manuscript references are in the right margin. For instance, the first line of passage [\hyperlink{SpassB}{β}] on page~\pageref{folio_break_sanskrit_example} has `\dots\tsans{vivara\raisebox{.5ex}{$\lceil$}ga.m dhanu}\dots', and correspondingly, `$\lceil$~ f.\thinspace 20v:\thinspace1~Kh' is in the right margin. This implies the words `\tsans{ga.m dhanu}\dots' begin on line 1 of f.20v in Kh. 
    \end{itemize}
\end{enumerate}

\subsubsection{Critical footnotes in \S\S~\ref{chapter_title_comparision_persian_sanskrit}, \ref{zijshahjahan_persian_english}, \& \ref{siddhantasindhu_sanskrit_english}}\label{critical_footnotes_in_sec_2_3_4}

\begin{enumerate}[topsep=0pt]
    \item Footnotes in \S~\ref{chapter_title_comparision_persian_sanskrit} are numbered 1, 2, 3 \etcp, whereas, those in \S\S~\ref{zijshahjahan_persian_english} \& \ref{siddhantasindhu_sanskrit_english} are numbered [i], [ii], [iii] \etc The numbers are reset at the beginning of each section. 
    \item I use repeated footnote marks for longer footnotes in the Persian text (in \S~\ref{zijshahjahan_persian_english}), \eg footnote~`\ref{wrap_footnote_example}' on page~\pageref{wrap_footnote_example} where `[vi]\==' and `\==[vi]' enclose the commented text in passage [\hyperlink{Ppass9}{9}].   
    \item An edited reading is separated from a variant (attested) reading by a right-square bracket {$\Big]$} like, for instance, \tsans{naama madhyaahnarekheti} {$\Big ]$} \tsans{naama dhyaahnarekheti}~Kh (footnote~\ref{emeneded_attested_sanskrit_example} on page~\pageref{emeneded_attested_sanskrit_example}) or \tfarsi{یا در}~{$\Big ]$}~\tfarsi{یاد در}~\SjB\ (footnote~\ref{emeneded_attested_persian_example} on page~\pageref{emeneded_attested_persian_example}).
    \item I use abbreviated forms of grammatical terms in the critical footnotes. For example, the \acrfull{modifier} \tsans{prathama} in the nominal compound \tsans{prathamaadhyaaye} is indicated as \tsans{prathama}\,$_\text{\acrshort{modifier}}$ in footnote~\ref{grammatical_abbreviation_example} on page~\pageref{grammatical_abbreviation_example}. The list of grammatical abbreviations used in this study can be seen on page~\pageref{acronym}.  
    \item The truncation (of letters) in long Sanskrit words is indicated by the \Nagari\ abbreviation symbol `\tsans{\selip}' (\textit{lāghava-cihna}), comparable in its use to an ellipsis. For example, \tsans{ga.nita\selip} for \tsans{ga.nitasau\-karyaartha.m}  or \tsans{\selip maa.m"saakhya} for \tsans{sphu.taapamaa.m"saakhya}. 
    \item I emphasise the letters in Persian and Sanskrit words by underlying them, \eg \tfarsi{\underline{قسمت}} or \tsans{unna\underline{ta}}. The emphasis is used in the critical footnotes to signify an orthographic feature or a scribal error that is overt.
\end{enumerate}

%%%%%%%%%%%%%%%%%%%%%%%%%%%%%%%%%%%%%%%%%%%%%%%%%%%%%%%%%%%%%%%%%%%%%%%%%%%%%%%%
\subsection{Format of the appendix and the glossary} \label{glossary_appendix_format} 
%%%%%%%%%%%%%%%%%%%%%%%%%%%%%%%%%%%%%%%%%%%%%%%%%%%%%%%%%%%%%%%%%%%%%%%%%%%%%%%%
All Persian and Sanskrit words in the appendix and the glossary are written with Perso-Arabic and \Nagari\ letters respectively, and are accompanied by corresponding Roman transcriptions/transliterations enclosed in parentheses. The grammatical terms are abbreviated in the appendix and the glossary; their expanded forms are listed on page~\pageref{acronym}. 

\subsubsection*{Appendix}\label{format_appendix}
The appendix includes a list of the Persian and Sanskrit verbs seen in the \ZijiShahJahani\ Discourse~II.6 and the \Siddhantasindhu\ Part~II.6 respectively. 
\begin{itemize}[topsep=0pt]
\item The Persian verbs are listed on page~\pageref{persian_verbs} in their infinitive form along with their corresponding present stems. I group together the attested (inflected) forms of these verbs and provide \textit{passage-markers} (in square brackets at the end) to locate them in the text in \S~\ref{zijshahjahan_persian_english}. For example,
{\par\centering\tfarsi{باشد} (\textit{bāshad})  \acrshort{present}-\acrshort{subjunctive}-\acrshort{singular}·\acrshort{third} `[he/she/it] should be' [\hyperlink{Ppass1}{1}, \hyperlink{Ppass3}{3}--\hyperlink{Ppass8}{8}, \hyperlink{Ppass10}{10}, \hyperlink{Ppass11}{11}].\par}
\item The Sanskrit verbs are indicated in their root-form beginning on page~\pageref{sanskrit_verbs}. The verbal roots (of different verb-class numbers) are grouped together based on their common meaning. The attested (inflected forms) are listed under their respective verbal root, and are accompanied by \textit{passage-markers} (in square brackets at the end) to locate them in the text in \S~\ref{siddhantasindhu_sanskrit_english}. For example,
{\par\centering
\tsans{syaat} (\textit{syāt}) \acrshort{optative}-\acrshort{active}-\acrshort{singular}·\acrshort{third} `[he/she/it] should be/exist' [\hyperlink{Spass3}{3}--\hyperlink{Spass7}{7}].
\par}
\end{itemize}

\subsubsection*{Glossary} \label{format_glossary}
Persian and Sanskrit technical expressions are listed in the glossary (beginning on page~\pageref{acronym}). They are derived from  \S\S~\ref{chapter_title_comparision_persian_sanskrit}, \ref{zijshahjahan_persian_english}, and \ref{siddhantasindhu_sanskrit_english} where they appear in bold in corresponding English translations. 
\begin{itemize}[topsep=0pt]
    \item Equivalent Persian and Sanskrit terms are grouped together under their common technical translation in English, separated from each other by a semicolon. Synonyms are separated by commas. For example, 
    {\par\centering\textbf{latitude}\quad \tfarsi{عرض}  (\ard) [\hyperlink{PEpass5}{5}, \hyperlink{PEpass6}{6}, \hyperlink{PEpass10}{10}]; \tsans{"sara} (\textit{śara}) [\hyperlink{SEpass6}{6}], \tsans{baa.na} (\textit{bāṇa}) [\hyperlink{SEpass10}{10}].\par}
    \item At the end of each entry, I provide the \textit{chapter-numbers} and/or the \textit{passage-markers} in square brackets to identify its location in the text. The identifiers refer to apposite passages or chapter-titles corresponding to the language of the entry. For instance, in the example above, \tfarsi{عرض}  (\ard) appears in passage~[\hyperlink{PEpass6}{6}] of \S~\ref{zijshahjahan_persian_english} (page~\pageref{passge_6_ard_glossary_format_example}); whereas, \tsans{"sara} (\textit{śara}) can be found in passage~[\hyperlink{SEpass6}{6}] of \S~\ref{siddhantasindhu_sanskrit_english} (page~\pageref{passge_6_sara_glossary_format_example}).     \item References to multiple chapter-numbers are separated by commas (without repeating `II'). For example,
    {\par\centering
    \textbf{ascendant}\quad \tfarsi{طالع} (\tali), pl.\thinspace \tfarsi{طوالع} (\tawali) \hyperlink{Pii11}{II.11}, \hyperlink{Pii21}{21}, \hyperlink{Pii22}{22}; \tsans{lagna} (\textit{lagna}) \hyperlink{Sii21}{II.21}\par}
    indicates that \tfarsi{طوالع} (\tawali) appears in the chapter-titles of the Persian chapters \hyperlink{Pii11}{II.11}, \hyperlink{Pii21}{II.21}, and \hyperlink{Pii22}{II.22} in \S~\ref{chapter_title_comparision_persian_sanskrit}. 
    \item Successive chapter-numbers or passage-markers are sometimes shown as a range to space space, \eg\enspace \textbf{definition}\quad \tsans{lak.sa.na} (\textit{lakṣaṇa}) \hyperlink{Sii2}{II.2}, \hyperlink{Sii8}{8}--\hyperlink{Sii13}{13}, \hyperlink{Sii17}{17}--\hyperlink{Sii19}{19}, \hyperlink{Sii22}{22}. 
    \item Mutually related technical translations in English are grouped together based on their linguistic or mathematical similarity. For instance, the heading \gls{distance_ce_true_declination} (on page~\pageref{glsentry-distance_ce_true_declination}) includes the expressions $\hookrightarrow$\,\gls{distance}, $\hookrightarrow$\,\gls{distance_celestial_object}, and 
    $\hookrightarrow$\,\gls{true_declination}. 
    \item The glossary entries are arranged following the English alphabetical order.
\end{itemize}



{\hindifont\selectfont
\OnehalfSpacing
\large
\setcounter{footnote}{0}
\renewcommand{\thefootnote}{[\roman{footnote}]} % for distinct numbering of Sanskrit edition (chapter vi) footnotes (starting at [i])
\setlength{\footnotesep}{\baselineskip} % for space between footnotes
\normalmarginpar % for direction of margin notes

\te{\marginnote{\raisebox{1ex}{$\ulcorner$}\kern-0.25ex\footnotesize f.~20r:\thinspace16~Kh}}{\raisebox{1.5ex}{$\ulcorner$}\kern-0.5ex}{\bfseries || atha .sa.s.thaadhyaayaspa.s.takraanti{\dev{j~naa}}nam\te{\footnote{\,Looking at the orthography of the other chapter-titles (in Part~II of Kh, \vid\ \S~\ref{chapter_title_comparision_persian_sanskrit}), 
\tsans{.sa.s.thaadhye spa.s.takraanti{\tsnb{ज्ञा}}nam} %
would be more consistent than \tsans{.sa.s.thaadhyaayaspa.s.takraanti{\tsnb{ज्ञा}}nam}.
Nevertheless, I maintain the reading attested in Kh in the absence of a second manuscript witness. Besides, the locative sense (\textit{adhi\-karaṇa}) of the modifier \tsans{.sa.s.tha-[.a]dhyaaya} in the \textit{tatpuruṣa} compound \tsans{.sa.s.tha-[.a]dhyaaya-spa.s.ta-kraanti-{\tsnb{ज्ञा}}nam}\,$_\text{\acrshort{nominative}-\acrshort{singular}}$ is identical to the use of the
prepositional phrase \tsans{.sa.s.tha-[.a]dhyaaye}\,$_\text{\acrshort{locative}-\acrshort{singular}}$ in 
the sentence \tsans{.sa.s.tha-[.a]dhye spa.s.ta-kraanti-{\tsnb{ज्ञा}}nam}\,$_\text{\acrshort{nominative}-\acrshort{singular}}$.}}||}
\bigskip

khagasya baa.no.anyataraapama.h\te{\footnote{\,\tsans{baa.no.anyatarapama.h} {$\normalsize ]$} \tsans{baa.nenyatarapama.h}\enskip Kh. The conjoined word \tsans{baa.nenyatarapama.h} in Kh can be meaninfully segemnted as \tsans{baa.ne}\,$_\text{\acrshort{locative}-\acrshort{singular}}$  + \tsans{anyatara-apama.h}\,$_\text{\acrshort{nominative}-\acrshort{singular}}$. However, this reading (the second noun \textit{in} the first) is contextually and semantically incoherent. The syntactic structure of the Sanskrit text mimics the syntax of the Persian text in passage~(1), \vid\ discussions on <INSERT PAGE \& SECTION REF>. This suggests that the emendation \tsans{baa.no}\,$_\text{\acrshort{nominative}-\acrshort{singular}}$ is better suited than \tsans{baa.ne}\,$_\text{\acrshort{locative}-\acrshort{singular}}$ as it agrees with \tsans{apama.h}\,$_\text{\acrshort{nominative}-\acrshort{singular}}$ in the subject-fronted noun phrase \tsans{khagasya-baana.h-.anyatara-apama.h puna.h}.}} puna-\\ ryadaa dvaya.m vaikadi"si sthita.m bhavet || \\
tadaa tayo.h sa.myutiranyathaantara.m\\ sphu.taapamaa.m"saakaakhya\te{\footnote{\,\tsans{\selip maa.m"saakhya} {\normalsize ]} \tsans{\selip maa.m"saakhyakhya}\enskip Kh, dittography of the second \tsans{khya}.}}
ihocyate svadik || 1 || \marginnote{\te{(1)}} 

sphu.taapamaa"nkasi{\dev{~nji}}nii sabhatrayadyujiivayaa || \\
nihanyate .adhariik.rtaa sphu.taapamajyakaa bhavet || 2 || \marginnote{\te{(2)}}%
\bigskip

paramakraantiko.tijyaa sphu.takraantya"nkajiivayaa || \\
hataanya\-kraantiko.tijyaaptaa syaatspa.s.taapamajyakaa || 3 || \marginnote{\te{(3)}}

ki.mvaa paramakraantijyaako.s.thikebhya.h\te{\footnote{\,An apposite reading of the compound \tsans{parama-kraanti-jyaa-ko.s.thikebhya.h} should be \tsans{parama-kraanti-}{\normalsize <}\tsans{ko.ti}{\normalsize >}\tsans{jyaa-ko.s.thakebhya.h}. This would agree with an identical construction in passage~(6) construed in the same mathematical context. There is, however, no visible evidence (\eg interlinear lacunae or scribal corrections) on f.~20r:\thinspace 20~Kh to suggest an omission. Therefore, I leave the attested reading unaltered in Sanskrit but include an emendation in my English translation.}} sphu.takraantya"nkajyayaa gu.nita-\\phalamutthaaya dvitiiyakraantiko.tijyayaa bhajellabdha.m spa.s.takraantijyaa syaat || \marginnote{\te{(4)}}

atha ca yadi khagasya  ba.no na syaattadaa \\
tasya kraantireva spa.s.takraantirbhavet || \marginnote{\te{(5)}} 
\newpage

atha kraantiryadi na syaatpuna.h "saro bhavettadaa baa.najyaa \\ paramakraantiko.tijyayaa sa.mgu.nyaadha.h kuryaat || \\
ki.mvaa paramakraantiko.tijyaako.s.thakebhyo baa.najyayaa \\ gu.nitaphalamutthaapayetspa.s.takraantijyaa baa.nadigbhavet || \marginnote{\te{(6)}}

yadi khagasya kraanti.h paramakraantitulyaa syaattadaa\\ spa.s.takraantya"nka eva sphu.takraantirbhavati || \marginnote{\te{(7)}}
\bigskip

{\bfseries || atha prakaaraantare.na ||}
\medskip

kadambavi.savadhruvadvayamupaiti\te{\footnote{\,The word \tsans{vi.sava} (as a part of a compound) appears several times in Kh. I suspect this is an irregular (vernacular?) variant of the word \tsans{vi.suva}/\tsans{vi.suvat} that denotes the `equinox/equinoctial point' in Sanskrit astronomical literature. \Vid\ \textcite[p.~4934a]{VacaspatyamVI} for the etymology of the word \tsans{vi.suva} (\textit{upapada tatpuṛusa})  or \tsans{vi.suvat} (\textit{matvarthīya taddhitavṛtti} or secondary nominal derivative from \tsans{vi.su}\,$_\text{\acrshort{indeclinable}}$ `in both directions'). In a lager \textit{tatpuruṣa} compound, \tsans{vi.suva}/\tsans{vi.suvat} refers to the equatorial reference frame, \eg in the genitive-compounds \tsans{vi.suva-v.rtta} `circle of the equinox' (\ie the celestial equator) and \tsans{vi.suvat-dhruva} `pole of the equinox' (\ie the celestial pole). The word \tsans{vi.sava} is not an attested form in any Sanskrit lexicon; however, it is consistently and frequently used throughout Kh. Therefore, I maintain \tsans{vi.sava}\,$_\text{\acrshort{irregular}}$ (as attested in Kh) in the \Nagari\ text but transliterate it using \tsans{vi.suva}\,$_\text{\acrshort{regular}}$ in my English translations. Both \tsans{vi.suva} and \tsans{vi.sava} have the same metrical signature ({\msf{\char"23D1}} {\msf{\char"23D1}} {\msf{\char"23D1}}).\label{vishava_deviant}}%
%
\footnote{\,\tsans{\selip dvayamupaiti} {\normalsize ]} \tsans{\selip dvayamupeti}\enskip Kh. In the (emended) conjoined word \tsans{\selip dvayam-upaiti}, the terminal verb  
\tsans{upaiti}\,$_\text{\acrshort{present}-\acrshort{indicative}-\acrshort{singular}·\acrshort{third}}$  is derived from \tsans{upe}\,$_\text{\acrshort{compound}-\acrshort{verb}}$. A regular sandhi of the words \tsans{upa}\,$_\text{\acrshort{preverb}}$ (indeclinable \textit{upasarga}) + \tsans{eti}\,$_\text{\acrshort{present}-\acrshort{indicative}-\acrshort{singular}·\acrshort{third}}$ produces \tsans{u\underline{pai}ti} and not \tsans{upeti} (\Panini's \Astadhyayi: 6.1.89). The conjugated form \tsans{eti} derives from either √\tsans{i}\,$_{\text{\acrshort{class}\textsubscript{2}}}$ or its intensified version √\tsans{e}\,$_{\text{\acrshort{class}\textsubscript{2}}}$ (\tsans{aa}\,$_\text{\acrshort{preverb}}$ + √\tsans{i}\,$_{\text{\acrshort{class}\textsubscript{2}}}$). Any interpretation of \tsans{upeti} (in Kh), \eg  \tsans{upeti}\,$_\text{\acrshort{noun}}$, is morphosyntactically inadmissible here. The choice of using the \textit{guṇa} vowel (\textit{e}-diacritic in \tsans{pe}) instead of the \textit{vṛddhi} vowel (\textit{ai}-diacritic in \tsans{pai}) is either a grammatical error or a scribal mistake.}} %
%
v.rtta.m ca yat \\ tadaayanamudiirita.m\te{\footnote{\,\tsans{\selip yanamudiirita.m} {\normalsize ]} \tsans{\selip yanamudiiriita.m}\enskip Kh. In the (emended) conjoined word \tsans{\selip [.a.a]yanam-udiiritam}, the terminal compound verb \tsans{ud}-√\tsans{iir}\,$_\text{\acrshort{class}\textsubscript{2}}$ takes the affix -\tsans{ita} to form \tsans{\selip udii\underline{ri}ta.m}\,$_\text{\acrshort{causative}-\acrshort{past}-\acrshort{passive}-\acrshort{participle}}$ (used as an adjective). The word \tsans{udiiriitam} (in Kh) is grammatically ill-formed; I suspect the \textit{ī}-diacritic in \tsans{rii} is a scribal hypercorrection.}} %
%
dhruvacatu.skayaata.m tathaa || \\ nabhogavi.savadvayopari\te{\footnote{\,\tsans{\selip vi.sava\selip}\,$_\text{\acrshort{irregular}}$  is identical to \tsans{\selip  vi.suva\selip}\,$_\text{\acrshort{regular}}$ or \tsans{\selip vi.suvat\selip}\,$_\text{\acrshort{regular}}$, \vid\ footnote~\ref{vishava_deviant}.}} %
%
patatsuv.rtta.m\te{\footnote{\,\tsans{patatsuv.rtta.m} {\normalsize ]} \tsans{tatsuv.rtta.m}\enskip Kh. The third quarter \tsans{nabhoga\selip ...\selip ca yad} of the verse in passage~(α) is metrically short by one syllable (hypometric): the verse otherwise follows a regular \textit{atyaṣṭi samavṛtta} meter called \textit{pṛthvī} with seventeen-syllables per quarter. The context of the verse, and its repetition in \Nityananda's \Sarvasiddhantaraja\ (\spastakrantyadhikara:\,verse~4, \textcite{MisraTD}), suggest \tsans{tatsuv.rtta.m} should be \tsans{patatsuv.rtta.m}.}} %
%
ca yad \\
bhacakrasad.r"saahvaya.m taditi kalpayedgolavit || 1 || \marginnote{\te{(α)}}


vi.savav.rttabhav.rttasad.r"sayorvivara{\raisebox{1.5ex}{$\ulcorner$}\kern-0.9ex}ga.m\te{\footnote{\,\tsans{vi.savav.rtta\selip}\,$_\text{\acrshort{irregular}}$ is identical to \tsans{vi.suvav.rtta\selip}\,$_\text{\acrshort{regular}}$, \vid\ footnote~\ref{vishava_deviant}.}} %
%
dhanuraayanav.rttajam || \te{\marginnote{\raisebox{1ex}{$\ulcorner$}\kern-0.25ex\footnotesize f.~20v:\thinspace1~Kh}}\\
bhavati yatkathita.h sa parasphu.taapama iti dyucarasya ca samprati || 2 || \marginnote{\te{(β)}}

bhavanacakrabhacakrasad.rk.sayorvivaraga.m dhanuraayanav.rttajam ||\\
bhavati yatsa pare.surihodito vi.savapaatayuge\te{\footnote{\,\tsans{vi.sava\selip}\,$_\text{\acrshort{irregular}}$ is identical to \tsans{vi.suva\selip}\,$_\text{\acrshort{regular}}$, \vid\ footnote~\ref{vishava_deviant}.}} %
%
sati kalpite || 3 || \marginnote{\te{(γ)}}

vi.savannabhogamadhye\te{\footnote{\,The word \tsans{vi.savat} is an attested secondary nominal derivative (from \tsans{vi.sa}\,$_\text{\acrshort{noun}}$ `poison') meaning `poisonous'. However, in the \textit{tatpuruṣa} compound \tsans{vi.savat-nabhoga-madhye}, I believe \tsans{vi.savat\selip}\,$_\text{\acrshort{irregular}}$, like \tsans{vi.sava}\,$_\text{\acrshort{irregular}}$, is identical to \tsans{\selip  vi.suvat\selip}\,$_\text{\acrshort{regular}}$, \vid\ footnote~\ref{vishava_deviant}.}} %
%
yatkoda.n.da.m bhav.rttasad.r"sasya ||\\
{\dev{j~ne}}ya.h sad.rgbhujo .asau bhaayanavivare sad.rkko.ti.h || 4 || \marginnote{\te{(δ)}}%
\bigskip

khagasya ko.tisi{\dev{~nji}}nii svabaa.nako.tijiivayaa ||\\
hataa .adhariik.rtaa\te{\footnote{\,\tsans{hataa .adhariik.rtaa} {\normalsize ]} \tsans{hataadhariinvataa}\enskip Kh. The conjoined words \tsans{hataadhariinvataa} in Kh can be segmented as 
\tsans{hataa}\,$_\text{\acrshort{past}-\acrshort{passive}-\acrshort{participle}}$ (from √\thinspace\tsans{han}\,$_{\text{\acrshort{class}\textsubscript{2}}}$) +
\tsans{.adhariinvataa}; however, the compound \tsans{.adhariinvataa} is etymologically defective by the rules of Pāṇinian grammar. The adverbial \textit{CvI}-suffixation to \tsans{adhara}\,$_\text{\acrshort{noun}-\acrshort{stem}}$ (making it \tsans{adharii}\,$_\text{\acrshort{preverb}}$) can only occur with terminal verbs √\thinspace\tsans{k.r}\,$_{\text{\acrshort{class}}_8}$, √\thinspace\tsans{bhuu}\,$_{\text{\acrshort{class}}_1}$, and √\thinspace\tsans{as}\,$_{\text{\acrshort{class}\textsubscript{2}}}$ when forming factitive compound verbs like √\thinspace\tsans{adharii-k.r} (\Panini's \Astadhyayi: 5.4.50), \vid\ \textcite[\textbf{1094}, p.~357]{Whitney}. \protect\\[0.5\baselineskip]
A regular sandhi of the words \tsans{hataa}\,$_\text{\acrshort{past}-\acrshort{passive}-\acrshort{participle}}$ 
+ \tsans{.adhariik.rtaa}\,$_\text{\acrshort{past}-\acrshort{passive}-\acrshort{participle}}$ (from √\thinspace\tsans{adharii-k.r}\,$_{\text{\acrshort{class}}_8}$) generates \tsans{hataadharii\underline{k.r}taa} that is morphologically correct and contextually apposite. Also, \tsans{.adhariik.r}\,$_\text{\acrshort{class}\textsubscript{8}}$ is variously attested in this chapter, as well as in \Nityananda's  \Sarvasiddhantaraja\ (\textcite{MisraTD}), in the same mathematical context. \Vid\ glossary entry: \gls{lowering}.}} %
%
bhavetsad.rk.sako.tisi{\dev{~nji}}nii || 5 ||\marginnote{\te{(8)}}


taddhanurnavatita"scyuta.m yadaa jaayate sad.r"sabaahusa.m{\dev{~nja}}kam ||\\
yaa nabhogavi"sikhasya si{\dev{~nji}}nii bhaajitaa .adharasad.rk.sadorjyayaa || 6 ||

\marginnote{\te{\textsuperscript{\tsans{...\selip hvayo bhavedvaa}\,$^\text{¶}$}(9)}}
taddhanu.h para"saraahvayo bhavedvaa pare.suparamaapamaakhyayo.h ||\\
sa.myutirviyutirasti ca kramaadgolabaa.nasamabhinnadiktayaa || 7 ||

sa grahasya paramasphu.taapamo jaayate yutiviyogadiksthita.h ||\\
evamabhranavato.adhiko yadaa khaa.s.tabhuu 180 parimitervi"sodhita.h || 8 || \marginnote{\te{(10)}}

parasphu.takraantibhavajyakaa hataa  sad.rk.sabaahujyakayaa .adhariik.rtaa ||  \\
tadiiya caapa.m bhavati sphu.taapamo digasya sa.myogaviyogadiksamaa || 9 || \marginnote{\te{(11)}}% 
}
\textbf{Now, the \gls{knowledge} (\textit{jñāna}) of the \gls{true_declination} (\textit{spaṣṭa-krānti}) in the sixth chapter.}\medskip

[Given] the \gls{latitude_celestial_object} (\textit{khagasya bāṇa}), [and] again, the \gls{other_declination} (\textit{anyatara-apama}) [\ie the second declination]: if indeed both should be situated in \gls{one_direction} (\textit{eka-diś}), then [we take] the \gls{sum} (\textit{saṃyuti}) of both of them; otherwise, [we take their] \gls{difference} (\textit{antara}). [The result] is known as the \glslink{curve_true_declination}{share of the true declination} (\textit{sphuṭa-apama-aṃśa}). Here, [it is] said to be [in] its \gls{own_direction} (\textit{sva-diś}). 1 \marginnote{(1)} 

The \gls{Sine_curve_true_declination} (\textit{sphuṭa-apama-aṅka-siñjinī}), \glslink{lowering}{having been lowered} (\textit{adharī-kṛtā}), \glslink{multiplication}{is multiplied} (\textit{ni-hanyate}) by the \gls{day_Sine_increased_by_three_signs} (\textit{sa-bha-traya-dyujīvā}) [\ie Cosine of the first declination of the longitude increased by 90\degree]. [The result] should be the \gls{Sine_true_declination} (\textit{sphuṭa-apama-jyakā}). 2 \marginnote{(2)}
\bigskip

The \gls{Cosine_maximum_declination} (\textit{parama-krānti-koṭijyā}) [\ie Cosine of the ecliptic obliquity], \glslink{multiplication}{having been multiplied} (\textit{hatā}) by the \gls{Sine_curve_true_declination} (\textit{sphuṭa-krānti-aṅka-jīvā}) [and] \glslink{division}{having been divided} (\textit{āptā}) by the \gls{Cosine_other_declination} (\textit{anya-krānti-koṭijyā}) [\ie Cosine of the second declination], should be the \gls{Sine_true_declination} (\textit{spaṣṭa-apama-jyakā}). 3 \marginnote{(3)} 


Or, \glslink{extract}{having extracted} (\textit{utthāya}) the \glslink{product_multiplication}{product of the multiplication} (\textit{guṇita-phala}) with the \gls{Sine_curve_true_declination} (\textit{sphuṭa-krānti-aṅka-jyā}) from the \glslink{table_Cosine_maximum_declination}{tables of the <Co>sine of the maximum declination} (\textit{parama-krānti-<koṭi>jyā-koṣṭhika}s) [\ie from the tables of the Cosine of the ecliptic obliquity], [one] \glslink{division}{should divide} (\textit{bhajet}) [that product] by the \glslink{Cosine_second_declination_degree}{Cosine of the second declination} (\textit{dvitīya-krānti-koṭijyā}). The [result] \gls{obtained} (\textit{labdha}) [\ie the quotient of the division] should be the \gls{Sine_true_declination} (\textit{spaṣṭa-krānti-jyā}).  \marginnote{(4)}

And now, if the \glslink{latitude_celestial_object}{latitude of a celestial object} (\textit{khagasya bāṇa}) should not exist, then its [first] \glslink{declination_degree}{declination} (\textit{krānti}) alone should be the \gls{true_declination} (\textit{spaṣṭa-krānti}). \marginnote{(5)}

Now, if the [first] \glslink{declination_degree}{declination} (\textit{krānti}) should not exist but the \glslink{latitude_celestial_object}{latitude} (\textit{śara}) should, then the \glslink{Sine_latitude}{Sine of the latitude} (\textit{bāṇa-jyā}) that
\glslink{multiplication}{must be multiplied} (\textit{saṃ-guṇyā}) by the \gls{Cosine_maximum_declination} (\textit{parama-krānti-koṭijyā}) [\ie by the Cosine of the ecliptic obliquity], \glslink{lowering}{should [be made] lower} (\textit{adhaḥ kuryāt}). Or, [one] \glslink{extract}{may [again] extract} (\textit{utthāpayet}) the \glslink{product_multiplication}{product of the multiplication} (\textit{guṇita-phala}) with the \glslink{Sine_latitude}{Sine of the latitude} (\textit{bāṇa-jyā}) from the \glslink{table_Cosine_maximum_declination}{tables of the Cosine of the maximum declination} (\textit{parama-krānti-koṭijyā-koṣṭhaka}s) [\ie from the tables of Cosine of the ecliptic obliquity]. [The result] should be the \gls{Sine_true_declination} (\textit{spaṣṭa-krānti-jyā}) in the \gls{direction_latitude} (\textit{bāṇa-dik}). \marginnote{(6)}

If the \glslink{declination_degree}{declination of a celestial object} (\textit{khagasya krānti}) should be equal to the \gls{maximum_declination} (\textit{parama-krānti}) [\ie the obliquity of the ecliptic], then the very \gls{curve_true_declination} (\textit{spa\-ṣṭa-krānti-aṅka}) becomes the \gls{true_declination} (\textit{sphuṭa-krānti}). \marginnote{(7)}%
\medskip

\textbf{Now, in another way.}
\medskip

And what \gls{circle} (\textit{vṛtta}) reaches both the \gls{ecliptic_pole} (\textit{kadamba}) and the \gls{celestial_pole} (\textit{viṣuvat}\,$_\text{\acrshort{regular}}$-\textit{dhruva}), that has been stated to be the \glslink{solstitial_colure}{[circle] belonging to the solstice} (\textit{āyana}[-\textit{vṛtta}]), and also, the \glslink{solstitial_colure}{[circle] passing through the four poles} (\textit{dhruva-catuṣka-yāta}[-\textit{vṛtta}]). And passing over a \gls{celestial_object} (\textit{nabhoga}) and the \gls{pair_equinoctial_points} (\textit{visuvat}\,$_\text{\acrshort{regular}}$-\textit{dvaya}), what [circle] is \gls{well_rounded} (\textit{su-vṛtta}), that the \gls{knower_spheres} (\textit{gola-vid}) should consider as
the \glslink{circle_congruent_ecliptic}{[circle] congruent to the ecliptic} (\textit{bhacakra-sadṛśa}[-\textit{vṛtta}]) by name. 1 \marginnote{(α)} 

What \gls{arc} (\textit{dhanus}) produced on the \gls{solstitial_colure} (\textit{āyana-vṛtta}) becomes situated in the  \gls{difference} (\textit{vivara}) between the \gls{celestial_equator} (\textit{viṣuva}\,$_\text{\acrshort{regular}}$-\textit{vṛtta}) and the \glslink{circle_congruent_ecliptic}{[circle] congruent to the ecliptic} (\textit{bhavṛtta-sadṛśa}[-\textit{vṛtta}]), that is the stated [arc of] \gls{maximum_true_declination} (\textit{para-sphuṭa-apama}) of the \gls{celestial_object}(\textit{dyucara}) just at that present moment. 2 \marginnote{(β)}

What \gls{arc} (\textit{dhanus}) produced on the \gls{solstitial_colure} (\textit{āyana-vṛtta}) becomes situated in the \gls{difference} (\textit{vivara}) between the \gls{ecliptic} (\textit{bhavana-cakra}) and the \glslink{circle_congruent_ecliptic}{[circle] congruent to the ecliptic} (\textit{bhacakra-sadṛkṣa}[-\textit{vṛtta}]), in this case, that is the declared [arc of] \gls{maximum_latitude} (\textit{para-iṣu}) when the \glslink{conjunction_equinox_node}{conjunction of the equinoctial point and the node of the orbit of the celestial object} (\textit{viṣuva}\,$_\text{\acrshort{regular}}$-\textit{pāta-yuga}) has been supposed. 3 \marginnote{(γ)}

What \gls{arc} (\textit{kodaṇḍa}) of the  \glslink{circle_congruent_ecliptic}{[circle] congruent to the ecliptic} (\textit{bhavṛtta-sadṛś}[-\textit{vṛtta}]) is between the \gls{equinoctial_point} (\textit{viṣuvat}\,$_\text{\acrshort{regular}}$) and the \gls{celestial_object} (\textit{nabhoga}), that [arc] should be known as the \gls{congruent_bhuja} (\textit{sadṛś-bhujā}); [and what is] between the \gls{celestial_object} (\textit{bha}) and the \gls{solstitial_colure} (\textit{āyana}[-\textit{vṛtta}]), [that should be known as] the \gls{congruent_koti} (\textit{sadṛś-koṭi}) [\ie complement of \textit{sadṛś-bhujā}.]~4~\marginnote{(δ)}
\bigskip

The \gls{Sine_koti_celestial_object} (\textit{khagasya koṭi-siñjinī}), \glslink{multiplication}{having been multiplied} (\textit{hatā}) by the \glslink{Cosine_latitude_celestial_object}{Cosine of its latitude} (\textit{sva-bāṇa-koṭijīvā}) [and] \glslink{lowering}{having been lowered} (\textit{adharī-kṛtā}), should be the \glslink{Sine_distance_celestial_object_solstitial_colure}{Sine of the congruent complementary arc} (\textit{sadrkṣa-koṭi-siñjinī}) [\ie Sine of the complement of the \textit{sadṛś-bhujā}]. 5 \marginnote{(8)}


When [the measure of] its \gls{arc} (\textit{dhanus}), \glslink{reducing_from_ninety}{having been reduced from ninety} [degrees] (\textit{navatitaś-cyuta}), is determined, [it] has the name \glslink{congruent_bhuja}{congruent arc} (\textit{sadṛśa-bāhu}). Or, what is the \glslink{Sine_latitude}{Sine of the latitude of a celestial object} (\textit{nabhoga-viśikhasya siñjinī}), \glslink{division}{having been divided} (\textit{bhājitā}) by the \gls{lowered_Sine_congruent_base} (\textit{adhara-sadrkṣa-dorjyā}), 6$\enskip$\ldots its \gls{arc} (\textit{dhanus}) should be [called] the \gls{maximum_latitude} (\textit{para-śara}) by name. \marginnote{(9)}

There is the \gls{sum} (\textit{saṃyuti}) or the \gls{difference} (\textit{viyuti}) of the two [quantities] known as the \gls{maximum_latitude} (\textit{para-iṣu}) and the \gls{maximum_declination} (\textit{parama-apama}) [\ie the obliquity of the ecliptic] with the  \glslink{latitude_celestial_object}{latitude} (\textit{bāṇa}) and the \gls{celestial_hemisphere} (\textit{gola}) [\ie the declination of the celestial object] in the \gls{same_different_directions} (\textit{sama-bhinna-diś}) respectively. 7$\enskip$That [result], being situated in the \gls{conjunction_disjunction_direction} (\textit{yuti-viyoga-dik}), becomes the \glslink{maximum_true_declination}{maximum true declination of a celestial object} (\textit{grahasya parama-sphuṭa-apama}). Thus, when [its measure is] \gls{greater} (\textit{adhika}) than \gls{ninety} [degrees] (\textit{abhra-nava}), [it is] \glslink{subtraction}{made to be subtracted} (\textit{viśodhita}) from a measure of \gls{one_hundred_eighty} [degrees] (\textit{kha-aṣṭa-bhū}). 8 \marginnote{(10)}

The \gls{Sine_maximum_true_declination} (\textit{para-sphuṭa-krānti-bhava-jyakā}), \glslink{multiplication}{having been multiplied} (\textit{hatā}) by the \gls{Sine_congruent_base} (\textit{sadṛkṣa-bāhu-jyakā}) [and] \glslink{lowering}{having been lowered} (\textit{adharī-kṛtā}), its \gls{arc} (\textit{cāpa}) becomes the \gls{true_declination} (\textit{sphuṭa-apama}). Its \gls{direction} (\textit{diś}) is the \gls{same} (\textit{sama}) as the \gls{conjunction_disjunction_direction} (\textit{saṃyoga-viyoga-dik}). 9 \marginnote{(11)}

\textbf{Sixth chapter} \\
On the \gls{knowledge} (\marifat) of the \glspl{distance_celestial_object} (\bud\idafaconsonant\ \kawakib\ \az\ \muaddil\ \alnahar).   
\bigskip

[Given] the \gls{latitude_celestial_object} (\ard\idafaconsonant\ \kawkab) and the \gls{second_declination_degree} (\mayl\idafaconsonant\ \thani\idafavowel\ \daraji\idafavowel\ \uy), if both should be in \gls{one_direction} (\yik\ \jahat), we \gls{sum} (*\jam\ \kardan) [them]; otherwise, we should take the \gls{difference} (\tafadul). And we call that [result] the \gls{argument_of_distance} (\hissi\idafavowel\ \bud) [\lit share of the distance]. And the \gls{direction_argument_of_distance} (\jahat\idafaconsonant\ \hissi\idafavowel\ \bud) should be the \gls{direction_sum} (\jahat\idafaconsonant\ \majmu) or the \gls{direction_residue} (\jahat\idafaconsonant\ \fadla). (1)

Then, we \glslink{low_multiplication}{low multiply} (*\munhatt\idafaconsonant\ \darb\ \kardan) the \gls{Sine_argument_of_distance} (\jayb\idafaconsonant\ \hissi\idafavowel\ \bud) in the \gls{Cosine_inverse_declination_degree_celestial_object} (\jayb\idafaconsonant\ \tamam\idafaconsonant\ \mayl\idafaconsonant\ \mankus\idafaconsonant\ \daraji\idafavowel\ \kawkab). The \gls{result} (\hasil) is the \gls{Sine_distance} (\jayb\idafaconsonant\ \bud). (2)
\bigskip

In another way, we \glslink{multiplication}{multiply} (*\darb\ \kardan) the \gls{Sine_argument_of_distance} (\jayb\idafaconsonant\ \hissi\idafavowel\ \bud) in the \gls{Cosine_maximum_declination} (\jayb\idafaconsonant\ \tamam\idafaconsonant\ \mayl\idafaconsonant\ \kulli) [\ie  in the Cosine of the ecliptic obliquity] and we \glslink{division}{divide} (*\qismat\ \kardan) the \gls{result} (\hasil) over the \gls{Cosine_second_declination_degree} (\jayb\idafaconsonant\ \tamam\idafaconsonant\ \mayl\idafaconsonant\ \thani\idafavowel\ \daraji) of that \gls{celestial_object} (\kawkab). The \glslink{quotient_division}{quotient of the division} (\kharij\idafaconsonant\ \qismat) should be the \gls{Sine_distance} (\jayb\idafaconsonant\ \bud), and its \gls{direction} (\jahat)  should be the \gls{direction_argument_of_distance} (\jahat\idafaconsonant\ \hissi\idafavowel\ \bud). (3)


And since they \gls{extract} (*\darardan) [the product of the multiplication with] the \gls{Sine_argument_of_distance} (\jayb\idafaconsonant\ \hissi\idafavowel\ \bud) from the \gls{table_Cosine_maximum_declination} (\jadval\idafaconsonant\ \jayb\idafaconsonant\ \tamam\idafaconsonant\ \mayl\idafaconsonant\ \kulli) [\ie in the table of the Cosine of the ecliptic obliquity] and they \glslink{division}{divide} (*\qismat\ \kardan) the \gls{result} (\hasil) over the \gls{Cosine_second_declination_degree} (\jayb\idafaconsonant\ \tamam\idafaconsonant\ \mayl\idafaconsonant\ \thani\idafavowel\ \daraji) of that \gls{celestial_object} (\kawkab), the \glslink{quotient_division}{quotient of the division} (\kharij\idafaconsonant\ \qismat) should be the \gls{Sine_distance} (\jayb\idafaconsonant\ \bud). (4)

And if a \gls{celestial_object} (\kawkab) should have no \glslink{latitude_celestial_object}{latitude} (\ard), the \gls{declination_degree} (\mayl\idafaconsonant\ \daraji\idafavowel\ \uy) should be the \glslink{distance_celestial_object}{distance} (\bud). (5)

And if the \glslink{latitude_celestial_object}{latitude} (\ard) [of a celestial object] should exist but \glslink{degree_celestial_object}{its degree} (\daraji\idafavowel\ \uy)  should have no \glslink{declination_degree}{declination}  (\mayl), we  \glslink{low_multiplication}{low multiply} (*\munhatt\idafaconsonant\ \darb\ \kardan) the \gls{Sine_latitude} (\jayb\idafaconsonant\ \ard\idafaconsonant\ \uy) in the \gls{Cosine_maximum_declination} (\jayb\idafaconsonant\ \tamam\idafaconsonant\ \mayl\idafaconsonant\ \kulli)  [\ie  in the Cosine of the ecliptic obliquity], or we \gls{extract} (*\darardan) [the product of this multiplication] from the preceding \gls{table} (\jadval) [\ie in the table of the Cosine of the maximum declination]. The \gls{result} (\hasil) should be the \gls{Sine_distance} (\jayb\idafaconsonant\ \bud), and its \gls{direction} (\jahat) should be the \gls{direction_latitude} (\jahat\idafaconsonant\ \ard). (6) 

And if the \gls{declination_degree} (\mayl\idafaconsonant\ \daraji\idafavowel\ \uy) should be the \gls{maximum_declination} (\mayl\idafaconsonant\ \kulli) [\ie the obliquity of the ecliptic], the \gls{argument_of_distance} (\hissatalbud) itself should be the \glslink{distance_celestial_object}{distance} (\bud). (7)
\bigskip


And in another way, we \glslink{low_multiplication}{low multiply} (*\munhatt\idafaconsonant\ \darb\ \kardan) the \gls{Sine_distance_celestial_object_nearest_solstice} (\jayb\idafaconsonant\ \bud\idafaconsonant\ \daraji\idafavowel\ \kawkab\ \az\ \inqilab\idafaconsonant\ \aqrab) in the \glslink{Cosine_latitude_celestial_object}{Cosine of the latitude of the celestial object} (\jayb\idafaconsonant\ \tamam\idafaconsonant\ \ard\idafaconsonant\ \kawkab). The \gls{result} (\hasil) should be the
\glslink{Sine_distance_celestial_object_solstitial_colure}{Sine of the distance of the celestial object from the `circle passing through the four poles'} [\ie from the `solstitial colure'] (\jayb\idafaconsonant\ \bud\idafaconsonant\ \kawkab\ \az\ \guillemotleft\dayiri\idafavowel\ \marri\ \biaqtab\idafaconsonant\ \arbai\guillemotright). (8)


Then, we \glslink{low_division}{low divide} (*\munhatt\idafaconsonant\ \qismat\ \kardan) the \glslink{Sine_latitude}{Sine of the latitude of a celestial object} (\jayb\idafaconsonant\ \ard\idafaconsonant\ \kawkab) over the \glslink{Cosine_distance_solstitial_colure}{Cosine of the distance [of the celestial object] from the `circle passing through the four poles'} (\jayb\idafaconsonant\ \tamam\idafaconsonant\ \bud\ \az\ \guillemotleft\dayiri\idafavowel\ \marri\ \biaqtab\idafaconsonant\ \arbai\guillemotright). And for the \glslink{quotient_division}{quotient of the division} (\kharij\idafaconsonant\ \qismat), we should take the \gls{arc} (\qaws) from the \gls{table_of_Sine} (\jadval\idafaconsonant\ \jayb). And we call it the \gls{first_arc} (\qaws\idafaconsonant\ \avval) and its \gls{direction} (\jahat) is the \glslink{direction_latitude}{direction of the latitude of the celestial object} (\jahat\idafaconsonant\ \ard\idafaconsonant\ \kawkab). (9)


Then, if the \glslink{latitude_celestial_object}{latitude} (\ard) and the \glslink{declination_degree}{declination of the degree of the celestial object} (\mayl\idafaconsonant\ \daraji\idafavowel\ \kawkab) both should be in \gls{one_direction} (\yik\ \jahat), we \gls{sum} (*\jam\ \kardan) the \gls{first_arc} (\qaws\idafaconsonant\ \avval) and the \gls{maximum_declination} (\mayl\idafaconsonant\ \kulli) [\ie the ecliptic obliquity]. And if [the sum] \gls{exceeds} (*\ziyadi\ \shudan) \gls{one_quarter} (\rabi) [\ie is greater than 90\degree], we should take the \gls{whole_sum} (\tamam\idafaconsonant\ \majmu) up to \gls{one_half} (\nisf) [\ie up to 180\degree]. And if they should be in \gls{different_directions} (\jahat\idafaconsonant\ \mukhtalif), we should take the \gls{difference} (\tafadul) between the two; the \gls{result} (\hasil) [in both cases] should be [called] 
the \gls{second_arc} (\qaws\idafaconsonant\ \duvum) and its \gls{direction} (\jahat) should be the \gls{direction_sum} (\jahat\idafaconsonant\ \majmu) or the \gls{direction_residue} (\jahat\idafaconsonant\ \fadla). (10)

Then, we \glslink{low_multiplication}{low multiply} (*\munhatt\idafaconsonant\ \darb\ \kardan) the \gls{Sine_second_arc} (\jayb\idafaconsonant\ \qaws\idafaconsonant\ \duvum) in the \glslink{Cosine_distance_solstitial_colure}{Cosine of the distance [of a celestial object] from the `circle passing through the four poles'} (\jayb\idafaconsonant\ \tamam\idafaconsonant\ \bud\ \az\ \guillemotleft\dayiri\idafavowel\ \marri\ \biaqtab\idafaconsonant\ \arbai\guillemotright). The result should be the \glslink{Sine_distance}{Sine of the distance of the celestial object} (\jayb\idafaconsonant\ \bud\idafaconsonant\ \kawkab) and its \gls{direction} (\jahat) should be the \gls{direction_second_arc}  (\jahat\idafaconsonant\ \qaws\idafaconsonant\ \duvum). (11)

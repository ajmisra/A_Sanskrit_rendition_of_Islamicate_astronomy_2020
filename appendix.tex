\appendix
% \renewcommand{\thesection}{\Alph{section}}
\phantomsection % Helps the TOC pick up and link to the Appendix properly
\addcontentsline{toc}{section}{Appendix: Persian and Sanskrit Verbs}
\OnehalfSpacing
%%%%%%%%%%%%%%%%%%%%%%%%%%%%%%%%%%%%%%%%

\section*{Appendix: Persian and Sanskrit Verbs}\label{Appendix_verbs}

The attested forms of Persian verbs from the \ZijiShahJahani\ Discourse~II.6 (\S~\ref{zijshahjahan_persian_english}) and of Sanskrit verbs from the \Siddhantasindhu\ Part~II.6 (\S~\ref{siddhantasindhu_sanskrit_english}) are listed below separately. At the end of each entry, \textbf{passage-markers} in square brackets point to its location in \S~\ref{zijshahjahan_persian_english} or \S~\ref{siddhantasindhu_sanskrit_english} accordingly.

\subsection*{Persian verbs in the \ZijiShahJahani, Discourse II.6} \label{persian_verbs}

\begin{enumerate}
    \item \tfarsi{بودن} (\textit{budan}) `to be' \hfill \tfarsi{باش} (\textit{bāsh}); \tfarsi{بود} [\acrshort{present}-\acrshort{stem}]
(\textit{buvad})\footnote{\,\Vid\ Chapter XI in \textcite[251--268]{Lenepveu_Hotz} for a diachronic study of the third singular present form of the verb `to be' from the indicative \tfarsi{بود} (\textit{buvad}) to the subjunctive \tfarsi{باشد} (\textit{bāshad}) in the Persian verbal system (10\textsuperscript{th}--16\textsuperscript{th} \ce).}
\begin{itemize}
        \item \tfarsi{باشند} (\textit{bāshand})  \acrshort{present}-\acrshort{subjunctive}-\acrshort{plural}·\acrshort{third} `[they] should be' [\hyperlink{Ppass1}{1}, \hyperlink{Ppass10}{10}]
          \item \tfarsi{باشد} (\textit{bāshad})  \acrshort{present}-\acrshort{subjunctive}-\acrshort{singular}·\acrshort{third} `[he/she/it] should be' [\hyperlink{Ppass1}{1}, \hyperlink{Ppass3}{3}--\hyperlink{Ppass8}{8}, \hyperlink{Ppass10}{10}, \hyperlink{Ppass11}{11}]
          \item \tfarsi{نباشد} (\textit{nabāshad}) \acrshort{present}-\acrshort{subjunctive}-\acrshort{singular}·\acrshort{third} (\acrshort{negation}) `[he/she/it] should not be' [\hyperlink{Ppass5}{5}, \hyperlink{Ppass6}{6}]
          \item \tfarsi{بود} (\textit{buvad}) \acrshort{present}-\acrshort{indicative}-\acrshort{singular}·\acrshort{third} `[he/she/it] is' [\hyperlink{Ppass2}{2}, \hyperlink{Ppass9}{9}]
    \end{itemize}
    \item \tfarsi{خواندن} (\textit{khāndan}) `to recite' \hfill \tfarsi{خوان} (\textit{khān}) [\acrshort{present}-\acrshort{stem}]
    \begin{itemize}
        \item \tfarsi{خوانیم}  (\textit{khānīm}) \acrshort{present}-\acrshort{indicative}-\acrshort{plural}·\acrshort{first} `[we] call' [\hyperlink{Ppass1}{1}, \hyperlink{Ppass9}{9}]
    \end{itemize}
    \item \tfarsi{درآوردن} (\textit{dar āvardan}) `to remove/produce/extract\newline\phantom{a}\hfill \tfarsi{در آور} (\textit{dar āvar}) [\textsc{variant}: \tfarsi{در آر} (\textit{dar ār})] [\acrshort{present}-\acrshort{stem}]
    \begin{itemize}
        \item \tfarsi{درآورند} (\textit{dar āvarand}) [\textsc{variant}: \tfarsi{درآرند} (\textit{dar ārand})] \acrshort{present}-\acrshort{indicative}-\acrshort{plural}·\acrshort{third}\newline `[they] remove/produce/extract' [\hyperlink{Ppass4}{4}]
        \item \tfarsi{درآوریم} (\textit{dar avarīm}) [\textsc{variant}: \tfarsi{درآریم} (\textit{dar ārim})]
        \acrshort{present}-\acrshort{indicative}-\acrshort{plural}·\acrshort{first}\newline `[we] remove/produce/extract' [\hyperlink{Ppass6}{6}]
    \end{itemize}
    \item \tfarsi{کردن} (\textit{kardan}) `to do/make'\footnote{\,The action verb \tfarsi{کردن} (\textit{kardan}) is often used to construct compound verbs, \eg \tfarsi{جمع کردن} (\textit{jama}\Ayn\ \textit{kardan}) `to sum/add' or \tfarsi{ضرب کنیم} (\textit{ḍarb kardan}) `to multiply'\label{compound_action_verb_persian}} \hfill  \tfarsi{کن} (\textit{kun}) [\acrshort{present}-\acrshort{stem}]
    \begin{itemize}
        \item \tfarsi{کنیم} (\textit{kunīm}) \acrshort{present}-\acrshort{indicative}-\acrshort{plural}·\acrshort{first} `[we] do/make' [\hyperlink{Ppass1}{1}--\hyperlink{Ppass3}{3}, \hyperlink{Ppass6}{6}, \hyperlink{Ppass8}{8}--\hyperlink{Ppass11}{11}]
        \item \tfarsi{کنند} (\textit{kunand}) \acrshort{present}-\acrshort{indicative}-\acrshort{plural}·\acrshort{third} `[they] do/make' [\hyperlink{Ppass4}{4}]
    \end{itemize}
    \item \tfarsi{گرفتن} (\textit{giriftan}) `to take/grab' \hfill  \tfarsi{گیر} (\textit{gīr})[\acrshort{present}-\acrshort{stem}]
    \begin{itemize}
        \item \tfarsi{بگیریم} (\textit{bigīrīm}) \acrshort{present}-\acrshort{subjunctive}-\acrshort{plural}·\acrshort{first} `[we] should take' [\hyperlink{Ppass1}{1}, \hyperlink{Ppass9}{9}, \hyperlink{Ppass10}{10}]
    \end{itemize}
\end{enumerate}
\clearpage

\subsection*{Sanskrit verbs in the \Siddhantasindhu, Part II.6} \label{sanskrit_verbs}

\begin{enumerate}
\item   \begin{enumerate}
        \item √\tsans{as} (√\textit{as}) \acrshort{class}$_\text{2}$  `to be/exist'
        \begin{itemize}
            \item \tsans{asti} (\textit{asti}) \acrshort{present}-\acrshort{indicative}-\acrshort{active}-\acrshort{singular}·\acrshort{third} `[he/she/it] is' [\hyperlink{Spass10}{10}]
            \item \tsans{sat} (\textit{sat}) \acrshort{present}-\acrshort{active}-\acrshort{participle} `being' [\hyperlink{SpassC}{γ}]        
            \item \tsans{syaat} (\textit{syāt}) \acrshort{optative}-\acrshort{active}-\acrshort{singular}·\acrshort{third} `[he/she/it] should be/exist' [\hyperlink{Spass3}{3}--\hyperlink{Spass7}{7}]
                \begin{itemize}
                \item \tsans{na syaat} (\textit{na syāt}) \acrshort{optative}-\acrshort{active}-\acrshort{singular}·\acrshort{third} (\acrshort{negation}) `[he/she/it] should not be/exist' [\hyperlink{Spass5}{5}, \hyperlink{Spass6}{6}]
                \end{itemize}
        \end{itemize}
        \item √\tsans{jan} (√\textit{jan}) \acrshort{class}$_\text{4}$  `to be born/determined'
        \begin{itemize}
            \item \tsans{jaayate} (\textit{jāyate}) \acrshort{present}-\acrshort{indicative}-\acrshort{middle}-\acrshort{singular}·\acrshort{third} `[he/she/it] comes to be'/`[he/she/it] is determined' [\hyperlink{Spass9}{9}, \hyperlink{Spass10}{10}]
       \end{itemize}
       \item √\tsans{bhuu} (√\textit{bhū}) \acrshort{class}$_\text{1}$  `to be/become'
        \begin{itemize}
            \item \tsans{bhavet} (\textit{bhavet}) \acrshort{optative}-\acrshort{active}-\acrshort{singular}·\acrshort{third} `[he/she/it] should be/become' [\hyperlink{Spass1}{1}, \hyperlink{Spass2}{2}, \hyperlink{Spass5}{5}, \hyperlink{Spass6}{6}, \hyperlink{Spass8}{8}, \hyperlink{Spass9}{9}]
            \item \tsans{bhavati} (\textit{bhavati}) \acrshort{present}-\acrshort{indicative}-\acrshort{active}-\acrshort{singular}·\acrshort{third} `[he/she/it] becomes' [\hyperlink{Spass7}{7}, \hyperlink{SpassB}{β}, \hyperlink{SpassC}{γ}, \hyperlink{Spass11}{11}]
        \end{itemize}
        \item √\tsans{sthaa} (√\textit{sthā}) \acrshort{class}$_\text{1}$  `to stand/situate'
        \begin{itemize}
            \item \tsans{sthita} (\textit{sthita})
            \acrshort{past}-\acrshort{passive}-\acrshort{participle} `being stationed/situated'(\acrshort{active}-sense) [\hyperlink{Spass1}{1}, \hyperlink{Spass10}{10}]
        \end{itemize}
        \end{enumerate}
\item   \begin{enumerate}
        \item √\tsans{udiir} (√\textit{udīr}) \acrshort{class}$_\text{2}$  `to state/utter'
        \begin{itemize}
            \item \tsans{udiirita} (\textit{udīrita})
            \acrshort{causative}-\acrshort{past}-\acrshort{passive}-\acrshort{participle} `has been stated to be' [\hyperlink{SpassA}{α}]
        \end{itemize}
        \item √\tsans{kath} (√\textit{kath}) \acrshort{class}$_\text{10}$  `to declare/tell' 
        \begin{itemize}
            \item \tsans{kathita} (\textit{kathita}) \acrshort{past}-\acrshort{passive}-\acrshort{participle}  `declared/told'  (\acrshort{adjective}-use) [\hyperlink{SpassB}{β}]
        \end{itemize}
        \item √\tsans{vac} (√\textit{vac}) \acrshort{class}$_\text{2}$  `to say'
        \begin{itemize}
            \item \tsans{ucyate} (\textit{ucyate}) \acrshort{present}-\acrshort{passive}-\acrshort{singular}·\acrshort{third} `[he/she/it] is said [to be]' [\hyperlink{Spass1}{1}]
        \end{itemize}
        \item √\tsans{vad} (√\textit{vad}) \acrshort{class}$_\text{1}$  `to declare/state'
        \begin{itemize}
            \item \tsans{udita} (\textit{udita}) \acrshort{past}-\acrshort{passive}-\acrshort{participle}  `declared/stated' (\acrshort{adjective}-use)  [\hyperlink{SpassD}{γ}]
        \end{itemize}
        \end{enumerate} 
\item \begin{enumerate}
        \item √\tsans{aap} (√\textit{āp}) \acrshort{class}$_\text{5}$ `to reach' (in arithmetic, `to divide')
        \begin{itemize}
            \item \tsans{aapta} (\textit{āpta}) \acrshort{past}-\acrshort{passive}-\acrshort{participle} `having been reached/divided' [\hyperlink{Spass3}{3}]
        \end{itemize}
        \item  √\tsans{bhaj} (√\textit{bhaj}) \acrshort{class}$_\text{1}$ `to divide'
        \begin{itemize}
            \item \tsans{bhajet} (\textit{bhajet}) \acrshort{optative}-\acrshort{active}-\acrshort{singular}·\acrshort{third} `[he/she/it] should divide' [\hyperlink{Spass4}{4}]
            \item \tsans{bhaajita} (\textit{bhājita}) \acrshort{causative}-\acrshort{past}-\acrshort{passive}-\acrshort{participle} `having been divided' (in a causal sense) [\hyperlink{Spass9}{9}]
        \end{itemize}
        \end{enumerate}        
\item \begin{enumerate}
        \item √\tsans{ni-han} (√\textit{ni-han}) \acrshort{class}$_\text{2}$  `to strike in' (in arithmetic, `to multiply')
        \begin{itemize}
            \item \tsans{ni-hanyate} (\textit{ni-hanyate}) \acrshort{present}-\acrshort{passive}-\acrshort{singular}·\acrshort{third} `[he/she/it] is struck/multiplied' [\hyperlink{Spass2}{2}]
            \item \tsans{hata} (\textit{hata}) \acrshort{past}-\acrshort{passive}-\acrshort{participle} `having been struck/multiplied' [\hyperlink{Spass3}{3}, \hyperlink{Spass8}{8}, \hyperlink{Spass11}{11}]
        \end{itemize}
        \item √\tsans{sa.m-gu.n} (√\textit{saṃ-guṇ}) \acrshort{class}$_\text{10}$  `to multiply' 
        \begin{itemize}
            \item \tsans{sa.m-gu.nya} (\textit{saṃ-guṇya}) \acrshort{gerundive} `to/must be multiplied' [\hyperlink{Spass6}{6}]
        \end{itemize}
        \end{enumerate}
\item \begin{enumerate}
    \item  √\tsans{cyu} (√\textit{cyu}) \acrshort{class}$_\text{1}$  `to deviate'/`be deprived of' (with \acrshort{ablative}-use)
    \begin{itemize}
        \item {\tsans{cyuta}} (\textit{cyuta}) \acrshort{past}-\acrshort{passive}-\acrshort{participle} `having deviated from'/`having been deprived of' [\hyperlink{Spass9}{9}]
    \end{itemize}
    \item √\tsans{vi-"sudh} (√\textit{vi-śudh}) \acrshort{class}$_\text{1}$  `to purify/subtract' 
    \begin{itemize}
    \item {\tsans{vi-"sodhita}} (\textit{vi-śodhita})  \acrshort{causative}-\acrshort{past}-\acrshort{passive}-\acrshort{participle} `made to be purified/subtracted' [\hyperlink{Spass10}{10}]
    \end{itemize}
    \end{enumerate}        
\item √\tsans{upe} (√\textit{upe}) \acrshort{class}$_\text{2}$ `to reach'
    \begin{itemize}
        \item \tsans{upaiti} (\textit{upaiti}) \acrshort{present}-\acrshort{indicative}-\acrshort{singular}·\acrshort{third} `[he/she/it] reaches' [\hyperlink{SpassA}{α}]
    \end{itemize}
\item √\tsans{utthaa} (√\textit{utthā}) \acrshort{class}$_\text{1}$ `to rise/extract'
    \begin{itemize}
        % \item \tsans{uttha} (\textit{uttha}) \acrshort{primary_nominal_derivative} `derived/extracted' (\acrshort{adjective}-use) 
        \item \tsans{utthaapayet} (\textit{uthayet}) \acrshort{causative}-\acrshort{optative}-\acrshort{active}-\acrshort{singular}·\acrshort{third} `[he/she/it] may raise/extract' [\hyperlink{Spass6}{6}]
        \item \tsans{utthaaya} (\textit{utthāya}) \acrshort{gerund} `having risen/extracted' [\hyperlink{Spass4}{4}]
    \end{itemize}    
\item √\tsans{k.r} (√\textit{kṛ}) \acrshort{class}$_\text{8}$  `to do/make'\footnote{\,The action verb √\tsans{k.r} (√\textit{kr}) is often compounded with an inflected nominal word to form a denominative verb (\acrshort{denominative}, \textit{nāmadhātu}), \eg √\tsans{adharii-k.r} (√\textit{adharī-kṛ}) `to make [something] low', \ie `to lower'.}
        \begin{itemize}
        \item \tsans{kuryaat} (\textit{kuryāt}) \acrshort{optative}-\acrshort{active}-\acrshort{singular}·\acrshort{third} `[he/she/it] should do/make' [\hyperlink{Spass6}{6}]
        \item \tsans{k.rta} (\textit{kṛta}) \acrshort{past}-\acrshort{passive}-\acrshort{participle} `having been done/made' [\hyperlink{Spass2}{2}, \hyperlink{Spass8}{8}, \hyperlink{Spass11}{11}]
        \end{itemize}
\item √\tsans{k.lp} (√\textit{kḷp}) \acrshort{class}$_\text{1}$  `to consider/suppose'
        \begin{itemize}
        \item \tsans{kalpayet} (\textit{kalpayet}) \acrshort{causative}-\acrshort{optative}-\acrshort{active}-\acrshort{singular}·\acrshort{third} `[he/she/it]' should consider/suppose' [\hyperlink{SpassA}{α}]
        \item \tsans{kalpita} (\textit{kalpita})  \acrshort{causative}-\acrshort{past}-\acrshort{passive}-\acrshort{participle} `has been considered/supposed' [\hyperlink{SpassC}{γ}]
        \end{itemize}        
\item √\tsans{pat} (√\textit{pat}) \acrshort{class}$_\text{1}$ `to fall/pass'
    \begin{itemize}
        \item \tsans{patat} (\textit{patat}) \acrshort{present}-
        \acrshort{active}-\acrshort{participle} `falling/passing' [\hyperlink{SpassA}{α}]
    \end{itemize}
\item √{\tsnb{ज्ञा}} (√\textit{jñā}) \acrshort{class}$_\text{9}$  `to know/understand'
    \begin{itemize}
    \item {\tsnb{ज्ञेय}} (\textit{jñeya}) \acrshort{gerundive} `to be known'/`to be understood' [\hyperlink{SpassD}{δ}]
    \end{itemize}
\end{enumerate}
\newglossaryentry{knowledge}
{
        name={knowledge},
        description={\tfarsi{معرفت} (\marifat)\\[5pt]
        \tsans{{\tsnb{ज्ञा}}na} (\textit{jñāna})}
}

\newglossaryentry{distance_celestial_object}
{
        name={distance of a celestial object from the celestial equator},     
        description={\tfarsi{بعد كوكب از معدّل النها} (\bud\idafaconsonant\ \kawkab\ \az\ \muaddil\ \alnahar) \\[5pt] 
        Alias: \begin{tabular}[t]{ll}
            \textbf{distance} & \tfarsi{بعد}  (\bud)
        \end{tabular}\\[5pt]
        the distance of a celestial object from the celestial equator measured along the great circle passing through the two celestial poles and the celestial object; in other words, the true declination of a celestial object.\\[5pt]
        \Cf \begin{tabular}[t]{ll}\protect\gls{true_declination} & \tsans{spa.s.ta-kraanti} (\textit{spaṣṭa-krānti})\end{tabular}},
        plural={distance of celestial objects from the celestial equator}
}

\newglossaryentry{celestial_equator}
{
        name={celestial equator},     
        description={\tfarsi{معدّل النها}  (\muaddil\ \alnahar)\\[5pt]
        \tsans{vi.suva-v.rtta} (\textit{viṣuva-vṛtta})\\[5pt] MS~Kh attests an irregular (vernacular?) form \tsans{vi.sava-v.rtta} (\textit{viṣava-v.rtta}), \vid\ \S\thinspace\ref{siddhantasindhu_sanskrit_english}:~footnote~\ref{vishava_deviant}.}
}

\newglossaryentry{ecliptic}
{
        name={ecliptic},     
        description={\tfarsi{فلك البروج} (\falak\ \alburuj)\\[5pt] 
        \tsans{bhavana-cakra} (\textit{bhavana-cakra}), \lit `circle of astrological houses'} 
}

\newglossaryentry{latitude_celestial_object}
{
        name={latitude of a celestial object},
        description={\tfarsi{عرض کوکب} (\ard\idafaconsonant\ \kawkab)\\[5pt]
        \tsans{khagasya baa.na} (\textit{khagasya bāṇa})\\[5pt]
        Alias: \begin{tabular}[t]{ll}
            \textbf{latitude} &\tfarsi{عرض}  (\ard)\\[5pt]
            & \tsans{"sara} (\textit{śara})
        \end{tabular}
        % the arc of the great circle passing through the two ecliptic poles and a celestial object, and lying between the celestial object and the ecliptic.
        }
}

\newglossaryentry{celestial_object}
{
        name={celestial object},
        description={\tfarsi{کوکب} (\kawkab), pl.\thinspace \tfarsi{كواكب} (\kawakib)\\[5pt]
       \tsans{khaga} (\textit{khaga}); \tsans{graha} (\textit{graha}); \tsans{dyucara} (\textit{dyucara}); \tsans{bha} (\textit{bha}); \tsans{nak.satra} (\textit{nakṣatra}); \tsans{nabhoga} (\textit{nabhoga}); \tsans{svaga} (\textit{svaga})\\[5pt]
        a heavenly body like a planet, star, or asterism that moves in the celestial sphere.},
        plural={celestial objects}
}



\newglossaryentry{second_declination_degree}
{
        name={second declination of its degree},
        description={\tfarsi{میل ثاني درجه او} (\mayl\idafaconsonant\ \thani\idafavowel\ \daraji\idafavowel\ \uy)\\[5pt]
        the \textit{second} declination (δ\textsubscript{2}) of a point on the ecliptic with
        ecliptic longitude (λ\degree), measured in degrees, corresponding to the position of a celestial object; in other words, the arc of the great circle passing through the two ecliptic poles and a point on the ecliptic (with longitude λ\degree), and lying between the ecliptic and the celestial equator.\\[5pt]
         \Cf \begin{tabular}[t]{ll}\protect\gls{other_declination} &\tsans{anyatara-apama} (\textit{anyatara-apama})\end{tabular}}
}

\newglossaryentry{Cosine_second_declination_degree}
{
        name={Cosine of the second declination of the degree},
        description={\tfarsi{جيب تمام میل ثاني درجه} (\jayb\idafaconsonant\ \tamam\idafaconsonant\ \mayl\idafaconsonant\ \thani\idafavowel\ \daraji)\\[5pt]
        Alias: \begin{tabular}[t]{l}
            \textbf{Cosine of the second declination}\\ \tsans{dvitiiya-kraanti-ko.tijyaa} (\textit{dvitīya-krānti-koṭijyā})
        \end{tabular}\\[5pt]
        $\mathcal{R}$\thinspace cosine of the \protect\gls{second_declination_degree}.\\[5pt]
         \Cf \begin{tabular}[t]{l}\protect\gls{Cosine_other_declination} \\\tsans{anya-kraanti-ko.tijyaa} (\textit{anya-krānti-koṭijyā})\end{tabular}}
}

\newglossaryentry{Cosine_other_declination}
{
        name={Cosine of the other declination},
        description={\tsans{anya-kraanti-ko.tijyaa} (\textit{anya-krānti-koṭijyā})\\[5pt]
        $\mathcal{R}$\thinspace cosine of the \protect\gls{other_declination}.\\[5pt]
        \Cf \begin{tabular}[t]{l}\protect\gls{Cosine_second_declination_degree}\\ \tfarsi{جيب تمام میل ثاني درجه} (\jayb\idafaconsonant\ \tamam\idafaconsonant\ \mayl\idafaconsonant\ \thani\idafavowel\ \daraji)\end{tabular}}
}

\newglossaryentry{one_direction}
{
        name={one direction},
        description={\tfarsi{یک جهت}  (\yik\ \jahat)\\[5pt]
        \tsans{eka-di"s} (\textit{eka-diś})\\[5pt]
        the orientation of two arc of a great circle towards a common pole.\\[5pt]
        \Cf \begin{tabular}[t]{ll}
        \protect\gls{direction_sum} & \tfarsi{جهت مجموع}  (\jahat\idafaconsonant\ \majmu)
        \end{tabular}}
}

 \newglossaryentry{sum}
{
        name={sum},
        description={\tfarsi{جمع} (\jam)\\[5pt]
        \tsans{sa.myuti} (\textit{saṃyuti})}
}

\newglossaryentry{difference}
{
        name={difference},
        description={\tfarsi{تفاضل} (\tafadul)\\[5pt]
        \tsans{antara}  (\textit{antara}); \tsans{vivara} (\textit{vivara}); \tsans{viyuti} (\textit{viyuti})\\[5pt]
        \Cf\begin{tabular}[t]{ll}
             \protect\gls{subtraction} & \tsans{vi"sodhana} (\textit{viśodhana}) \acrshort{noun}
        \end{tabular}}
}

\newglossaryentry{argument_of_distance}
{
        name={argument of the distance},
        description={\tfarsi{حصّهٔ بعد} (\hissi\idafavowel\ \bud), \lit `share of the distance'\\[5pt]
        Alias: \begin{tabular}[t]{ll}
             \textbf{share of the distance} &\tfarsi{حصّة البعد} (\hissatalbud) in Arabic
        \end{tabular}\\[5pt]
        the arc of the great circle passing through the two ecliptic poles and a celestial object, and lying between the celestial object  and the celestial equator. \Vid\ \S\thinspace\ref{zijshahjahan_persian_english}:~{\footnotesize \P}\thinspace(1).\\[5pt]
        \Cf \begin{tabular}[t]{ll}\protect\gls{curve_true_declination} &\tsans{sphu.ta-apama-a"nka} (\textit{sphuṭa-apama-aṅka})\end{tabular}}
}

\newglossaryentry{direction_argument_of_distance}
{
        name={direction of the argument of the distance},
        description={\tfarsi{جهت حصّهٔ بعد} (\jahat\idafaconsonant\ \hissi\idafavowel\ \bud)\\[5pt]
        the orientation of the \protect\gls{argument_of_distance} towards the north or the south ecliptic pole with respect to the celestial equator. 
        }
}


\newglossaryentry{direction}
{
        name={direction},
        description={\tfarsi{جهت} (\jahat)\\[5pt]
        \tsans{di"s} (\textit{diś})\\[5pt]
        the orientation of an arc of a great circle towards a northern or southern pole.}
}

\newglossaryentry{direction_sum}
{
        name={direction of the sum},
        description={\tfarsi{جهت مجموع}  (\jahat\idafaconsonant\ \majmu)\\[5pt]
        the orientation of the \protect\gls{sum} (addition) of two arcs of a great circle towards a northern or southern pole.\\[5pt]
        \Cf \begin{tabular}[t]{ll}
        \protect\gls{one_direction} & \tfarsi{یک جهت}  (\yik\ \jahat)
        \end{tabular}}        
}

\newglossaryentry{direction_residue}
{
        name={direction of the residue},
        description={\tfarsi{جهت فضل} (\jahat\idafaconsonant\ \fadla)\\[5pt]
        the orientation of the \protect\gls{difference} (residue) between two arcs of a great circle towards a northern or southern pole.\\[5pt]
        \Cf \begin{tabular}[t]{ll}
            \protect\gls{different_directions} & \tfarsi{جهت مختلف} (\jahat\idafaconsonant\ \mukhtalif)
        \end{tabular}}
}

\newglossaryentry{low_multiplication}
{
        name={low-multiplication},
        description={\begin{tabular}[t]{ll}
        to \textbf{low-multiply} & 
        \tfarsi{منحطّ ضرب کردن} (\munhatt\idafaconsonant\ \darb\ \kardan) \acrshort{infinitive}
        \end{tabular} \\[5pt]
        multiplication of sexagesimal numbers, and the division of the result (of the multiplication) by 60 (Radius); in other words, shifting the fractional point leftwards to \textit{lower} the value of the sexagesimal number.\\[5pt]
             \Cf\begin{tabular}[l]{lll}
              \protect\gls{lowering}& to \textbf{lower} &\tsans{adharii-}√\tsans{k.r} \acrshort{class}\textsubscript{8} [\acrshort{infinitive}]
        \end{tabular}}
}

\newglossaryentry{lowering}
{       
        name={lowering},
        description={\begin{tabular}[t]{ll}
         \textbf{having been lowered} &\tsans{adharii-k.rta} (\textit{adharī-kṛta}) \acrshort{past}-\acrshort{passive}-\acrshort{participle}\\[5pt]
         \textbf{should [be made]  lower} & \tsans{adha.h kuryaat} (\textit{adhaḥ kuryāt}) \acrshort{optative}-\acrshort{active}-\acrshort{singular}$\cdot$\acrshort{third}
        \end{tabular}\\[5pt]
        \textit{lowering} the value of the sexagesimal number, \ie dividing it by 60 (Radius); in effect, shifting the fractional point leftwards.\\[5pt]
        \Cf \begin{tabular}[t]{ll}\protect\gls{low_multiplication} & \tfarsi{منحطّ ضرب} (\munhatt\idafaconsonant\ \darb)\\[5pt]
        \protect\gls{low_division} &\tfarsi{منحطّ قسمت} (\munhatt\idafaconsonant\ \qismat)\end{tabular}}
}

\newglossaryentry{Sine_argument_of_distance}
{
        name={Sine of the argument of the distance},
        description={\tfarsi{جيب حصّهٔ بعد} (\jayb\idafaconsonant\ \hissi\idafavowel\ \bud)\\[5pt]
        $\mathcal{R}$\thinspace sine of the \protect\gls{argument_of_distance}.\\[5pt]
        \Cf \begin{tabular}[t]{l}\protect\gls{Sine_curve_true_declination}\\ \tsans{sphu.ta-apama-a"nka-si{\tsnb{ञ्जि}}nii}  (\textit{sphuṭa-apama-aṅka-siñjinī})\end{tabular}}         
}

\newglossaryentry{Sine_curve_true_declination}
{
        name={Sine of the curve of true declination},
        description={\tsans{sphu.ta-apama-a"nka-si{\tsnb{ञ्जि}}nii}  (\textit{sphuṭa-apama-aṅka-siñjinī}); \tsans{sphu.ta-kraanti-a"nka-jivaa} (\textit{sphuṭa-krānti-aṅka-jīvā}); \tsans{sphu.ta-kraanti-a"nka-jyaa} (\textit{sphuṭa-krānti-aṅka-jyā})\\[5pt]
        $\mathcal{R}$\thinspace sine of the \protect\gls{curve_true_declination}.\\[5pt]
        \Cf \begin{tabular}[t]{ll}\protect\gls{Sine_argument_of_distance} &\tfarsi{جيب حصّهٔ بعد} (\jayb\idafaconsonant\ \hissi\idafavowel\ \bud)\end{tabular}}        
}

\newglossaryentry{Sine_distance}
{
        name={Sine of the distance},
        description={\tfarsi{جيب بعد} (\jayb\idafaconsonant\ \bud)\\[5pt]
        Alias: \begin{tabular}[t]{l}
        \textbf{Sine of the distance of a celestial object} \\\tfarsi{جيب بعد کوکب}  (\jayb\idafaconsonant\ \bud\idafaconsonant\ \kawkab)                \end{tabular}\\[5pt]
         $\mathcal{R}$\thinspace sine of the \protect\gls{distance_celestial_object}.\\[5pt]
        \Cf \begin{tabular}[t]{ll}\protect\gls{Sine_true_declination} &\tsans{sphu.ta-apama-jyakaa} (\textit{sphuṭa-apama-jyakā})\end{tabular}}
}


\newglossaryentry{Sine_true_declination}
{
        name={Sine of the true declination},
        description={\tsans{sphu.ta-apama-jyakaa} (\textit{sphuṭa-apama-jyakā}); 
                    \tsans{spa.s.ta-apama-jyakaa} (\textit{spaṣṭa-apama-jyakạ}); 
                    \tsans{spa.s.ta-kraanti-jyakaa} (\textit{spaṣṭa-krānti-jyakā}) \\[5pt]
        $\mathcal{R}$\thinspace sine of the \protect\gls{true_declination}.\\[5pt]
        \Cf \begin{tabular}[t]{ll}\protect\gls{Sine_distance} &\tfarsi{جيب بعد} (\jayb\idafaconsonant\ \bud)\end{tabular}}
}

\newglossaryentry{Cosine_maximum_declination}
{
        name={Cosine of the maximum declination},
        description={\tfarsi{جیب تمام میل کلّی} (\jayb\idafaconsonant\ \tamam\idafaconsonant\ \mayl\idafaconsonant\ \kulli)\\[5pt]
        \tsans{parama-kraanti-ko.tijyaa} (\textit{parama-krānti-koṭijyā})\\[5pt]
        $\mathcal{R}$\thinspace cosine of the \protect\gls{maximum_declination}.}
} 

\newglossaryentry{table_Cosine_maximum_declination}
{
        name={table of the Cosine of the maximum declination},
        description={\tfarsi{جدول جيب تمام میل کلّی} (\jadval\idafaconsonant\ \jayb\idafaconsonant\ \tamam\idafaconsonant\ \mayl\idafaconsonant\ \kulli)\\[5pt]
        table of the product of the $\mathcal{R}$\thinspace cosine of the \protect\gls{maximum_declination} [\ie $\mathcal{R}$\thinspace cosine of the \textbf{obliquity of the ecliptic}] with sixty sexagesimal numbers from 1 to 60.}
} 

\newglossaryentry{Cosine_inverse_declination_degree_celestial_object}
{
        name={Cosine of the inverse declination of the degree of a celestial object},
        description={\tfarsi{جیب تمام میل منکوس درجه كوكب}  (\jayb\idafaconsonant\ \tamam\idafaconsonant\ \mayl\idafaconsonant\ \mankus\idafaconsonant\ \daraji\idafavowel\ \kawkab)\\[5pt]
        $\mathcal{R}$\thinspace cosine of the \protect\gls{inverse_declination}.\\[5pt]
        \Cf \begin{tabular}[t]{l}\protect\gls{day_Sine_increased_by_three_signs}\\ \tsans{sa-bha-traya-dyujiivaa} (\textit{sa-bha-traya-dyujīvā})\end{tabular}} 
}

\newglossaryentry{inverse_declination}
{
        name={inverse declination of the degree of a celestial object},
        description={\tfarsi{میل منکوس درجه كوكب} (\mayl\idafaconsonant\ \mankus\idafaconsonant\ \daraji\idafavowel\ \kawkab)\\[5pt]
        the first \protect\glslink{declination_degree}{declination of the degrees} (δ\textsubscript{1}) of a point on the ecliptic with longitude equal to the ecliptic longitude (λ\degree) of a celestial object increased by 90\degree, \ie δ\textsubscript{1}(λ\degree~+~90\degree).\footnote{The use of the term `inverse declination' (\almayl\ \almakus) to mean δ\textsubscript{1}(λ\degree\ + 90\degree) first appears in the works of thirteenth century \ce\ Mar\={a}gha astronomers: for example, \ZijIlkhani\ of \alTusi, \vid\ \textcite[188]{HamadanialTusi}, and \Tajalazyaj\ of \alMaghribi, \vid\ \textcite[196]{Dorce}.}}
}

\newglossaryentry{day_Sine_increased_by_three_signs}
{
        name= {day-Sine [of the longitude] increased by three zodiacal signs},
        description={\tsans{sa-bha-traya-dyujiivaa} (\textit{sa-bha-traya-dyujīvā}), \lit `the day-radius corresponding to an increase of three zodiacal signs'\\[5pt]
        the radius ($\mathcal{R}$\thinspace cosine) of the parallel of ecliptic declination (δ\textsubscript{1}) of a celestial object corresponding to its ecliptic longitude (λ\degree) increased by three zodiacal signs (90\degree), \ie $\mathcal{R}$\thinspace cos$\Big[$δ\textsubscript{1}(λ\degree + 90\degree)$\Big]$.\\[5pt]
        \Cf \begin{tabular}[t]{l}\protect\gls{Cosine_inverse_declination_degree_celestial_object} \\\tfarsi{جیب تمام میل منکوس درجه كوكب}  \\(\jayb\idafaconsonant\ \tamam\idafaconsonant\ \mayl\idafaconsonant\ \mankus\idafaconsonant\ \daraji\idafavowel\ \kawkab)\end{tabular}}   
}

\newglossaryentry{result}
{
        name={result},
        description={\tfarsi{حاصل} (\hasil)\\[5pt]
        the result of a mathematical operation, \eg protect\gls{product_multiplication} or \protect\gls{quotient_division}.}
}

\newglossaryentry{multiplication}
{
        name ={multiplication},
        description ={\begin{tabular}[t]{l l}
           to \textbf{multiply}  & \tfarsi{ضرب کردن} (\darb\ \kardan) 
                  \acrshort{infinitive}\\[5pt]
                  & √\tsans{ni-han} (√\textit{ni-han}) \acrshort{class}\textsubscript{2} \\[5pt]
                  & √\tsans{sam-han} (√\textit{sam-han}) \acrshort{class}\textsubscript{2} \\[5pt]
        \textbf{is multiplied} & \tsans{ni-hanyate} (\textit{ni-hanyate})
        \acrshort{present}-\acrshort{passive}-\acrshort{singular}$\cdot$\acrshort{third}    \\[5pt]
         \textbf{having been multiplied} & \tsans{hata} (\textit{hata}) \acrshort{past}-\acrshort{passive}-\acrshort{participle}\\[5pt]
         \textbf{to be multiplied} & \tsans{sa.mgu.nya} (\textit{saṃguṇya}) \acrshort{future}-\acrshort{passive}-\acrshort{participle}
        \end{tabular}}
}

\newglossaryentry{maximum_declination}
{
        name={maximum declination},
        description={\tfarsi{میل کلّی} (\mayl\idafaconsonant\ \kulli), \lit `total declination'\\[5pt]
        \tsans{parama-kraanti} (\textit{parama-kraanti}), \lit `greatest declination'\\[5pt]
        the arc of the solstitial colure lying between the ecliptic and the celestial equator; also known as the \textbf{obliquity of the ecliptic}. \\[5pt]
        The \ZijiShahJahani, following the \ZijUlughBeg, and the \Siddhantasindhu\ use an ecliptic obliquity of 23;\thinspace 30,\thinspace17 degrees.}
}        
        
\newglossaryentry{division}
{
        name={division},
        description={\begin{tabular}[t]{l l}
            to \textbf{divide}  & \tfarsi{قسمت کردن} (\qismat\ \kardan) \acrshort{infinitive}\\[5pt]
            & √\tsans{aap} (√\textit{aap}) \acrshort{class}$_\text{5}$\\[5pt]
             & √\tsans{bhaj} (√\textit{bhaj}) \acrshort{class}$_\text{1}$\\[5pt]
            [one] \textbf{should divide} &\tsans{bhajet} (\textit{bhajet}) \acrshort{optative}-\acrshort{active}-\acrshort{singular}$\cdot$\acrshort{third}\\[5pt]
            \textbf{having been divided} (causal meaning) & \tsans{bhaajita} (\textit{bhājita}) \acrshort{causative}-\acrshort{past}-\acrshort{passive}-\acrshort{participle}\\[5pt]
            \textbf{having been divided} &
            \tsans{aapta} (\textit{āpta}) \acrshort{past}-\acrshort{passive}-\acrshort{participle}
        \end{tabular}} 
}     

\newglossaryentry{quotient_division}
{
        name={quotient of division},
        description={\tfarsi{خارج قسمت} (\kharij\idafaconsonant\ \qismat)\\[5pt]
        \Cf \begin{tabular}[t]{ll}
        \protect\gls{obtained} & \tsans{labdha} (\textit{labdha})
        \end{tabular}}
}

\newglossaryentry{obtained}
{
        name={obtained},
        description={\tsans{labdha} (\textit{labdha}), \lit `what is obtained or acquired'\\[5pt]
        \textit{tacitly}, the quotient of the division.\\[5pt] 
        \Cf \begin{tabular}[t]{ll}
        \protect\gls{quotient_division} & \tfarsi{خارج قسمت} (\kharij\idafaconsonant\ \qismat)
        \end{tabular}}
}

\newglossaryentry{table}
{
        name={table},
        description={\tfarsi{جدول} (\jadval)\\[5pt]
        \tsans{ko.s.thaka} (\textit{koṣṭhaka}) [\textit{varia lectio}: \tsans{ko.s.thika} (\textit{koṣṭhika})]\\[5pt]
        astronomical or mathematical table.},
        plural={tables}
}        

\newglossaryentry{declination_degree}
{
        name={declination of its degree},
        description={\tfarsi{میل درجه او} (\mayl\idafaconsonant\ \daraji\idafavowel\ \uy) \\[5pt]
        Aliases: \begin{tabular}[t]{ll}
                \textbf{declination of the degree} & \tfarsi{میل درجه كوكب}\\
                  \textbf{of a celestial object} &     (\mayl\idafaconsonant\ \daraji\idafavowel\ \kawkab)\\[5pt]
                \textbf{declination} &\tfarsi{میل} (\mayl)\\[5pt] &\tsans{kraanti} (\textit{krānti})\\[5pt]
                \textbf{declination of a celestial object} & \tsans{khagasya kraanti} (\textit{khagasya krānti})
                \end{tabular}\\[5pt]
          the \textit{first} declination (δ\textsubscript{1}) of a point on the ecliptic with ecliptic longitude (λ\degree), measured in degrees, corresponding to the position of a celestial object; in other words, the arc of the great circle passing through the two celestial poles and a point on the ecliptic (with longitude λ\degree), and lying between the ecliptic and the celestial equator.}
}

\newglossaryentry{degree_celestial_object}
{
        name={degree of a celestial object},
        description={\tfarsi{درجه کوکب} (\daraji\idafavowel\ \kawkab) \\[5pt]
        Alias: \begin{tabular}[t]{ll}
        \textbf{its degree} & \tfarsi{درجه او} (\daraji\idafavowel\ \uy)
                \end{tabular}\\[5pt]
        the ecliptic longitude (λ\degree) of a celestial object measured in degrees.\\[5pt]
        \Cf\begin{tabular}[t]{ll}
        \protect\gls{arc_ecliptic_longitude_celestial_object}\\
        \tsans{khagasya bhujaa} (\textit{khagasya bhujā})\\[5pt]
        \protect\gls{Sine_distance_celestial_object_nearest_solstice}\\
        \tfarsi{جيب بعد درجه کوکب از انقلاب اقرب} \\(\jayb\idafaconsonant\ \bud\idafaconsonant\ \daraji\idafavowel\ \kawkab\ \az\ \inqilab\idafaconsonant\ \aqrab)
        \end{tabular}}
}



\newglossaryentry{Sine_latitude}
{
        name={Sine of its latitude},
        description={\tfarsi{جيب عرض او} (\jayb\idafaconsonant\ \ard\idafaconsonant\ \uy)\\[5pt]
        Aliases: \begin{tabular}[t]{l}
             \textbf{Sine of the latitude of a celestial object} 
             \\ \tfarsi{جیب عرض کوکب} (\jayb\idafaconsonant\ \ard\idafaconsonant\ \kawkab)\\[5pt]
             \textbf{Sine of the latitude} \enskip  \tsans{baa.na-jyaa}  (\textit{bāṇa-jyā})\\[5pt]
             \textbf{Sine of the latitude of a celestial object} \\ \tsans{nabhoga-vi"sikhasya si}\tsnb{ञ्जि}\tsans{nii} (\textit{nabhoga-viśikhasya siñjinii})
        \end{tabular}\\[5pt]
        $\mathcal{R}$\thinspace sine of the \protect\gls{latitude_celestial_object}.}
}

\newglossaryentry{direction_latitude}
{
        name={direction of the latitude},
        description={\tfarsi{جهت عرض} (\jahat\idafaconsonant\ \ard)\\[5pt]
        \tsans{baa.na-di"s} (\textit{bāṇa-diś})\\[5pt]
        Alias: \begin{tabular}[t]{l}
     \textbf{direction of the latitude of a celestial object} \\ \tfarsi{جهت عرض کوکب} (\jahat\idafaconsonant\ \ard\idafaconsonant\ \kawkab)
     \end{tabular}\\[5pt]
     the orientation of the \protect\gls{latitude_celestial_object} towards the north or the south ecliptic pole with respect to the ecliptic.}
}

\newglossaryentry{Sine_distance_celestial_object_nearest_solstice}
{
        name={Sine of the distance of the degree of a celestial object from the nearest solstice},
        description={\tfarsi{جيب بعد درجه کوکب از انقلاب اقرب} (\jayb\idafaconsonant\ \bud\idafaconsonant\ \daraji\idafavowel\ \kawkab\ \az\ \inqilab\idafaconsonant\ \aqrab)\\[5pt]
        $\mathcal{R}$\thinspace sine of the \protect\gls{distance_degree_celestial_object_nearest_solstice}; equal to the $\mathcal{R}$\thinspace cosine of the \protect\gls{distance_degree_celestial_object_nearest_equinox}.\\[5pt]
        \Cf \begin{tabular}[t]{l}
        \protect\gls{Sine_koti_celestial_object} \\
        \tsans{khagasya ko.ti-si}\tsnb{ञ्जि}\tsans{nii} (\textit{khagasya koṭi-siñjini})
        \end{tabular}}
}

\newglossaryentry{Cosine_latitude_celestial_object}
{
        name={Cosine of the latitude of a celestial object},
        description={\tfarsi{جیب تمام عرض کوکب} (\jayb\idafaconsonant\ \tamam\idafaconsonant\ \ard\idafaconsonant\ \kawkab)\\[5pt]
         Alias: \begin{tabular}[t]{ll}
        \textbf{Cosine of its latitude} &\tsans{sva-baa.na-ko.tijiivaa} (\textit{sva-bāṇa-koṭijīvā})
        \end{tabular}\\[5pt]
        $\mathcal{R}$\thinspace cosine of the \protect\gls{latitude_celestial_object}.}
}

\newglossaryentry{Sine_congruent_koti}
{
        name={Sine of the congruent complementary arc},
        description={ \tsans{sad.rk.sa-ko.ti-si}\tsnb{ञ्जि}\tsans{nii} (\textit{sadṛkṣa-koti-siñjinī})\\[5pt]
        $\mathcal{R}$\thinspace sine of the \protect\gls{congruent_koti}.\\[5pt]
        \Cf\begin{tabular}[t]{l}
        \protect\gls{Sine_distance_celestial_object_solstitial_colure} \\
        \tfarsi{جیب بعد کوکب از
        \tfarsib{دایرهٔ ماره باقطاب اربعه}} \\
        (\jayb\idafaconsonant\ \bud\idafaconsonant\ \kawkab\ \az\ \guillemotleft\dayiri\idafavowel\ \marri\ \biaqtab\idafaconsonant\ \arbai\guillemotright)
        \end{tabular}}
}

\newglossaryentry{Sine_distance_celestial_object_solstitial_colure}
{
        name={Sine of the distance of a celestial object from the `circle passing through the four poles'},
        description={\vspace{-\baselineskip}\tfarsi{جیب بعد کوکب از
        \tfarsib{دایرهٔ ماره باقطاب اربعه}} (\jayb\idafaconsonant\ \bud\idafaconsonant\ \kawkab\ \az\ \guillemotleft\dayiri\idafavowel\ \marri\ \biaqtab\idafaconsonant\ \arbai\guillemotright)\\[5pt]
        $\mathcal{R}$\thinspace sine of the \protect\gls{distance_celestial_object_solstice}.\\[5pt]
        \Cf\begin{tabular}[t]{l}
            \protect\gls{Sine_congruent_koti} \\  \tsans{sad.rk.sa-ko.ti-si}\tsnb{ञ्जि}\tsans{nii} (\textit{sadṛkṣa-koti-siñjinī})
        \end{tabular}}
}

\newglossaryentry{low_division}
{
        name={low-division},
        description={\begin{tabular}[t]{ll}
         to \textbf{low divide}  & \tfarsi{منحطّ قسمت کردن } (\munhatt\idafaconsonant\ \qismat\ \kardan) \acrshort{infinitive}
        \end{tabular} \\[5pt]
        division of sexagesimal numbers, with the divisor first divided by 60 (Radius); in other words, \textit{lowering} the sexagesimal value of the divisor by first shifting its fractional point leftwards.\\[5pt]
        \Cf\begin{tabular}[l]{lll}
              \protect\gls{lowering}& to \textbf{lower} &\tsans{adharii-}√\tsans{k.r} (\textit{adharī-√kṛ}) \acrshort{class}\textsubscript{8} [\acrshort{infinitive}]
        \end{tabular}}
}

\newglossaryentry{Cosine_distance_solstitial_colure}
{
        name={Cosine of the distance of a celestial object from the `circle passing through the four poles'},
        description={\vspace{-\baselineskip}\tfarsi{جیب تمام بعد از
        \tfarsib{دایرهٔ ماره باقطاب اربعه}} (\jayb\idafaconsonant\ \tamam\idafaconsonant\ \bud\ \az\ \guillemotleft\dayiri\idafavowel\ \marri\ \biaqtab\idafaconsonant\ \arbai\guillemotright)\\[5pt]
        $\mathcal{R}$\thinspace cosine of the \protect\gls{distance_celestial_object_solstice}.}
}

\newglossaryentry{arc}
{
        name={arc},
        description={\tfarsi{قوس} (\qaws)\\[5pt]
        \tsans{dhanus} (\textit{dhanus}); \tsans{koda.n.da} (\textit{kodaṇḍa}); \tsans{caapa} (\textit{cāpa})}
}

\newglossaryentry{table_of_Sine}
{
        name={table of Sine},
        description={\tfarsi{جدول جيب} (\jadval\idafaconsonant\ \jayb)\\[5pt]
        table of $\mathcal{R}$\thinspace sine values.\\[5pt]
        The \ZijiShahJahani\ and the \Siddhantasindhu\ tabulate the Sine values for every minute of arc from 0;\thinspace 0 degree to 360;\thinspace 0 degrees. The maximum value of the Sine (\textit{sinus totus}), identified with the Radius ($\mathcal{R}$), is 60;\thinspace 0.}
}

\newglossaryentry{first_arc}
{
        name={first arc},
        description={\tfarsi{قوس اوّل} (\qaws\idafaconsonant\ \avval)\\[5pt]
        the arc of the solstitial colure between the ecliptic and the great circle passing through the two equinoctial points and a celestial object [\ie the \protect\gls{circle_congruent_ecliptic}]. \Vid\ \S\thinspace\ref{zijshahjahan_persian_english}:~{\footnotesize \P}\thinspace(9).\\[5pt]
        \Cf \begin{tabular}[t]{ll}
          \protect\gls{maximum_latitude}  &  \tsans{para-i.su} (\textit{para-iṣu})
            \end{tabular}
            }
}

\newglossaryentry{one_quarter}
{
        name={one-quarter},
        description={\tfarsi{ربع} (\rub)\\[5pt]
        one-fourth or a quarter of a circle, \ie 90\degree.\\[5pt]
        \Cf\begin{tabular}[t]{ll}
        \protect\gls{ninety} & \tsans{abhra-nava} (\textit{abhra-nava})
        \end{tabular}}
}

\newglossaryentry{exceeds}
{
        name={exceeds},
        description={\begin{tabular}[t]{ll}
             to \textbf{exceed} &  \tfarsi{زیادی شدن} (\ziyadi\ \shudan) \acrshort{infinitive}
                    \end{tabular}\\[5pt]
            \Cf\begin{tabular}[t]{ll}
                 \protect\gls{greater}& \tsans{adhika} (\textit{adhika})  
            \end{tabular}}
}

\newglossaryentry{whole_sum}
{
        name={whole sum},
        description={\tfarsi{تمام مجموع} (\tamam\idafaconsonant\ \majmu)\\[5pt]
        \textit{tacitly}, the additional amount added to the sum (of arcs) to bring the total up to a one-quarter (90\degree), one-half (180\degree), three-quarters (270\degree), or one-whole (360\degree) turn of a circle.}
}

\newglossaryentry{one_half}
{
        name={one-half},
        description={\tfarsi{نصف دور} (\nisf)\\[5pt]
        one-half of a circle, \ie 180\degree.\\[5pt]
        \Cf \begin{tabular}[t]{ll}
           \protect\gls{one_hundred_eighty} & \tsans{kha-a.s.ta-bhuu} (\textit{akha-aṣṭa-bhū})
        \end{tabular}}
}

\newglossaryentry{different_directions}
{
        name={different directions},
        description={\tfarsi{جهت مختلف} (\jahat\idafaconsonant\ \mukhtalif)\\[5pt]
        the opposing orientation of two arc of a great circle in two antipodal directions.\\[5pt]
        \Cf\begin{tabular}[t]{ll}
           \protect\gls{direction_residue} & \tfarsi{جهت فضل} (\jahat\idafaconsonant\ \fadla) 
        \end{tabular}}
}

\newglossaryentry{second_arc}
{
        name={second arc},
        description={\tfarsi{قوس دوم} (\qaws\idafaconsonant\ \duvum)\\[5pt]
        the arc of the solstitial colure between the celestial equator and the great circle passing through the two equinoctial points and a celestial object [\ie the \protect\gls{circle_congruent_ecliptic}]. \Vid\ \S\thinspace\ref{zijshahjahan_persian_english}:~{\footnotesize \P}\thinspace(10).\\[5pt]
        \Cf \begin{tabular}[t]{ll}
          \protect\gls{maximum_true_declination}  &  \tsans{para-sphu.ta-apama} (\textit{para-sphuṭa-apama})
            \end{tabular}}
} 

\newglossaryentry{Sine_second_arc}
{
        name={Sine of the second arc},
        description={\tfarsi{جیب قوس دوم} (\jayb\idafaconsonant\ \qaws\idafaconsonant\ \duvum)\\[5pt]
        $\mathcal{R}$\thinspace sine of the \protect\gls{second_arc}.\\[5pt]
        \Cf \begin{tabular}[t]{l}
        \protect\gls{Sine_maximum_true_declination} \\
        \tsans{para-sphu.ta-kraanti-bhava-jyakaa} (\textit{para-sphuṭa-krānti-bhava-jyakā})
        \end{tabular}}
}

\newglossaryentry{direction_second_arc}
{
        name={direction of the second arc},
        description={\tfarsi{جهت قوس دوم} (\jahat\idafaconsonant\ \qaws\idafaconsonant\ \duvum)\\[5pt]
        the orientation of the \protect\gls{second_arc} towards the north or the south celestial pole with respect to the celestial equator.}
}


\newglossaryentry{true_declination}
{
        name={true declination},
        description={\tsans{spa.s.ta-kraanti} (\textit{spaṣṭa-krānti}); \tsans{sphu.ta-apama} (\textit{sphuṭa-apama})\\[5pt]
        the arc of the great circle passing through the two celestial poles and a celestial object, and lying between the celestial object and the celestial equator.\\[5pt]
         \Cf \begin{tabular}[t]{l}\protect\gls{distance_celestial_object}\\ \tfarsi{بعد کوکب} (\bud\idafaconsonant\ \kawkab)\end{tabular}}
}

\newglossaryentry{other_declination}
{
        name={other declination},
        description={\tsans{anyatara-apama} (\textit{anyatara-apama}) \\[5pt]
        the \textit{second} declination (δ\textsubscript{2}) of a point on the ecliptic with
        ecliptic longitude (λ\degree), measured in degrees, corresponding to the position of a celestial object; in other words, the arc of the great circle passing through the two ecliptic poles and a point on the ecliptic (with longitude λ\degree), and lying between the ecliptic and the celestial equator.\\[5pt]
        \Cf \begin{tabular}[t]{ll}\protect\gls{second_declination_degree} &\tfarsi{میل ثاني درجه} (\mayl\idafaconsonant\ \thani\idafavowel\ \daraji)\end{tabular}}
}

\newglossaryentry{curve_true_declination}
{
        name={curve of true declination},
        description={\tsans{sphu.ta-apama-a"nka} (\textit{sphuṭa-apama-aṅka});             \tsans{spa.s.ta-kraanti-a"nka} (\textit{spaṣṭa-krānti-aṅka})\\[5pt]
        Alias: \begin{tabular}[t]{ll}
               \textbf{share of the true declination} &\tsans{sphu.ta-apama-a.m"sa} (\textit{sphuṭa- apama-aṃśa})
                \end{tabular}\\[5pt]
        the arc of the great circle passing through the two ecliptic poles and a celestial object, and lying between a celestial object and the celestial equator. \Vid\ \S\thinspace\ref{siddhantasindhu_sanskrit_english}:~{\footnotesize \P}\thinspace(1).\\[5pt]
         \Cf \begin{tabular}[t]{ll}
         \protect\gls{argument_of_distance} & \tfarsi{حصّهٔ بعد} (\hissi\idafavowel\ \bud)
         \end{tabular}%
         }
}

\newglossaryentry{own_direction}
{
        name={own direction},
        description={\tsans{sva-di"s} (\textit{sva-diś})\\[5pt]
        \textit{tacitly}, the \protect\gls{direction_sum} or the \protect\gls{direction_residue}, according to orientation of a celestial object. \Vid\ \S\thinspace\ref{siddhantasindhu_sanskrit_english}:~{\footnotesize \P}\thinspace(1).\\[5pt]
        \Cf \begin{tabular}[t]{l}
           \protect\gls{conjunction_disjunction_direction}  \\ \tsans{yuti-viyoga-di"s} (\textit{yuti-viyoga-diś}) 
        \end{tabular}}
}

\newglossaryentry{product_multiplication}
{
        name={product of multiplication},
        description={\tsans{gu.nita-phala} (\textit{guṇita-phala})\\[5pt]
        Alias:\begin{tabular}[t]{ll}
         \textbf{result of multiplication} &   \tfarsi{حاصل ضرب} (\hasil\idafaconsonant\ \darb)  \end{tabular}}
}

\newglossaryentry{result_multiplication_division}
{
        name={result of multiplication and division},
        description={ \tsans{gu.nana-bhajana-phala} (\textit{guṇana-bhajana-phala})\\[5pt]
        \Cf:\begin{tabular}[t]{ll}
         \protect\gls{product_multiplication} &  \tsans{gu.nita-phala} (\textit{guṇita-phala}\\[5pt]
         \protect\gls{obtained} & \tsans{labdha} (\textit{labdha})
         \end{tabular}}
}

\newglossaryentry{extract}
{
        name={extract},
        description={\begin{tabular}[t]{ll}
             to \textbf{extract} &   \tfarsi{درآوردن}  (\daravardan) \acrshort{infinitive}\\
             &\textsc{variant}:  \tfarsi{درآردن} (\darardan), \lit `to bring out'
        \end{tabular}\\[5pt]
        to obtain, extract, or compute a numerical value from a \protect\gls{table}.}
}

\newglossaryentry{circle}
{
        name={circle},
        description={\tfarsi{دایره} (\textit{dāyiri})\\[5pt]
        \tsans{v.rtta} (\textit{vṛtta})}
}

\newglossaryentry{ecliptic_pole}
{
        name={ecliptic pole},
        description={\tsans{kadamba} (\textit{kadamba})}
}

\newglossaryentry{celestial_pole}
{
        name={celestial pole},
        description={\tsans{vi.suvat-dhruva} (\textit{viṣuvat-dhruva})\\[5pt] MS~Kh attests an irregular (vernacular?) form \tsans{vi.sava-wholedhruva} (\textit{viṣava-dhruva}), \vid\ \S\thinspace\ref{siddhantasindhu_sanskrit_english}:~footnote~\ref{vishava_deviant}.}
}

\newglossaryentry{solstitial_colure}
{
        name={solstitial colure},
        description={\begin{tabular}[t]{ll}
                \textbf{circle belonging to the solstice} &\tsans{aayana-v.rtta} (\textit{āyana-vṛtta})\\[5pt]
                \textbf{circle passing through the four poles} &
                \tfarsi{دایرهٔ ماره باقطاب اربعه} \\
                & (\textit{\dayiri\idafavowel\ \marri\ \biaqtab\idafaconsonant\ \arbai})\\[5pt]
                &\tsans{dhruva-catu.ska-yaata-v.rtta}\\
                &(\textit{dhruva-catuṣka-yāta-vṛtta})
            \end{tabular}\\[5pt]
            a great circle of the celestial sphere passing through the two ecliptic poles and two celestial poles, and marking the solstitial points at its intersection with the ecliptic.}
}

\newglossaryentry{well_rounded}
{
        name={well rounded},
        description={\tsans{su-v.rtta} (\textit{su-vṛtta})\\[5pt]
        a great circle (orthodrome) of a sphere, \ie a circle on a sphere that is concentric with the centre of the sphere and passes through two antipodal points on the sphere's surface.}
}

\newglossaryentry{pair_equinoctial_points}
{
        name={pair of equinoctial points},
        description={\tsans{vi.suvat-dvaya} (\textit{viṣuvat-dvaya})\\[5pt]
        the pair of vernal (\textit{mahā-viṣuva}) and autumnal (\textit{jala-viṣuva}) equinoctial points on the ecliptic (or celestial equator).\\[5pt] MS~Kh attests an irregular (vernacular?) form \tsans{vi.sava-dvayo\selip} (\textit{viṣava-dvayo}\dots), \vid\ \S\thinspace\ref{siddhantasindhu_sanskrit_english}:~footnote~\ref{vishava_deviant}.}
}

\newglossaryentry{circle_congruent_ecliptic}
{
        name={circle congruent to the ecliptic},
        description={\tsans{bhacakra-sad.r"sa-v.rtta} (\textit{bhacakra-sadṛśa-vṛtta}), \lit `circle resembling the ecliptic'; \tsans{bhacakra-sad.rk.sa-v.rtta} (\textit{bhacakra-sadṛkṣa-vṛtta}); \tsans{bhav.rtta-sad.r"sa-v.rtta} (\textit{bhavṛtta-sadṛśa-vṛtta})\\[5pt]
        a great circle passing through the two equinoctial points and a celestial object, and resembling the ecliptic, \ie having its degrees of arc measured from the vernal equinoctial point  (\textit{meṣādi} `first point of Aries' 0\degree). \Vid\ \S\thinspace\ref{siddhantasindhu_sanskrit_english}:~{\footnotesize \P}\thinspace(α).
        }
}

\newglossaryentry{knower_spheres}
{
        name={knower of spheres},
        description={\tsans{gola-vid} (\textit{gola-vid})\\[5pt]
        astronomers/mathematicians who know the `science of the spheres'.} 
}

\newglossaryentry{maximum_true_declination}
{
        name={maximum true declination},
        description={\tsans{para-sphu.ta-apama} (\textit{para-sphuṭa-apama})\\[5pt]
        Alias:\begin{tabular}[t]{l}
           \textbf{maximum true declination of a celestial object} \\  \tsans{grahasya para-sphu.ta-apama} (\textit{grahasya para-sphuṭa-apama})
        \end{tabular}\\[5pt]
        the arc of the solstitial colure between the celestial equator and the \protect\gls{circle_congruent_ecliptic} [\ie the great circle passing through the two equinoctial points and a celestial object]. \Vid\ \S\thinspace\ref{siddhantasindhu_sanskrit_english}:~{\footnotesize \P}\thinspace(β).\\[5pt]
        \Cf \begin{tabular}[t]{ll}
          \protect\gls{second_arc}  &  \tfarsi{قوس دوم} (\qaws\idafaconsonant\ \duvum)
            \end{tabular}}
} 

\newglossaryentry{Sine_maximum_true_declination}
{
        name={Sine of the maximum true declination},
        description={\tsans{para-sphu.ta-kraanti-bhava-jyakaa} (\textit{para-sphuṭa-krānti-bhava-jyakā}), \lit `the maximum true declination becoming or turned into its Sine' \\[5pt]
        $\mathcal{R}$\thinspace sine of the \protect\gls{maximum_true_declination}.\\[5pt]
        \Cf\begin{tabular}[t]{ll}
            \protect\gls{Sine_second_arc} &  \tfarsi{جیب قوس دوم} (\jayb\idafaconsonant\ \qaws\idafaconsonant\ \duvum)
            \end{tabular}}
}

\newglossaryentry{maximum_latitude}
{
        name={maximum latitude},
        description={\tsans{para-i.su} (\textit{para-iṣu}); \tsans{para-"sara} (\textit{para-śara})\\[5pt]
        the arc of the solstitial colure between the ecliptic and the \protect\gls{circle_congruent_ecliptic} [\ie the great circle passing through the two equinoctial points and a celestial object]. \Vid\ \S\thinspace\ref{siddhantasindhu_sanskrit_english}:~{\footnotesize \P}\thinspace(γ).\\[5pt]
        \Cf \begin{tabular}[t]{ll}
          \protect\gls{first_arc}  &  \tfarsi{قوس اوّل} (\qaws\idafaconsonant\ \avval)
            \end{tabular}}
}        

\newglossaryentry{conjunction_equinox_node}
{
        name={conjunction of the equinoctial point and the node of the orbit of a celestial object},
        description={\vspace{-\baselineskip}\tsans{vi.suva-paata-yuga} (\textit{viṣuva-pāta-yuga}), \lit `coupling of the equinoctial point and the node'\\[5pt]
        the coincidence of the nodes of a celestial object's orbit (typically considered to be inclined to the ecliptic) with the two equinoctial points; in order words, the position of the orbit of a celestial object with the longitude of its ascending (or descending) node being 0\degree.}
}

\newglossaryentry{equinoctial_point}
{
        name={equinoctial point},
        description={\tsans{vi.suvat} (\textit{viṣuvat}). \\[5pt]
        MS~Kh attests an irregular (vernacular?) form \tsans{vi.savat} (\textit{viṣavat}), \vid\ \S\thinspace\ref{siddhantasindhu_sanskrit_english}:~footnote~\ref{vishava_deviant}.}
}

\newglossaryentry{congruent_bhuja}
{
        name={congruent arc},
        description={\tsans{sad.r"s-bhujaa} (\textit{sadṛś-bhujā}); \tsans{sad.rk.sa-baahu} (\textit{sadṛkṣa-bāhu})\\[5pt]
        the arc of the \protect\gls{circle_congruent_ecliptic} that lies between the equinoctial point and the celestial object. \Vid\ \S\thinspace\ref{siddhantasindhu_sanskrit_english}:~{\footnotesize \P}\thinspace(δ).\\[5pt]
        \textsc{remark}:  More generally, \tsans{bhujaa} (\textit{bhujā}) [\lit `hand/arm', aliases: \tsans{baahu} (\textit{bāhu}) or \tsans{dos} (\textit{dos})] of an angle is a technical term in Sanskrit mathematical sciences denoting the amount of degrees \textit{already elapsed in odd quadrants} and the amount of degrees \textit{to be elapsed in even quadrants} of any circle.}
}

\newglossaryentry{arc_ecliptic_longitude_celestial_object}
{
        name={arc of ecliptic longitude of a celestial object},
        description={\tsans{khagasya bhujaa} (\textit{khagasya bhujā})\\[5pt]
        the arc of ecliptic longitude of a celestial object, taken as a \textit{bhujā} measure, \vid\ \textsc{remark} in \protect\gls{congruent_bhuja}.\\[5pt]
        \Cf \begin{tabular}[t]{l}
          \protect\gls{distance_degree_celestial_object_nearest_equinox} \\
          \tfarsi{بعد درجه کوکب از اعتدال اقرب} \\
        (\textit{\bud\idafaconsonant\ \daraji\idafavowel\ \kawkab\ \az\ \itidal\idafaconsonant\ \aqrab})\\[5pt]
        \protect\gls{degree_celestial_object}\enskip  \tfarsi{درجه کوکب} (\daraji\idafavowel\ \kawkab)
        \end{tabular}}
}        
        
\newglossaryentry{distance_degree_celestial_object_nearest_equinox}
{   
        name={distance of the degree of a celestial object from the nearest equinox},
        description={\tfarsi{بعد درجه کوکب از اعتدال اقرب} (\textit{\bud\idafaconsonant\ \daraji\idafavowel\ \kawkab\ \az\ \itidal\idafaconsonant\ \aqrab})\\[5pt]
        the arc of the ecliptic between the celestial object and the nearest equinoctial points (`first point of Aries' 0\degree\ or `first point of Libra' 180\degree).\\[5pt]
        \Cf \begin{tabular}[t]{l}
        \protect\gls{arc_ecliptic_longitude_celestial_object} \\\tsans{khagasya bhujaa} (\textit{khagasya bhujā})\\[5pt]
        \protect\gls{degree_celestial_object}\enskip  \tfarsi{درجه کوکب} (\daraji\idafavowel\ \kawkab) 
        \end{tabular}}
}

\newglossaryentry{congruent_koti}
{
        name={congruent complementary arc},
        description={\tsans{sad.r"s-ko.ti} (\textit{sadṛś-koṭi})\\[5pt]
        complement of the \protect\gls{congruent_bhuja} to 90\degree, in other words, the arc of the \protect\gls{circle_congruent_ecliptic} lying between the celestial object and the solstital~colure. \Vid\ \S\thinspace\ref{siddhantasindhu_sanskrit_english}:~{\footnotesize \P}\thinspace(δ).\\[5pt]
       \textsc{remark}: More generally, \tsans{ko.ti} (\textit{koṭi}) [\lit `extremity'] of an angle is a technical term in Sanskrit mathematical sciences denoting the amount of degrees \textit{to be elapsed in odd quadrants} and the amount of degrees \textit{already elapsed in even quadrants} of any circle.\\[5pt]
       \Cf\begin{tabular}[t]{l}
         \protect\gls{distance_celestial_object_solstice}  \\
        \tfarsi{بعد کوکب از
        \tfarsib{دایرهٔ ماره باقطاب اربعه}} \\
        (\textit{\bud\idafaconsonant\ \kawkab\ \az\ \guillemotleft\dayiri\idafavowel\ \marri\ \biaqtab\idafaconsonant\ \arbai\guillemotright})
       \end{tabular}}
}

\newglossaryentry{distance_celestial_object_solstice}
{
        name={distance of a celestial object from the ``circle passing through the four poles"},
        description={\tfarsi{بعد کوکب از
        \tfarsib{دایرهٔ ماره باقطاب اربعه}} (\textit{\bud\idafaconsonant\ \kawkab\ \az\ \guillemotleft\dayiri\idafavowel\ \marri\ \biaqtab\idafaconsonant\ \arbai\guillemotright})\\[5pt]
        the angular distance of a celestial object from the solstitial colure, measured along a great circle passing through the two equinoctial points and the celestial object [\ie, \protect\gls{circle_congruent_ecliptic}].\\[5pt]
        \Cf\begin{tabular}[t]{ll}
         \protect\gls{congruent_koti} & \tsans{sad.r"s-ko.ti} (\textit{sadṛś-koṭi}) 
        \end{tabular}}
}

\newglossaryentry{arc_complementary_ecliptic_longitude_celestial_object}
{
        name={complement of the arc of ecliptic longitude of a celestial object},
        description={\tsans{khagasya ko.ti} (\textit{khagasya koṭi})\\[5pt]
        the arc of ecliptic longitude of a celestial object, taken as a \textit{koṭi} measure, \vid\ \textsc{remark} in \protect\gls{congruent_koti}.\\[5pt]
        \Cf \begin{tabular}[t]{l}
          \protect\gls{distance_degree_celestial_object_nearest_solstice} \\
          \tfarsi{بعد درجه کوکب از انقلاب اقرب} \\
        (\textit{\bud\idafaconsonant\ \daraji\idafavowel\ \kawkab\ \az\ \inqilab\idafaconsonant\ \aqrab})
        \end{tabular}}
}        
        
\newglossaryentry{distance_degree_celestial_object_nearest_solstice}
{
        name={distance of the degree of a celestial object from the nearest solstice},
        description={\tfarsi{بعد درجه کوکب از انقلاب اقرب}  (\textit{\bud\idafaconsonant\ \daraji\idafavowel\ \kawkab\ \az\ \inqilab\idafaconsonant\ \aqrab})\\[5pt]
        the arc of the ecliptic between the celestial object and the nearest solstitial points (`first point of Capricorn' 90\degree\ or `first point of Cancer' 270\degree).\\[5pt]
        \Cf\begin{tabular}[t]{l}
           \protect\gls{arc_complementary_ecliptic_longitude_celestial_object}  \\
           \tsans{khagasya ko.ti} (\textit{khagasya koṭi})
        \end{tabular}}
}

\newglossaryentry{Sine_koti_celestial_object}
{
        name={Sine of the complement of the arc of ecliptic longitude of a celestial object},
        description={\tsans{khagasya ko.ti-si}\tsnb{ञ्जि}\tsans{nii} (\textit{khagasya koṭi-siñjini})\\[5pt]
        $\mathcal{R}$\thinspace sine of the \protect\gls{arc_complementary_ecliptic_longitude_celestial_object}; equal to the $\mathcal{R}$\thinspace cosine of the \protect\gls{arc_ecliptic_longitude_celestial_object}.\\[5pt]
        \Cf\begin{tabular}[t]{l}
        \protect\gls{Sine_distance_celestial_object_nearest_solstice}\\
        \tfarsi{جيب بعد درجه کوکب از انقلاب اقرب} \\(\jayb\idafaconsonant\ \bud\idafaconsonant\ \daraji\idafavowel\ \kawkab\ \az\ \inqilab\idafaconsonant\ \aqrab)
        \end{tabular}}
}

\newglossaryentry{reducing_from_ninety}
{
        name={complement of an arc to ninety degrees},
        description={\begin{tabular}[t]{ll}
            \textbf{having been reduced from ninety}& \tsans{navatita"scyuta} (\textit{navatitaś-cyuta})\\
             & \acrshort{past}-\acrshort{passive}-\acrshort{participle}\\
             \end{tabular}\\[5pt]
            the measure of arc (in degrees) deviated/deprived from 90\degree; in other words, complement of the measure of an arc to 90\degree.}
}

\newglossaryentry{lowered_Sine_congruent_base}
{
        name={lowered Sine of the congruent arc},
        description={\tsans{adhara-sad.rk.sa-dorjyaa} (\textit{adhara-sadrkṣa-dorjyā})\\[5pt]
        $\mathcal{R}$\thinspace sine of \protect\glslink{congruent_bhuja}{congruent arc} divided by 60 (Radius).}
}

\newglossaryentry{Sine_congruent_base}
{
        name={Sine of the congruent arc},
        description={\tsans{sad.rk.sa-baahu-jyakaa} (\textit{sadrkṣa-bāhu-jyakā})\\[5pt]
        $\mathcal{R}$\thinspace sine of \protect\glslink{congruent_bhuja}{congruent arc}.}
}

\newglossaryentry{celestial_hemisphere}
{
        name={celestial hemisphere},
        description={\tsans{gola} (\textit{gola})\\[5pt]
        \textit{tacitly}, the declination or orientation of a celestial object in the northern or southern halves of the celestial sphere.}
}

\newglossaryentry{same_different_directions}
{
        name={same or different directions},
        description={\tsans{sama-bhinna-di"s} (\textit{sama-bhinna-diś})\\[5pt]
        \protect\gls{one_direction} (\ie \protect\gls{direction_sum}) or the \protect\gls{different_directions} (\ie \protect\gls{direction_residue}).}     
}

\newglossaryentry{same}
{
        name={same},
        description={\tsans{sama} (\textit{sama}), \lit `equal, identical, or even'\\[5pt]
        \textit{tacitly}, similarly oriented.}
}

\newglossaryentry{conjunction_disjunction_direction}
{
        name={direction of the conjunction or the disjunction},
        description={\tsans{yuti-viyoga-di"s} (\textit{yuti-viyoga-diś}); \tsans{sa.myoga-viyoga-di"s} (\textit{saṃyoga-viyoga-diś}) \\[5pt]
        the \protect\gls{direction_sum} (direction of conjunction) or the \protect\gls{direction_residue} (direction of disjunction). \Vid\ \S\thinspace\ref{siddhantasindhu_sanskrit_english}:~{\footnotesize \P}\thinspace(10).\\[5pt]
     \Cf \begin{tabular}[t]{ll}
           \protect\gls{own_direction}  & \tsans{sva-di"s} (\textit{sva-diś}) \\[5pt]
        \end{tabular}}        
}

\newglossaryentry{greater}
{
        name={greater},
        description={\tsans{adhika} (\textit{adhika}) \acrshort{secondary_nominal_derivative} (\acrshort{adjective})\\[5pt]
        \Cf\begin{tabular}[t]{ll}
             \protect\gls{exceeds} & \tfarsi{زیادی شدن} (\textit{\ziyadi\ \shudan})  
        \end{tabular}}
}

\newglossaryentry{ninety}
{
        name={ninety},
        description={\tsans{abhra-nava} (\textit{abhra-nava})\\[5pt]
        digits `0-9', read as `90', in \bhutasamkhya\ word-numerals.\\[5pt]
        \Cf\begin{tabular}[t]{ll}
        \protect\gls{one_quarter} & \tfarsi{ربع} (\rub)
        \end{tabular}}
}

\newglossaryentry{one_hundred_eighty}
{
        name={one hundred and eighty},
        description={\tsans{kha-a.s.ta-bhuu} (\textit{akha-aṣṭa-bhū}\\[5pt]
        digits `0-8-1', read as `180', in \bhutasamkhya\ word-numerals.\\[5pt]
        \Cf \begin{tabular}[t]{ll}
            \protect\gls{one_half} & \tfarsi{نصف دور} (\nisf)
        \end{tabular}}
}

\newglossaryentry{subtraction}
{
        name={subtraction},
        description={\begin{tabular}[t]{ll}
             \textbf{having been subtracted} & \tsans{vi"sodhita} (\textit{viśodhita}) \acrshort{causative}-\acrshort{past}-\acrshort{passive}-\acrshort{participle} 
        \end{tabular}\\[5pt]
        \Cf\begin{tabular}[t]{ll}
             \protect\gls{difference} & \tsans{antara} (\textit{antara}) 
        \end{tabular}}
}


%%%%%%%%%%%%%%%%%%%%%%

\newglossaryentry{shadow}
{
        name={shadow of a gnomon},
        description={\tfarsi{ظلّ} (\zill)\\[5pt]
        \tsans{chaayaa} (\textit{chāyā})}
}

\newglossaryentry{definition}
{
        name={definition},
        description={\tsans{lak.sa.na} (\textit{lakṣaṇa})},
        plural={definitions}
}

\newglossaryentry{Sine}
{
        name={Sine},
        description={\tfarsi{جيب} (\jayb)\\[5pt]
        \tsans{jyaa} (\textit{jyā}); \tsans{jyakaa} (\textit{jyakā}); \tsans{si}\tsnb{ञ्जि}\tsans{nii} (\textit{siñjinī})\\[5pt]
        $\mathcal{R}$\thinspace sine of an arc (in degrees).}
}

\newglossaryentry{Versed_Sine}
{
        name={Versed Sine},
        description={\tsans{"sara} (\textit{śara})\\[5pt]
        Alias:\begin{tabular}[t]{ll}
            \textbf{Sagitta} & \tfarsi{سهم} (\sahm) \\
        \end{tabular}\\[5pt]
        the Versine of an arc (in degrees), \ie ${\Big[}\mathcal{R} - \mathcal{R}$\thinspace cos\thinspace(arc\degree)$\Big]$.\\[5pt]
        \textsc{remark}: the \textbf{Cosine} of an arc, \ie $\mathcal{R}$\thinspace cosine of an arc (in degrees), is called\\[5pt]
        \tfarsi{جيب تمام} (\jayb\idafaconsonant\ \tamam)\\[5pt]
        \tsans{ko.ti-jiivaa} (\textit{koṭi-jīvā}); \tsans{ko.ti-jyaa} (\textit{koṭi-jyā}); \tsans{dyu-jiivaa} (\textit{dyu-jīvā}) [\lit `day-Sine'].}
}

\newglossaryentry{difference_between_two_cells}
{
        name={difference between two cells},
        description={\tsans{dvi-ko.s.tha-antara} (\textit{dvi-koṣṭha-antara})\\[5pt]
        \textit{tacitly}, the difference between the values in two successive rows of an astronomical table\\[5pt]
        \Cf\begin{tabular}[t]{ll}
         \protect\gls{between_two_rows}  &  \tfarsi{ما بين السطرين} (\ma\ \bayn\ \alsatrayn)
        \end{tabular}}
}

\newglossaryentry{demonstration}
{
        name={demonstration},
        description={\tsans{saadhana} (\textit{sādhana}); \tsans{upapatti} (\textit{upapatti})}
}

\newglossaryentry{method_of_identity}
{
        name={method of identity},
        description={\tsans{ananyatva-prakaara} (\textit{ananyatva-prakāra})\\[5pt]
        \textit{tacitly}, an argument following a method identical to one previously stated.}
}

\newglossaryentry{method_of_interpolation}
{
        name={method of interpolation},
        description={\tfarsi{عمل تعدیل} (\amal\idafaconsonant\ \tadil), \lit `operation of adjustment'\\[5pt]
        the method of interpolation to determine intermediate values in an astronomical table.}
}

\newglossaryentry{between_two_rows}
{
        name={between two rows},
        description={\tfarsi{ما بين السطرين} (\ma\ \bayn\ \alsatrayn), \lit `between two lines' or `interlinear'\\[5pt]
        \textit{tacitly}, between the values in two successive rows of an astronomical table.\\[5pt]
         \Cf\begin{tabular}[t]{ll}
         \protect\gls{difference_between_two_cells}  &  \tsans{dvi-ko.s.tha-antara} (\textit{dvi-koṣṭha-antara})
        \end{tabular}}
}

\newglossaryentry{desired_time}
{
        name={desired time},
        description={\tsans{abhimata-samaya} (\textit{abhimata-samaya})\\[5pt]
        Alias:\begin{tabular}[t]{ll}
             \textbf{time}& \tfarsi{وقت} (\vaqt), pl.\thinspace \tfarsi{اوقات} (\avqat)
            \end{tabular}\\[5pt]
        the time of occurrence of an astronomical event.}
}


\newglossaryentry{circle_of_declination}
{
        name={circle of declination},
        description={\tsans{kraanti-suutra} (\textit{krānti-sūtra}), \lit, `the string/cord of declination'\\[5pt]
        a great circle passing through the two celestial poles and a celestial object, and along which the declination of the object is measured.\footnote{\Nityananda\ defines the \gls{circle_of_declination} in the \Siddhantasindhu, Part II.5, verse~1 (f.\thinspace 19: 19--20~Kh):\\[5pt]
        \tsans{svagavi.suvadhruvayugmoparigatamiha bhavati yadv.rttam || tatkraantisuutrasa.m}\tsnb{ज्ञं} \tsans{puurvaacaaryairvinirdi.s.tam ||}}}
}

\newglossaryentry{declination_parts_ecliptic}
{
        name={declination of parts of the ecliptic},
        description={\tfarsi{ميل اجزاء فلك البروج}  (\mayl\idafaconsonant\ \ajza\idafaconsonant\ \falak\ \alburuj)\\[5pt]
        the declination (in degrees) of the different parts of the ecliptic; in other words, the declination of different points on the ecliptic with different ecliptic longitudes.}
}

\newglossaryentry{maximum_elevation_depression_celestial_object}
{
        name={maximum elevation and depression of a celestial object},
        description={\tfarsi{غایت ارتفاع و انخفاض كوكب} (\ghayat\idafaconsonant\ \irtifa\ \va\ \inkhifad\idafaconsonant\ \kawkab)\\[5pt]
        the maximum elevation (above the local horizon) or maximum depression (below the local horizon) of a celestial object measured in degrees.\\[5pt]
        \Cf\begin{tabular}[t]{l}
          \protect\gls{degrees_maximum_elevation} \\
          \tsans{parama-unnata-a.m"sa} (\textit{parama-unnata-aṃśa})\\[5pt]
          \protect\gls{degrees_maximum_depression} \\
          \tsans{adha.hstha-parama-bhaaga} (\textit{adhaḥstha-parama-bhaaga})
          \end{tabular}},
        plural={maximum elevation and depression of celestial objects}
}

\newglossaryentry{elevation}
{
        name={elevation},
        description={\tfarsi{ارتفاع} (\irtifa)\\[5pt]
        Alias:\begin{tabular}[t]{l}
            \textbf{desired degrees of elevation} \\
            \tsans{abhiipsita-unnata-a.m"sa} (\textit{abhīpsita-unnata-aṃśa}); \\
         \tsans{abhii.s.ta-unnata-a.m"sa} (\textit{abhīṣṭa-unnata-aṃśa});  
        \end{tabular}\\[5pt]
        the degrees of the elevation or altitude of a celestial object (above the local horizon), and equal to the complement of its zenith distance.}
}


\newglossaryentry{degrees_maximum_elevation}
{
        name={degrees of the maximum elevation},
        description={\tsans{parama-unnata-a.m"sa} (\textit{parama-unnata-aṃśa})\\[5pt]
        the maximum value of the \protect\gls{elevation}.\\[5pt]
        \Cf\begin{tabular}[t]{l}
          \protect\gls{maximum_elevation_depression_celestial_object}  \\ \tfarsi{غایت ارتفاع و انخفاض كوكب} (\ghayat\idafaconsonant\ \irtifa\ \va\ \inkhifad\idafaconsonant\ \kawkab) 
        \end{tabular}}
}

\newglossaryentry{depression}
{
        name={depression},
        description={\tfarsi{انخفاض} (\inkhifad)\\[5pt]
        Alias:\begin{tabular}[t]{l}
         \textbf{degrees of depression} \\
         \tsans{adhara-a.m"sa} (\textit{adhara-aṃśa}); \tsans{adhara-a.m"saka} (\textit{adhara-aṃśaka})
        \end{tabular}\\[5pt]
        the degrees of the depression of a celestial object (below the local horizon); in other words, the complement of its nadir distance.}
}

\newglossaryentry{degrees_maximum_depression}
{
        name={degrees of the maximum depression},
        description={\tsans{adha.hstha-parama-bhaaga} (\textit{adhaḥstha-parama-bhāga})\\[5pt]
        the maximum value of the \protect\gls{depression}.\\[5pt]
        \Cf\begin{tabular}[t]{l}
          \protect\gls{maximum_elevation_depression_celestial_object}  \\ \tfarsi{غایت ارتفاع و انخفاض كوكب} (\ghayat\idafaconsonant\ \irtifa\ \va\ \inkhifad\idafaconsonant\ \kawkab) 
        \end{tabular}}
}

\newglossaryentry{right_ascension}
{
        name={right ascension of the degrees of the ecliptic (zodiacal signs)},
        description={\begin{tabular}[t]{l}
        \textbf{ascensions [of the ecliptic] at the line of the terrestrial equator}\\
        \tfarsi{مطالع خطّ استوا} (\matali\ \khatt\idafaconsonant\ \istiva)\\[5pt]
           \textbf{ascensions [of the ecliptic] in the right sphere}  \\
           \tfarsi{مطالع فلك مستقیم} (\matali\ \falak\idafaconsonant\ \mustaqim) \\[5pt]
           \textbf{rising [of the zodiacal signs] at the terrestrial equator in degrees}\\ 
           \tsans{vyak.sa-udaya-a.m"sa} (\textit{vyakṣa-udaya-aṃśa})\\[5pt]
           \textbf{rising [of the zodiacal signs] at Laṅkā in degrees} \\
            \tsans{la"nkaa-udaya-a.m"sa} (\textit{laṅkā-udaya-aṃśa})
        \end{tabular} 
        % the perpendicular ascension of an arc of the celestial equator co-arising with an arc of the ecliptic (typically, the zodiacal signs or a fixed amount of degrees of the ecliptic) on the eastern horizon at a place with no latitude, \ie the terrestrial equator.
        }
}

\newglossaryentry{ascendant}
{
        name={ascendant},
        description={\tfarsi{طالع} (\tali), pl.\thinspace \tfarsi{طوالع} (\tawali)\\[5pt]
        \tsans{lagna} (\textit{lagna})
        Alias:\begin{tabular}[t]{ll}
            \textbf{ascendant at each time}& \tfarsi{طالع هر وقت} (\tali\ \har\ \vaqt)\\[5pt]
            \textbf{ascendant at that time} & \tsans{taatkaalika-lagna} (\textit{tātkālika-lagna})\\[5pt]
            \textbf{[ecliptic] degrees of the ascendants} & \tsans{vilagna-a.m"saka} (\textit{vilagna-aṃśaka})
            \end{tabular}\\[5pt]            
        the ascendant (orient ecliptic point) rising on the eastern horizon at a given place and time, expressed as zodiacal sign, degrees, minutes, \etc of ecliptic longitude.},
        plural={ascendants}
}

\newglossaryentry{ascension}
{
        name={ascension},
        description={\tfarsi{مطلع} (\matla), pl.\thinspace \tfarsi{مطالع} (\matali)\\[5pt]
         the ascension or rising of the zodiacal signs (degrees of the ecliptic) on the eastern horizon at a given place and time.},
        plural={ascensions}
}

\newglossaryentry{rising}
{
        name={rising},
        description={\tfarsi{طالع} (\tali), pl.\thinspace \tfarsi{طوالع} (\tawali)\\[5pt]
        \tsans{udaya} (\textit{udaya})\\[5pt]
        Alias:\begin{tabular}[t]{ll}
            \textbf{time of rising} & \tsans{udaya-samaya} (\textit{udaya-samaya})  
        \end{tabular}}
}

\newglossaryentry{setting}
{
        name={setting},
        description={\tfarsi{غروب} (\ghurub)\\[5pt]
        \tsans{asta} (\textit{asta})\\[5pt]
        Alias:\begin{tabular}[t]{ll}
            \textbf{time of setting} & \tsans{asta-samaya} (\textit{asta-samaya})  
        \end{tabular}}
}

\newglossaryentry{oblique_ascension}
{
        name={oblique ascension of the degrees of the ecliptic (zodiacal signs)},
        description={\begin{tabular}[t]{l}
           \textbf{ascensions [of the ecliptic] of a locality}\enskip 
           \tfarsi{مطالع بلد} (\matali\idafaconsonant\ \balad)\\[5pt]
           \textbf{rising times [of the zodiacal signs] in one's own location in degrees} \\
           \tsans{nija-udaya-a.m"sa} (\textit{nija-udaya-aṃśa}); \tsans{nija-udaya-a.m"saka} (\textit{nija-udaya-aṃśaka}); \\\tsans{sva-udaya-a.m"sa} (\textit{sva-udaya-aṃśa})\\[5pt]
           \textbf{rising times in one's own location in degrees}\\
           \tsans{sva-udaya-a.m"sa} (\textit{sva-udaya-aṃśa})\\
           \end{tabular}
        %   the oblique ascension of an arc of the celestial equator co-arising with an arc of the ecliptic (typically, the zodiacal signs or a fixed amount of degrees of the ecliptic) on the eastern horizon at a place (a town/city/country) with non-zero latitude.
           }
}

\newglossaryentry{equation_of_daylight}
{
        name={equation of daylight},
        description={\tfarsi{تعديل النها} (\tadil\ \alnahar)\\[5pt]
        Alias:\begin{tabular}[t]{ll}
            \textbf{ascensional difference} & \tsans{cara} (\textit{cara})
        \end{tabular}
        the arc of the celestial equator corresponding to the difference between the right and oblique ascensions of a celestial object; in other words, the arc of the celestial equator between the six o'clock circle (at a place of given latitude) and the hour circle of a celestial object at rising.
        }
}

\newglossaryentry{arc_of_daylight}
{
        name={arc of daylight},
        description={\tfarsi{قوس النهار} (\qaws\ \alnahar)\\[5pt]
        the arc of the diurnal circle (in degrees) of a celestial object (at a place of given latitude) lying between the rising and setting points on the local horizon and passing through the local meridian.}
}

\newglossaryentry{arc_of_night}
{
        name={arc of night},
        description={\tfarsi{قوس الليل} (\qaws\ \allayl)\\[5pt]
        the arc of the diurnal circle (in degrees) of a celestial object (at a place of given latitude) lying  below the horizon; in other words, the arc of the diurnal circle between the setting and rising points on the local horizon and passing through the local anti-meridian.}
}

\newglossaryentry{hours_of_daylight}
{
        name={hours of daylight},
        description={\tfarsi{ساعات النهار} (\saat\ \alnahar)\\[5pt]
        the measure of the \protect\gls{arc_of_daylight} in hours.}
}

\newglossaryentry{hours_of_night}
{
        name={hours of night},
        description={\tfarsi{ساعات الليل} (\saat\ \allayl)\\[5pt]
        the measure of the \protect\gls{arc_of_night} in hours. \\[5pt]
        \Cf\begin{tabular}[t]{ll}
         \protect\gls{hours_of_day_and_night} & \tsans{dina-raatri-horaa} (\textit{dina-rātri-horā})
        \end{tabular}}
}

\newglossaryentry{oblique_diurnal_circle}
{
        name={oblique diurnal circle},
        description={\tsans{dina-raatri-vaama-v.rtta} (\textit{dina-rātri-vāma-vṛtta}), \lit, the sideways day-and-night circle\\[5pt]
        the diurnal circle of a celestial object at a place with non-zero latitude (a non-equatorial place).}
}

\newglossaryentry{hours_of_day_and_night}
{
        name={hours of day and night},
        description={\tsans{dina-raatri-horaa} (\textit{dina-rātri-horā})\\[5pt]
        \Cf\begin{tabular}[t]{ll}
         \protect\gls{hours_of_daylight} & \tfarsi{ساعات الليل} (\saat\ \allayl)  \\[5pt]
         \protect\gls{hours_of_night} & \tfarsi{ساعات الليل} (\saat\ \allayl)  
        \end{tabular}}
}

\newglossaryentry{inverse_procedure}
{
        name={inverse procedure},
        description={\tsans{viloma-kriyaa} (\textit{viloma-kriyā})\\[5pt]
        Alias:\begin{tabular}[t]{ll}
            \textbf{inverse method} & \tfarsi{عمل عکس} (\amal\idafaconsonant\ \aks)
        \end{tabular}\\[5pt]
       the method of deriving the argument (unknown) from the value (known) by reversing the mathematical relation between the two.}
}


\newglossaryentry{degree_transit_celestial_object}
{       name={[ecliptic] degree of [meridian] transit of a celestial object},
        description={\tfarsi{درجهٔ ممرّ كوكب} (\daraji\idafavowel\ \mamarr\idafaconsonant\ \kawkab)\\[5pt]
        Alias:\begin{tabular}[t]{l}
            \textbf{degrees of the [meridian] ecliptic point at the time of rising}\\
            \textbf{of a celestial object}\enskip \tsans{bha-udaya-lagna-a.m"sa} (\textit{bha-udaya-lagna-aṃśa}) 
        \end{tabular}\\[5pt]
        the ecliptic longitude (in degrees) of a point on the ecliptic that culminates with a celestial object simultaneously; in other words, the longitude of a zodiacal sign (or a point on the ecliptic) that transits the local meridian at the same time as the celestial object.},
        sort={ecliptic degree of meridian transit of a celestial object}
}

\newglossaryentry{ascension_transit}
{
        name={ascensions of [the degrees] of [meridian] transit},
        description={\tfarsi{مطالع  ممرّ} (\matali\idafaconsonant\ \mamarr)\\[5pt]
         Alias:\begin{tabular}[t]{l}
            \textbf{degrees of equatorial ascension of the [meridian] ecliptic point at the}\\
            \textbf{time of rising of a celestial object}\enskip \tsans{bha-udaya-lagna-vyak.sa-udaya-a.m"sa}\\ (\textit{bha-udaya-lagna-vyakṣa-udaya-aṃśa}) 
        \end{tabular}\\[5pt]
        the right ascension corresponding to the \protect\gls{degree_transit_celestial_object}; in other words, right ascension (in degrees) of a zodiacal sign (or a point on the ecliptic) that transits the local meridian at the same time as the celestial object.},
}

\newglossaryentry{rule_of_three}
{
        name={rule of three},
        description={\tsans{trai-raa"sika} (\textit{trai-rāśika}), \lit `relating to three numbers (\textit{rāśi}s)'\\[5pt]
        (in arithmetic) a rule of proportion for calculating an unknown quantity (\textit{icchā-phala} `desired result') from three known qualities (\textit{pramāṇa} `measure', \textit{phala} `result', and \textit{icchā} `desired').}
}

\newglossaryentry{correlated_numbers}
{
        name={correlated numbers},
        description={\tsans{paraspara-sambandhi-raa"si} (\textit{paraspara-sambandhi-rāśi}\\[5pt]
        (in arithmetic) numbers mutually related by proportionality.)}
}

\newglossaryentry{azimuth}
{
        name={azimuth},
        description={\tfarsi{سمت} (\samt)\\[5pt]
        Alias:\begin{tabular}[t]{ll}
        \textbf{degrees of azimuth in one's own location} & \tsans{sva-di"s-a.m"sa} (\textit{sva-diś-aṃśa})\\[5pt]
        \textbf{degree of azimuth} & \tsans{di"s-a.m"sa} (\textit{diś-aṃśa}) 
        \end{tabular}
        % the angular orientation (in degrees) of a great circle passing through the local zenith, a celestial object, and the local nadir; and measured with respect to a cardinal direction of the local horizon.
        }
}

\newglossaryentry{meridian_line}
{
        name={line of the meridian},
        description={\tsans{yaamya-uttara-rekhaa} (\textit{yāmya-uttara-rekhā}), \lit `line going from the south to the north'\\[5pt]
        Alias:\begin{tabular}[t]{ll}
            \textbf{line of midday} & \tfarsi{خطّ نصف النهار} (\khatt\ \nisf\ \alnahar)\\[5pt]
            & \tsans{madhyaahna-rekhaa} (\textit{madhyāhna-rekhaa})
        \end{tabular}\\[5pt]
        the north-south line along the intersection of the perpendicular planes of the horizontal and meridional circles in one's own location.}
}

\newglossaryentry{north_south_direction}
{
        name={northern and southern direction},
        description={\tsans{saumya-yaamya-di"s} (\textit{saumya-yāmya-diś})},
        plural={northern and southern directions}
}



\newglossaryentry{longitude_and_latitude_terrestrial}
{
        name={longitude and latitude of a locality},
        description={\tfarsi{طول و عرض بلد} (\tul\ \va\ \ard\idafaconsonant\ \balad)\\[5pt]
        Alias:\begin{tabular}[t]{l}
            \textbf{degrees of [terrestrial] longitude and latitude}  \\
            \tsans{de"saantara-ak.sa-a.m"sa} (\textit{deśāntara-akṣa-aṃśa}) 
        \end{tabular}
        the terrestrial longitude and latitude in a locality (town/city/country/region).}
}


\newglossaryentry{ecliptic_zenith_distance}
{
        name={zenith distance of the nonagesimal point},
        description={\tsans{d.rkk.sepa} (\textit{dṛkkṣepa})\\[5pt]
        the distance of the nonagesimal point from the local zenith (at a given place and time); in other words, the arc of the vertical to the ecliptic passing through the local zenith and lying between the zenith and the ecliptic (at its nonagesimal or central ecliptic point) at a given place and time.\\[5pt]
        \Cf\begin{tabular}[t]{ll}
          \protect\gls{latitude_visible_climate}  & \tfarsi{عرض اقلیم رؤیت}  (\ard\idafaconsonant\ \iqlim\idafaconsonant\ \ruyat)
        \end{tabular}}
}

\newglossaryentry{ecliptic_pole_zenith_distance}
{
        name={zenith distance of the ecliptic pole},
        description={\tsans{d.rggati} (\textit{dṛggati})\\[5pt]
        the complement of \protect\gls{ecliptic_zenith_distance} to 90\degree\ (at a given place and time); in other words, the distance of the ecliptic pole from the local zenith or the arc of the vertical to the ecliptic passing through the local zenith and lying between the zenith and the ecliptic pole at a given place and time.}
}

\newglossaryentry{latitude_visible_climate}
{
        name={latitude of the visible climate},
        description={\tfarsi{عرض اقلیم رؤیت}  (\ard\idafaconsonant\ \iqlim\idafaconsonant\ \ruyat)\\[5pt]
        the complement of the elevation of the ecliptic above the horizon at a given place and time; in other words, the zenith distance of the nonagesimal point.\\[5pt]
        \Cf\begin{tabular}[t]{ll}
          \protect\gls{ecliptic_zenith_distance}  & \tsans{d.rkk.sepa} (\textit{dṛkkṣepa})
        \end{tabular}}
}

\newglossaryentry{degrees_separation_two_celestial_objects}
{
        name={degrees [of separation] between two celestial objects},
        description={\tsans{dvi-nak.satra-antara-a.m"saka} (\textit{dvi-nakṣatra-antara-aṃśaka})\\[5pt]
        Alias:\begin{tabular}[t]{l}
        \textbf{distance between between two celestial objects} \\ \tfarsi{بعد میان دو کوکب} (\bud\idafaconsonant\ \miyan\idafaconsonant\ \duvum\idafaconsonant\ \kawkab)
        \end{tabular}}
}

\newglossaryentry{determination}
{
        name={determination},
        description={\tfarsi{استخراج} (\istikhraj), \lit `bringing out or extraction'}
}

\newglossaryentry{azimuth_qibla}
{
        name={azimuth of \qibla},
        description={\tfarsi{سمت قبله} (\samt\idafaconsonant\ \qibla)\\[5pt]
        the azimuth of the angle of \qibla\ (direction of \alkabah\ in Mecca), measured with to the northern direction of the \protect\gls{meridian_line} at a terrestrial location.}
}

\newglossaryentry{inclination_azimuth_qibla}
{
        name={inclination of the azimuth of \qibla},
        description={\tfarsi{انحراف سمت قبله} (\inhiraf\idafaconsonant\ \samt\idafaconsonant\ \qibla)\\[5pt]
        Alias:\begin{tabular}[t]{ll}
        \textbf{inclination} & \tfarsi{انحراف} (\inhiraf) 
        \end{tabular}\\[5pt]
        the supplement of \protect\gls{azimuth_qibla} to 180\degree; in other words, the deviation of the angle of \qibla, measure with respect to the southern direction of the \protect\gls{meridian_line} at a terrestrial location.}
}

\newglossaryentry{genus}
{
        name=genus,
        description={\tfarsi{جنس} (\jins)\\[5pt]
        Alias:\begin{tabular}[t]{ll}
        \textbf{belong to a particular genus} & 
        \tsans{jaatiiya} (\textit{jātīya})
        \end{tabular}}
}

\newglossaryentry{square_root}
{
        name={square root},
        description={\tfarsi{جذر} (\jadr)\\[5pt]
        \tsans{muula} (\textit{mūla})}
}

\newglossaryentry{position}
{
        name={position},
        description={\tfarsi{مرتبه} (\martaba), pl.\thinspace \tfarsi{مراتب} (\maratib)\\[5pt]
        \tsans{sthaana} (\textit{sthāna})\\5pt]
        \textit{tacitly}, the rank or degree of the digits in a sexagesimal number; in other words, the place value of the digits.},
        plural={positions}        
}

\newglossaryentry{elevation_sexagesimal}
{
        name={elevated [rank]},
        description={\tfarsi{مرفوع} (\marfu), pl.\thinspace \tfarsi{مرفوعات} (\marfuat), \lit `raised up'\\[5pt]
        Alias:\begin{tabular}[t]{ll}
        \textbf{revolution}  & \tsans{parivarta} (\textit{parivarta}) 
        \end{tabular}\\[5pt]
        \textit{tacitly}, the place values higher than degrees in a sexagesimal number; in other words, 
        place values of the order of 60\textsuperscript{\textit{n}} with \textit{n} =1,\,2,\,3\dots. Also identified with the position of integer-revolutions (in sexagesimal numbers) for angular measures of arc.},
        plural={elevated [ranks]}
}

\newglossaryentry{degree}
{
        name={degree},
        description={\tfarsi{درج} (\daraj)\\[5pt]
        \textit{tacitly}, the first integer place value of a sexagesimal number.}
}

\newglossaryentry{parts_of_degree}
{
        name={[fractional] parts of a degree},
        description={\tfarsi{اجزاء درج} (\ajza\idafaconsonant\ \daraj)},
        sort={fractional parts of a degree}
}

\newglossaryentry{minute}
{
        name={minute},
        description={\tfarsi{دقيقه} (\daqiqa), pl.\thinspace \tfarsi{دقایق} (\daqaiq)\\[5pt]
        \tsans{kalaa} (\textit{kalā})\\[5pt]
        the first fractional place value of a sexagesimal number or a unit of time.},
        plural={minutes}
}

\newglossaryentry{second}
{
        name={second},
        description={\tfarsi{ثانيه} (\thaniya), pl.\thinspace \tfarsi{ثواني} (\thawani)\\[5pt]
        the second fractional place value of a sexagesimal number or a unit of time.},
        plural={seconds}
}


\newglossaryentry{digit}
{
        name={digit},
        description={\tsans{a"nka} (\textit{aṅka})\\[5pt]
        \textit{polysemously} a digit/number, a mark/sign, a curve/arc, \etcp.},
        plural={digits}
}